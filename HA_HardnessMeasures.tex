% Hoda Abbasi
% 16 March 2016

\documentclass[]{book}
\input Latex_macros/Definitionen.tex
\usepackage{a4}
\usepackage{graphicx}
\usepackage{tikz-qtree}
\usepackage[active]{srcltx}
\usepackage[all]{xy}
\usepackage{enumerate}

\setcounter{tocdepth}{4}
\setcounter{secnumdepth}{4}
\newcommand{\Schrift}{report}

\begin{document}
\title{Hardness measures of clause-sets}

\author{Hoda Abbasi\\
        PhD Candidate\\
        Computer Science Department\\
        Swansea University\\}
\maketitle

\tableofcontents
%%%%%%%%%%%%%%%%%%%%%%%%%%%%%%%%%%%%%%%%%%%%%%%%%%%%%%%%%%
\chapter{Introduction}
\label{cha:Introduction}

The aim of this report is to investigate different hardness measures for a clause-set in the literature and then, to propose a method based on Prover-Delayer games. 

The report is organized as follows. In Chapter \ref{cha:Preliminaries} some general preliminaries are presented. Then, in Chapter \ref{cha:Hardness Measures} hardness is defined as a measure for complexity of unsatisfiable clause-set. This measure, then is extended for arbitrary clause-set. Chapter \ref{cha:XOR-constraints} explains a system of XOR-constraint and ??????

In Chapter \ref{cha:hdgame}, the proposed method to obtain the hardness of a clause-set is given. %Finally, Chapter \ref{cha:concl} presents some questions regarding hardness measures and the hardness game and then, it concludes the report.
%%%%%%%%%%%%%%%%%%%%%%%%%%%%%%%%%%%%%%%%%%%%%%%%%%%%%%%%%%
\chapter{Preliminaries}
\label{cha:Preliminaries}

\section{Clause-sets}
\label{sec:Clause-sets}

The infinite set of variables is denoted by $\Va$. A partial assignment $\vp$ is a map which assigns a unique value in $\{0,1\}$ to each elemet of a finite set of variables. The domain of this map is a set of variables denoted by $\var(\vp)$. The set of all partial assignments is indicated by $\Pass$ and for $V \in \pote(\Va)$, $\Tass(V)$ is the set of total assignments over $V$. The empty partial assignment is denoted by $\epa:= \es \in \Pass$. For a partial assignment, the number of variables is defined as $n(\vp):=\abs{\var(\vp)}$. For two partial assignments $\vp, \psi \in \Pass$, the composition operation is defined as $\vp \circ \psi := \psi \cup (\vp \sm \var(\psi)) \in \Pass$ which is the union of their variables if they do not conflict. In the case of conflicting variables, the second assignment is considered. This operation has associative property and it is commutative if $\vp, \psi$ do not clash.

A literal is a pair $(v,\ve)$ with $v \in \Va$ and $\ve \in \{0,1\}$. The set of all literals is $\Lit$. Two literals $x, y \in \Lit$ clash if they have a same variable but different values. For a set $L \sse \Lit$ we define $\var(L) := \set{\var(x) : x \in L}$, $\lit(L):= \set{x \in \Lit : \var(x) \in \var(L)}$ and $\ol{L} := \lit(L) \sm L$. A clause is defined as a finite and clash-free set of literals. The set of all clauses is denoted by $\Cl$ and $\bot := \es \in \Cl$ is the empty clause. 

A clause-set is a finite set of clauses and $\Cls$ is the set of all clause-sets. The empty clause-set is indicated by $\top := \es \in \Cls$. For $F \in \Cls$ we define $\var(F) := \bc_{C \in F} \var(C) \in \pote(\Va)$ and $\lit(F) := \var(F) \cup \ol{\var(F)}$. The number of variables in $F$ is denoted by $n(F) := \abs{\var(F)} \in \NNZ$ and the number of clauses is $c(F) := \abs{F} \in \NNZ$. The number of literal occurrences in $F$ is also denoted by $\ell(F) := \sum_{C \in F} \abs{C} \in \NNZ$. If literals occuring in $F$ are indicated by $\bigcup F \subset \Lit$, then a pure literal for $F \in \Cls$ is defined as $x \in \bigcup F$ and $\ol{x} \not \in \bigcup F$. A full clause $C$  for a clause-set $F$ is defined as $C \in F$ and $\var(C) = \var(F)$, and a clause-set $F$ is called full if every clause is full. The full clause-set for a finite set $V$ of variables is denoted by $A(V)$.

Partial assignments are extended from variables to literals using $\vp(\ol{v})=\ol{\vp(v)}$. The operation of a partial assignment on a clause-set is defined as $\vp * F := \set{C \sm \vp : C \in F \wedge C \cap \ol{\vp} = \es} \in \Cls$. This means removing all clauses having at least one literal with $\vp (x)=1$ and then, removing from all clauses the literals with $\vp (x)=0$. A clause-set $F$ is called satisfiable if there exists a partial assignment $\vp$ that $\vp * F = \top$. The set of all satisfiable clause-sets is indicated by $\Sat := \set{F \in \Cls \mb \ex\, \vp \in \Pass : \vp * F = \top}$ while $\Usat := \Cls \sm \Sat$. If $\vp * F = \top$ then the partial assignment is called a satisfying assignment for $F$.

\section{Implication-relation}
\label{sec:imprel}

The implication-relation for two clause-sets $F, F'$ is defined as $F \models F'$ if $\fa\, \vp \in \Pass : \vp * F = \top \Ra \vp * F' = \top$. This relation can also be considered between a clause $C$ and a clause-set $F$ if $F \models \set{C}$ (which is indicated as $F \models C$). For  $F \in \Usat$ the only implication-relation is $F \models \bot$.

A clause $C$ is called an implicate of a clause-set $F$ if $F \models C$ and it is called a prime implicate if there no $ C' \sse C$ as an implicate of $F$. The set of all prime implicates of £F£ is denoted by $\primec_0(F) \in \Cls$. A clause $C$ is called an implicant of a clause-set $F$ if $C$ as a partial assignment is a satisfying assignment for $F$. In other words, $C$ must fulfill $C * F=\top$. A prime implicant is a minimal implicant and the set of all prime implicants is indicated by $\primec_1(F) \in \Cls$. For example, for $F \in \Usat$ we have $\primec_0(F) = \set{\bot}$ and $\primec_1(F) = \top$.

Two clause-sets $F, G$ are called logically equivalent if $F \models G$ and $G \models F$. In this case, we have $\primec_0(F) = \primec_0(G)$ and $\primec_1(F) = \primec_1(G)$.

\section{Resolution}
\label{sec:Resolution}

The resolution is an operation applied to two clauses $C,D$ which clash in exactly one variable and produce a new clause. The result of the resolution for $C \cap \overline D = \{ x \}$ which is called resolvent is defined as $C \diamond D := (C \cup D) \setminus \{x, \overline x\} $. $C,D$ are called resolvable clauses and $x$ is called the resolution literal.

Using resolution operation, a resolution tree $T$ is produced for a clause-set $F$. The tree is indicated by $T : F \vdash C$ which clause $C$ is the root (conclusion) of $T$. The leaves of $T$ (axioms) are the clauses of $F$ and each inner node is the resolvent of its two parents. The number of nodes in $T$ is called tree-resolution complexity and denoted by $\comptr(R) \in \NN$. A resolution proof of a clause $C$ from a clause-set $F$ is a resolution tree $T : F \vdash C$ and a resolution refutation of $F$ is a resolution proof that drives $\bot$.

\section{Generalised unit-clause-propagation}
\label{sec:rkred}

\section{Horton-Strahler number}
\label{sec:hs}

The Horton-Strahler number is a measure of branching complexity for trees. Here, the Horton-Strahler number is defined for a resolution tree $T$ and denoted by $\hts(T) \in \NNZ$. To obtain $\hts(T)$, we start with leaves (axioms) whose Horton-Strahler number are defined as $\hts(T) := 0$. Then, for each inner node with two children $T_1, T_2$, we have two cases. If $\hts(T_1)= \hts(T_2)$, then $\hts(T) := \hts(T_1)+ 1$. Otherwise, $\hts(T) := \max(\hts(T_1),\hts(T_2))$.

\section{Hypergraph}
\label{sec:hpg}

A hypergraph is a pair $G=(V,E)$, where $V$ is a set and $E$ is a set of finite subsets of $V$. The set $V$ is called vetices and the set $E$ is called hyperedges. For example, $V=\{ v_1, v_2, v_3 \}$ with $E=\{ e_1, e_2, e_3\} = \{ \{ v_1\}, \{v_2, v_3\}, \{ v_1, v_2, v_3\}\}$ is a hypergraph.

In the case that loops and parallel edges are allowed, a general graph is indicated by $(V,E,\eta)$ which $\eta$ is a map defined as $\eta : E \ra \set{ e \sse V : 1 \leq \abs{e} \leq 2}$.
%%%%%%%%%%%%%%%%%%%%%%%%%%%%%%%%%%%%%%%%%%%%%%%%%%%%%%%%%%
\chapter{Hardness measures}
\label{cha:Hardness Measures}

%%%%%%%%%%%%%%%%%%%%%%%%%%%%%%%%%%%%%%%%%%%%%%%%%%%%%%%%%%
\chapter{XOR representation}
\label{cha:XOR-representation}


%%%%%%%%%%%%%%%%%%%%%%%%%%%%%%%%%%%%%%%%%%%%%%%%%%%%%%%%%%
\chapter{The hardness game}
\label{cha:hdgame}

%%%%%%%%%%%%%%%%%%%%%%%%%%%%%%%%%%%%%%%%%%%%%%%%%%%%%%%%%%
%\chapter{Conclusion and open problems}
%\label{cha:concl}

%%%%%%%%%%%%%%%%%%%%%%%%%%%%%%%%%%%%%%%%%%%%%%%%%%%%%%%%%%
\newpage
\bibliographystyle{plainurl}
\bibliography{HA_ResearchRef}

\end{document}

