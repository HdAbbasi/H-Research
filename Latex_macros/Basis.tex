% 23.12.1999
%
% Die Basis fuer Folien wie fuer normale Texte.
%
%
\input Latex_macros/Bibliographie.tex
\input Latex_macros/Adressen.tex
%
% -----------------------------------------------------------------------------------------------------------------
% ENVIRONMENT (Latex-Umgebung)
% -----------------------------------------------------------------------------------------------------------------
%
\scrollmode
%
\usepackage{amsmath}
\usepackage{amsfonts}
\usepackage{amssymb}
\usepackage{latexsym}
\usepackage{stmaryrd}
\usepackage{array}
\usepackage{exscale}
%
%\renewcommand{\baselinestretch}{1.0}
%
% ########################################################
% ----------------------------------------------------------------------------------------------------------------
% GENERAL CONSTRUCTS (Allgemeine Konstrukte)
% -----------------------------------------------------------------------------------------------------------------
% ########################################################
%
\newcommand{\nc}{\newcommand}
\newcommand{\ol}{\overline}
\newcommand{\ul}{\underline}
\newcommand{\es}{\emptyset}
\newcommand{\sm}{\setminus}
\newcommand{\ve}{\varepsilon}
\newcommand{\vp}{\varphi}
\newcommand{\bw}{\bigwedge}
\newcommand{\bv}{\bigvee}
\newcommand{\bc}{\bigcup}
\newcommand{\bca}{\bigcap}
\newcommand{\Lra}{\Leftrightarrow}
\newcommand{\Lora}{\Longrightarrow}
\newcommand{\Lla}{\Longleftarrow}
\newcommand{\Llra}{\Longleftrightarrow}
\newcommand{\Ra}{\Rightarrow}
\newcommand{\La}{\Leftarrow}
\newcommand{\ra}{\rightarrow}
\newcommand{\lora}{\longrightarrow}
\newcommand{\la}{\leftarrow}
\newcommand{\lra}{\leftrightarrow}
\newcommand{\da}{\downarrow}
\newcommand{\ub}{\underbrace}
\newcommand{\ob}{\overbrace}
\newcommand{\sst}{\subset}
\newcommand{\sse}{\subseteq}
\newcommand{\spt}{\supset}
\newcommand{\spe}{\supseteq}
\newcommand{\fa}{\forall}
\newcommand{\ex}{\exists}
\newcommand{\mr}{\mathrm}
\newcommand{\mc}{\mathcal}
\newcommand{\mf}{\mathfrak}
\newcommand{\vtr}{\vartriangleright}
\newcommand{\trd}{\triangledown}
\newcommand{\DMO}{\DeclareMathOperator}
%
\newcommand{\DST}{\displaystyle}
\newcommand{\TST}{\textstyle}
\newcommand{\SST}{\scriptstyle}
\newcommand{\SSST}{\scriptscriptstyle}
%
\newcommand{\NN}{\mathbb{N}}
\newcommand{\NNZ}{\NN_0}
\newcommand{\ZZ}{\mathbb{Z}}
\newcommand{\QQ}{\mathbb{Q}}
\newcommand{\RR}{\mathbb{R}}
\newcommand{\CC}{\mathbb{C}}
%
\newcommand{\AAM}{\mathbb{A}}
\newcommand{\BB}{\mathbb{B}}
\newcommand{\DD}{\mathbb{D}}
\newcommand{\EE}{\mathbb{E}}
\newcommand{\FF}{\mathbb{F}}
\newcommand{\GG}{\mathbb{G}}
\newcommand{\HH}{\mathbb{H}}
\newcommand{\KK}{\mathbb{K}}
\newcommand{\LL}{\mathbb{L}}
\newcommand{\MM}{\mathbb{M}}
\newcommand{\OO}{\mathbb{O}}
\newcommand{\PP}{\mathbb{P}}
\newcommand{\SSM}{\mathbb{S}}
\newcommand{\TT}{\mathbb{T}}
\newcommand{\UU}{\mathbb{U}}
\newcommand{\VV}{\mathbb{V}}
\newcommand{\WW}{\mathbb{W}}
%
% Lists in mathematical textmode with (appropriate) linebreaks:
\mathchardef\breakingcomma\mathcode`\,
{\catcode`,=\active
  \gdef,{\breakingcomma\discretionary{}{}{}}
}
\newcommand{\mathlist}[1]{$\mathcode`\,=\string"8000 #1$}
%
% Alte Beweisumgebungen, nun obsolet (anstelle verwende man die prf/prfd-
% Umgebung mit \Qed):
\newcommand{\Ende}{\ \rule{0.4em}{1.7ex}}
\newcommand{\pr}{\noindent\textbf{Proof:}\quad}
\newcommand{\prd}{\noindent\textbf{Beweis:}\quad}
%%
\newcommand{\ha}{\hspace*{5mm}}
\newcommand{\hb}{\hspace*{10mm}}
\newcommand{\hc}{\hspace*{15mm}}
\newcommand{\hd}{\hspace*{20mm}}
\newcommand{\he}{\hspace*{25mm}}
%
\newcommand{\mar}[1]{\makebox[0cm]{#1}}
%
% 16.10.2003
\newcommand{\NAA}{\setlength{\itemsep}{0pt}} % Null Aufzaehlungs-Abstand
%
% Tilde-Symbol ~ ("richtig"); fragiles Kommando (wg. \raisebox)
% Do not use within the \url command --- there just use "~" !
\newcommand{\rilde}{{\large\raisebox{-1ex}[0mm][0mm]{\~{}}\hspace{-0.05em}}}
%
% Deutsche Quotierung (30.11.2003); alter Makro-Name war ``\Quot''
\newcommand{\Qu}[1]{\glqq{}{#1}\grqq{}}
%
% Fuer Diagramme mittels xymatrix; for diagrams using xymatrix:
\newcommand{\aru}{\ar @{-}} % ungerichtete Kante; undirected edge
% gerichtet: einfach \ar[...]; for a directed edge just use \ar[...]
\newcommand{\arug}{\ar @{--}} % gestrichelte ungerichtete Kante; dashed undirected edge
\newcommand{\arde}{\ar @{->>}} % ``epimorphe'' gerichtete Kante; epimorphic directed edge
\newcommand{\ardm}{\ar @{^(->}} % ``monomorphe'' gerichtete Kante; monomorphic directed edge
\newcommand{\ardg}{\ar @{-->}} % gestrichelte gerichtete Kante; dottet directed edge
\newcommand{\arddg}{\ar @{<-->}} % gestrichelte beidseitig-gerichtete Kante; dashed edge directed in both directions
%
% Wiedergabe von Programmen
\usepackage{listings}
\lstloadlanguages{Pascal,C++,Java}
\newcommand{\Pascal}{\lstset{language=Pascal,keywordstyle=\bfseries,breaklines,breakindent=30pt}}
\newcommand{\Cpp}{\lstset{language=C++,keywordstyle=\bfseries,breaklines,breakindent=30pt}}
\newcommand{\Java}{\lstset{language=Java,keywordstyle=\bfseries,breaklines,breakindent=30pt}}
\newcommand{\inl}[1]{\lstinline$#1$}
%
% ########################################################
% ----------------------------------------------------------------------------------------------------------------
% SETS AND BASIC LOGIC (Mengen und elementare Logik)
% -----------------------------------------------------------------------------------------------------------------
% ########################################################
%
\newcommand{\und}{{\:\wedge\:}} % and
\newcommand{\oder}{{\:\vee\:}} % or
%
\newcommand{\mb}{{\:|\:}} % Mengenbildner; set creation
\newcommand{\set}[1]{\{ #1 \}}
\newcommand{\setb}[1]{\big \{ \, #1 \, \big \}}
%
\DeclareMathOperator{\dom}{dom}
\DeclareMathOperator{\id}{id}
\DeclareMathOperator{\cod}{cod} % Codomain (28.11.2003)
\DeclareMathOperator{\rg}{rg} % Wertemenge ("range")
\DeclareMathOperator{\tcomp}{\trans{\circ}} % transposed composition of maps%
\DeclareMathOperator{\simrv}{\,\sim\hspace{-0.05em}}
\DeclareMathOperator{\simlv}{\!\sim\,}
\nc{\simlvi}[1]{\!\sim_{#1}}
\DeclareMathOperator{\rstr}{|} % Einschraenkung; restriction
%
% Kardinalitaeten (cardinalities):
\DeclareMathOperator{\card}{card}
%
% Projektionen und Injektionen; projections and injections
\DeclareMathOperator{\proj}{pr}
\DeclareMathOperator{\inj}{in}
%
% 1.1.2004 Mengenoperationen
%
\newcommand{\rbca}[1]{\bca_{#1}\nolimits} % relative intersection
% ToDo: How to produce \rbca{X}_{i \in I} ?! (\sideset is too restricted,
% since the symbol will always be in displaystyle, and \DeclareMathOperator* doesn't allow
% arguments).
\DeclareMathOperator*{\rbcaX}{\bca_X\nolimits}
\DeclareMathOperator{\symdif}{\vartriangle} % symmetrische Differenz
\DeclareMathOperator{\addcup}{{\stackrel{\text{\raisebox{-2.2ex}[-0ex][-0ex]{\large$\cdot$}}}{\cup}}} % Vereinigung, die disjunkt ist; disjoint union
\DeclareMathOperator*{\addbcup}{{\stackrel{\text{\raisebox{-4.2ex}[-0ex][-0ex]{\Large$\cdot$}}}{\bigcup}}} % entsprechend grosse Vereinigung, die disjunkt ist; big disjoint union
\nc{\apprel}[3]{{#1}(#2)_{(#3)}} % \apprel{R}{M}{k} : Anwendung der Relation auf M bzgl. Stufe k; application of relation R to set M at level k
%
\DeclareMathOperator{\Rel}{\mf{REL}} % Menge aller binaeren Relationen
\DeclareMathOperator{\Abb}{\mf{MAP}} % Menge der Abbildungen
\DeclareMathOperator{\Abbi}{\mf{MAP}_i} % Menge der injektiven Abbildungen
\DeclareMathOperator{\Abbs}{\mf{MAP}_s} % Menge der surjektiven Abbildungen
\DeclareMathOperator{\Tra}{\mf{T}} % Menge der Transformationen
\DeclareMathOperator{\Per}{\mf{S}} % Menge der Permutationen
\DeclareMathOperator{\Pert}{\Per_t} % Menge der transitiven Permutationen
\DeclareMathOperator{\Ptr}{\mf{PT}} % partielle Transformationen
\DeclareMathOperator{\fix}{fix} % Menge der Fixpunkte einer Transformation
\DeclareMathOperator{\Peri}{\Per_i} % Menge der injektiven Transformationen; set of injective transformations
\DeclareMathOperator{\Pers}{\Per_s} % Menge der surjektiven Transformationen; set of surjective transformations
%
% Spezielle Relationen
%
\DeclareMathOperator{\Rrel}{\Rel_r} % reflexive Relationen
\DeclareMathOperator{\Srel}{\Rel_s} % symmetrische Relationen
\DeclareMathOperator{\Trel}{\Rel_t} % transitive Relationen
%
\newcommand{\gef}[1]{{#1}^{((*))}} % Menge aller geordneten endlichen Folgen
\newcommand{\ef}[1]{{#1}^{(*)}} % Menge aller endlichen Folgen
\newcommand{\stdf}[1]{{#1}^{*}} % Menge aller endlichen Standardfolgen
\newcommand{\stdfn}[1]{{#1}^{+}} % Menge aller endlichen nichtleeren Standardfolgen
\DeclareMathOperator{\konkat}{\sqcup} % Konkatenation (23.10.2006)
%
% Operationen fuer Mengensysteme; operations for set-systems
%
\DeclareMathOperator{\cmpl}{\complement^1} % Komplement elementweise fuer Mengensysteme; elementwise complement for set-systems
\nc{\cmpli}[1]{\complement^1_{#1}} % mit einem Index; with an index
\DeclareMathOperator{\cmplz}{\complement^0} % Komplement bzgl. einer Grundmenge; complement w.r.t. a base-set
\nc{\cmplzi}[1]{\complement^0_{#1}} % mit einem Index; now with an index
\DeclareMathOperator{\cmplzo}{\complement^*} % Komposition von compl und complz; composition of compl and complz
\nc{\cmplzoi}[1]{\complement^*_{#1}} % mit einem Index; with an index
\DeclareMathOperator{\fsigma}{{\mf{F}}_{\sigma}} % Abschluss unter abzaehlbaren Vereinigungen
\newcommand{\gdelta}[2]{{\mf{G}}_{\sigma}^{#1}(#2)} % Abschluss unter abzaehlbaren Durchschnitten
\DeclareMathOperator{\gdeltao}{\mf{G}_{\sigma}}
\DeclareMathOperator{\fs}{{\mf{F}}_{s}} % Abschluss unter endlichen Vereinigungen
\DeclareMathOperator{\fss}{{\mf{F}}_{s}^*} % Abschluss unter endlichen nichtleeren Vereinigungen
\newcommand{\gd}[2]{\mf{G}_{\mr{d}}^{#1}(#2)} % Abschluss unter abzaehlbaren Durchschnitten
\newcommand{\gds}[1]{\mf{G}_{\mr{d}}^*(#1)} % Abschluss unter abzaehlbaren nichtleeren Durchschnitten
\newcommand{\gsigma}[2]{\sigma_{#1}(#2)} % erzeugte sigma-Algebra; generated sigma-algebra
\newcommand{\gsigmar}[1]{\sigma_{\mr{r}}(#1)} % erzeugter sigma-Ring; generated sigma-ring
% 
% Speziell zur axiomatischen Mengenlehre:
%
\nc{\zf}{\mr{ZF}}
\nc{\zfmf}{\zf^0} % minus Fundierung
\nc{\zfc}{\mr{ZFC}}
\nc{\zfcmf}{\zfc^0} % minus Fundierung
\nc{\bst}{\mr{BST}} % Bourbaki set theory
%
% ########################################################
% ----------------------------------------------------------------------------------------------------------------
% NUMBERS (Zahlen)
% -----------------------------------------------------------------------------------------------------------------
% ########################################################
%
% Zahlenmengen (sets of numbers):
\newcommand{\nni}{\NNZ \cup \{+\infty\}} % natuerliche Zahlen mit 0 und unendlich; natural numbers with 0 and infinity
\newcommand{\nnpi}{\NN \cup \{+\infty\}} % natuerliche Zahlen mit unendlich; natural numbers with infinity ("p" for "positive")
% Abschluss von R (7.1.2004):
\newcommand{\rri}{\ol{\RR}} % RR + -+unendlich
\newcommand{\cci}{\ol{\CC}} % CC + unendlich (die Riemannsche Zahlkugel)
%
% 11.7.2003: Teilbereiche
\newcommand{\tb}[2]{\set{#1, \dots, #2}} % Teilbereich
%
% Realteil und Imaginaerteil (8.1.2004)
\DeclareMathOperator{\re}{Re}
\DeclareMathOperator{\im}{Im}
%
\DeclareMathOperator{\sgn}{sgn} % Vorzeichen; sign
%
\providecommand{\abs}[1]{\lvert #1 \rvert} % 29.8.2000
\providecommand{\norm}[1]{\lVert #1 \rVert} % 29.11.2000
%
\DeclareMathOperator{\ld}{ld} % logarithmus dualis; logarithm to base 2
\DeclareMathOperator{\fld}{fld} % floor of logarithm dualis
%
% ########################################################
% ----------------------------------------------------------------------------------------------------------------
% MATRICES (Matrizen)
% -----------------------------------------------------------------------------------------------------------------
% ########################################################
%
%
\DeclareMathOperator{\rank}{rank} % 1.7.2000
\providecommand{\inprod}[1]{\left\langle #1 \right\rangle} % Skalarprodukt
%
\DeclareMathOperator{\Q}{\mf{Q}} % die ``qualitive Klasse'' einer Matrix; the ``qualitative class'' einer Matrix
%
% 6.9.2001: Menge aller reellen Matrizen
\DeclareMathOperator{\M}{\mc{M}} % deaktiviert 13.1.2007
% \DeclareMathOperator{\Mpm}{\M(\set{-1,0,+1})} % deaktiviert 13.1.2007
%
\DeclareMathOperator{\Mat}{\mc{M}} % Menge aller Matrizen
\DeclareMathOperator{\Rows}{R} % Zeilenmenge
\DeclareMathOperator{\Columns}{C} % Spaltenmenge
\DeclareMathOperator{\tr}{tr} % Spur
\newcommand{\trans}[1]{#1^{\hspace{0.05em}\mr{t}}} % Transposition
\DeclareMathOperator{\Gl}{GL} % lineare Gruppe; linear group
\DeclareMathOperator{\Sl}{SL} % spezielle lineare Gruppe; special linear group
\DeclareMathOperator{\Orth}{\mc{O}} % orthogonale Gruppe
\DeclareMathOperator{\per}{per} % Permanente
%
% ########################################################
% ----------------------------------------------------------------------------------------------------------------
% ORDER THEORY (Ordnungstheorie)
% -----------------------------------------------------------------------------------------------------------------
% ########################################################
%
% Intervalle (20.12.2000): ``a'' fuer abgeschlossen, ``o'' fuer offen.
% Die *-Versionen erlauben variable Groesse.
\makeatletter
\DeclareRobustCommand{\genericinterval}[2]{%
  \@ifstar{\genericinterval@star{#1}{#2}}{\genericinterval@nostar{#1}{#2}}}
\newcommand{\genericinterval@star}[4]{\mathopen{}\mathclose{\left#1#3,#4\right#2}}
\newcommand{\genericinterval@nostar}[4]{\mathopen{#1}#3,#4\mathclose{#2}}
\newcommand{\iaa}{\genericinterval[]}
\newcommand{\ioo}{\genericinterval][}
\newcommand{\ioa}{\genericinterval]]}
\newcommand{\iao}{\genericinterval[[}
\makeatother
%
\nc{\untit}[2]{{#1}^{#2 \downarrow}} % untere Iterierte
\nc{\obit}[2]{{#1}^{#2 \uparrow}} % obere Iterierte
%
\DeclareMathOperator{\inttop}{\tau_{\mr{O}}} % (offene) Intervalltopologie; open interval topology
\DeclareMathOperator{\rointtop}{\tau_+} % rechts-offene Intervalltopologie
\DeclareMathOperator{\lointtop}{\tau_-} % links-offene Intervalltopologie
%
\DeclareMathOperator{\sid}{\mc{IDL}} % Menge der Ideale in einem Verband; set of ideals in a lattice
\DeclareMathOperator{\skid}{\mc{CID}} % Menge der Koideale; set of co-ideals
\DeclareMathOperator{\smid}{\sid_m} % Menge der maximalen echten Ideale; set of maximal proper ideals
\DeclareMathOperator{\smkid}{\skid_m} % Menge der maximalen echten Koideale; set of maximal proper co-ideals
%
\DeclareMathOperator*{\ordsum}{+} % ordinale Summe; ordinal sum
\DeclareMathOperator{\lexprod}{\times_{\mathrm{l}}} % lexikographisches Produkt; lexicographical product
\DeclareMathOperator*{\lexprodb}{\sideset{}{^\mathrm{l}}\prod\hspace*{-0.25em}} % grosse Form; big form
\DeclareMathOperator{\colexprod}{\times_{\mathrm{c}}} % kolexikographisches Produkt; colexicographical product
\DeclareMathOperator*{\colexprodb}{\sideset{}{^\mathrm{c}}\prod\hspace*{-0.25em}}

%
% ########################################################
% ----------------------------------------------------------------------------------------------------------------
% GRAPH THEORY (Graphentheorie)
% -----------------------------------------------------------------------------------------------------------------
% ########################################################
%
\DeclareMathOperator{\nachbarn}{\Gamma}
\DeclareMathOperator{\enachbarn}{N}
\DeclareMathOperator{\nachbarnr}{\Gamma_{\!\mr{r}}}
\DeclareMathOperator{\nachbarnz}{\widetilde{\Gamma}}
\DeclareMathOperator{\nachbarnzr}{\widetilde{\Gamma}_{\!\mr{r}}}
%
% Incidence matrices (Inzidenzmatrizen)
%
\DeclareMathOperator{\inzEK}{\mc{I}^{\mr{V}}}
\DeclareMathOperator{\inzEKe}{\mc{I}^{\mr{V}}_1}
\DeclareMathOperator{\inzEKz}{\mc{I}^{\mr{V}}_2}
\nc{\inzEKi}[1]{\mc{I}^{\mr{V}}_{#1}}
\DeclareMathOperator{\inzKE}{\mc{I}^{\mr{E}}}
\DeclareMathOperator{\inzKEe}{\mc{I}^{\mr{E}}_1}
\DeclareMathOperator{\inzKEz}{\mc{I}^{\mr{E}}_2}
\nc{\inzKEi}[1]{\mc{I}^{\mr{E}}_{#1}}
\DeclareMathOperator{\inz}{I}
\DeclareMathOperator{\tinz}{\trans{\inz}} % transponierte Inzidenzmatriz; transposed incidence matrix
%
% Adjacency matrices (Adjazenzmatrizen)
%
\DeclareMathOperator{\adjE}{\mc{A}^{\mr{V}}}
\DeclareMathOperator{\adjEe}{\mc{A}^{\mr{V}}_1}
\DeclareMathOperator{\adjEz}{\mc{A}^{\mr{V}}_2}
\nc{\adjEi}[1]{\mc{A}^{\mr{V}}_{#1}}
\DeclareMathOperator{\adjor}{\mc{A}_{\mr{S}}} % orientierte Adjazenzmatrix
\DeclareMathOperator{\adjK}{\mc{A}^{\mr{E}}}
\DeclareMathOperator{\adj}{A}
%
% Ranks and degrees
%
% \deg ist schon Latex
\DeclareMathOperator{\degmin}{\mu\!\deg}
\DeclareMathOperator{\degmax}{\nu\!\deg}
\DeclareMathOperator{\degdur}{\widetilde{\deg}} % Durchschnitt
\DeclareMathOperator{\ideg}{idg} % Innengrad
\DeclareMathOperator{\odeg}{odg} % Aussengrad
\DeclareMathOperator{\degmaxl}{\nu\!\deg_{<}}
\DeclareMathOperator{\degl}{\deg_{<}}
%
\DeclareMathOperator{\rankmin}{\mu\!\rank}
\DeclareMathOperator{\rankmax}{\nu\!\rank}
\DeclareMathOperator{\rankdur}{\widetilde{\rank}}
\DeclareMathOperator{\rankmaxl}{\nu\!\rank_{<}}
\DeclareMathOperator{\rankl}{\rank_{<}}
% 
% Graphparameter
%
\DeclareMathOperator{\vertexcon}{\kappa} % Eckenzusammenhangsgrad
\DeclareMathOperator{\edgecon}{\lambda} % Kantenzusammenhangsgrad
\DeclareMathOperator{\treewidth}{tw} % Baumweite
\DeclareMathOperator{\girth}{g} % Taillenweite
\DeclareMathOperator{\circumference}{cf} % Umfang
\DeclareMathOperator{\length}{lgth} % Laenge
\DeclareMathOperator{\npm}{\Phi} % number of perfect matchings
%
\DeclareMathOperator{\concomp}{cc} % connected components
\DeclareMathOperator{\nconcomp}{ncc} % number of connected components
%
\DeclareMathOperator{\indprim}{ip} % Primitivitaetsindex
\DeclareMathOperator{\indimprim}{iip} % Imprimitivitaetsindex
%
% Spezielle Graphen
\DeclareMathOperator{\bouquet}{B}
\DeclareMathOperator{\dipol}{D}
\DeclareMathOperator{\jkg}{J} % Johnson-Kneser-Graph
\DeclareMathOperator{\vjkg}{VK} % Johnson-Kneser-Graph
%
% Hypergraphen:
\DeclareMathOperator{\Tr}{Tr} % Transversalen
\DeclareMathOperator{\Ind}{Ind} % independent sets
\DeclareMathOperator{\Zuo}{Mat} % matchings
\DeclareMathOperator{\Pzuo}{PMat} % perfect matchings
\DeclareMathOperator{\St}{St} % stars
\DeclareMathOperator{\Ints}{Ints} % intersecting hypergraphs
\DeclareMathOperator{\Cov}{Cov} % coverings
\DeclareMathOperator{\closse}{clo_{\sse}} % Abschluss unter Teilmengenbildung
\DeclareMathOperator{\clospe}{clo_{\supseteq}} % Abschluss unter Obermengenbildung
\DeclareMathOperator{\edgemg}{ML} % Kantenmultigraph
\newcommand{\dualh}[1]{\trans{#1}} % dualer Hypergraph
\DeclareMathOperator{\kneserg}{K} % Knesergraph
\DeclareMathOperator{\knesern}{\tau_0} % Kneserzahl
\DeclareMathOperator{\nis}{nis} % Nichtschneidungszahl; non-intersection number
%
% Designs:
%
\DeclareMathOperator{\PBD}{PBD}
\nc{\BD}[1]{{#1}\text{-}\mr{BD}}
\DeclareMathOperator{\BIBD}{BIBD}
\DeclareMathOperator{\Steiner}{S}
\DeclareMathOperator{\SteinerTriple}{STS}
\DeclareMathOperator{\SteinerQuadruple}{SQS}
\DeclareMathOperator{\progeo}{PG} % projektiver Raum
\DeclareMathOperator{\affgeo}{AG} % affiner Raum
\newcommand{\vgrd}[2]{g_{#1,#2}}
%
% Association schemes (Assoziationsschemata):
%
\newcommand{\aspf}[1]{p^{(#1)}} % parameters of the first kind (Parameter der ersten Art)
\DeclareMathOperator{\astriv}{A_t} % trivial association scheme (triviales Assoziationsschema)
%
% Conways Problem
\DeclareMathOperator{\KochenSpecker}{KS}
\DeclareMathOperator{\KochenSpeckerErw}{KS'}
%
% Rangdiskrepanz
\DeclareMathOperator{\rankd}{rd}
%
% Maximale Anzahl von Zusammenhangskomponenten nach Aufspaltung
\DeclareMathOperator{\mnconcomp}{mncc}
%
% Graphenprodukte (10.10.2004)
%
\DeclareMathOperator{\gpk}{\Box} % Graphenprodukt: kartesisch
\DeclareMathOperator{\gpw}{\times} % Graphenprodukt: schwaches
\DeclareMathOperator{\gps}{\boxtimes} % Graphenprodukt: starkes
\DeclareMathOperator{\gjoin}{\boxdot} % Join von Graphen
\DeclareMathOperator{\gjoinplus}{\boxplus} % Join von Graphen
%
% Ordnungstheorie (26.2.2005)
\DeclareMathOperator{\Ketten}{\mc{L}}
\DeclareMathOperator{\Antiketten}{\mc{A}}
\DeclareMathOperator{\comparable}{\Bumpeq}
\DeclareMathOperator{\incomparable}{\parallel}
%
% 21.11.2004
\DeclareMathOperator{\pot}{\PP} % Potenzmenge; power set
\DeclareMathOperator{\pote}{\PP_f} % endliche Potenzmenge; finite power set
\DeclareMathOperator{\potfv}{\overrightarrow{\PP}} % Vorwaerts-Potenzmengenfunktor
\newcommand{\potfvi}[1]{\overrightarrow{\PP}_{\!\!#1}} % Vorwaerts-Potenzmengenfunktor mit Index
\DeclareMathOperator{\potfvn}{\overrightarrow{\PP}^{\!*}} % nichtleerer Vorwaerts-Potenzmengenfunktor
\newcommand{\potfvni}[1]{\overrightarrow{\PP}^{\!*}_{\!\!#1}} % nichtleerer Vorwaerts-Potenzmengenfunktor mit Index
\DeclareMathOperator{\potfr}{\overleftarrow{\PP}} % Rueckwaerts-Potenzmengenfunktor
\newcommand{\potfri}[1]{\overleftarrow{\PP}_{\!\!#1}} % Rueckwaerts-Potenzmengenfunktor mit Index
%
%
% ########################################################
% -----------------------------------------------------------------------------------------------------------------
% MATROIDS (Matroide)
% -----------------------------------------------------------------------------------------------------------------
% ########################################################
%
\DeclareMathOperator{\mtris}{\mc{I}} % unabh\"angige Mengen eines Matroids (independent sets of a matroid)
\DeclareMathOperator{\mtrdp}{\mc{D}} % abh\"angige Mengen eines Matroids (dependent sets of a matroid)
%
% ########################################################
% -----------------------------------------------------------------------------------------------------------------
% ALGEBRA (Algebra)
% -----------------------------------------------------------------------------------------------------------------
% ########################################################
%
% Gruppoide (25.6.2006)
%
\providecommand{\huelle}[1]{\langle #1 \rangle} % Huelle; hull (closure)
\DeclareMathOperator{\linksnull}{LZ} % Linksnull-Halbgruppe
\DeclareMathOperator{\rechtsnull}{RZ} % Rechtsnull-Halbgruppe
% Halbringe (30.10.2004)
\DeclareMathOperator{\can}{Can} % kuerzbare Elemente (in Monoiden)
\DeclareMathOperator{\addcan}{Can^+}
\DeclareMathOperator{\multcan}{Can^{\ast}}
\DeclareMathOperator{\sol}{Sol} % loesbare Elemente (in Monoiden)
\DeclareMathOperator{\addsol}{Sol^+}
\DeclareMathOperator{\multsol}{Sol^{\ast}}
\DeclareMathOperator{\inv}{Inv} % invertierbare Elemente (in Monoiden)
\DeclareMathOperator{\addinv}{Inv^+} % Menge der additiv invertierbaren Elemente
\DeclareMathOperator{\multinv}{Inv^{\ast}} % Menge der multiplikativ invertierbaren Elemente
\DeclareMathOperator{\idemp}{Ip} % idempotente Elemente (in Monoiden)
\DeclareMathOperator{\addidemp}{Ip^+}
\DeclareMathOperator{\multidemp}{Ip^{\ast}}
\nc{\konv}[2]{{#1}[{#2}]} % Halbringsbildung durch Konvolution
\DeclareMathOperator{\radikal}{Rad} % Radikal
%\DeclareMathOperator{\ker}{ker}
\DeclareMathOperator{\ordH}{ord} % Ordnung eines Elementes in einer Halbgruppe
\DeclareMathOperator{\indH}{ind} % Index eines Elementes in einer Halbgruppe
\DeclareMathOperator{\perH}{per} % Periode eines Elementes in einer Halbgruppe
%
% Zentralisatoren; centralisors
\DeclareMathOperator{\cent}{C}
\DeclareMathOperator{\centr}{Z} % Zentrum; centre
%
% Ideale; ideals
\DeclareMathOperator{\slids}{\mc{LIDS}} % Menge der Linksideale in einer Halbgruppe; set of left ideals in a semigroup
\DeclareMathOperator{\srids}{\mc{RIDS}} % Menge der Rechtsideale in einer Halbgruppe; set of right ideals in a semigroup
\DeclareMathOperator{\sids}{\mc{IDS}} % Menge der Ideale in einer Halbgruppe; set of ideals in a semigroup
%
% Greensche Relationen
\DeclareMathOperator{\qol}{\le_{\mc{L}}}
\DeclareMathOperator{\qor}{\le_{\mc{R}}}
\DeclareMathOperator{\qoh}{\le_{\mc{H}}}
\DeclareMathOperator{\qoj}{\le_{\mc{J}}}
\DeclareMathOperator{\eql}{\mc{L}}
\DeclareMathOperator{\eqr}{\mc{R}}
\DeclareMathOperator{\eqh}{\mc{H}}
\DeclareMathOperator{\eqj}{\mc{J}}
\DeclareMathOperator{\eqd}{\mc{D}}
%
% Aktionen und Operationen (22.4.2007)
\DeclareMathOperator{\orbit}{\mc{O}}
\DeclareMathOperator{\stab}{Stab}
\nc{\actpres}[1]{\Phi_{#1}} % Aktionen als Darstellungen; actions as representations
%
% Gruppen; groups
%
% Konjugation; conjugation:
\DeclareMathOperator{\conjc}{K} % Konjugationsklasse eines Elements oder einer Untergruppe; conjugacy class of an element or of a subgroup
%
% Permutationsgruppen; permutation groups
%
\DeclareMathOperator{\cycle}{\mr{cyc}} % Zyklus zu einem Tupel; cycle associated with a tuple
\DeclareMathOperator{\cyclet}{ct} % Zyklentyp einer Permutation; cycle type of a permutation
\DeclareMathOperator{\cyclen}{cn} % Zyklenanzahl einer Permutation; number of cycles of a permutation
%
% Ringe; rings
\DeclareMathOperator{\slid}{\mc{LID}} % Menge der Linksideale in einem Ring; set of left ideals in a ring
\DeclareMathOperator{\srid}{\mc{RID}} % Menge der Rechtsideale; set of right ideals
% \sid ist die Menge der Ideale (wie fuer Verbaende); set of ideals
% \smid ist die Menge der maximalen echten Ideale (wie fuer Verbaende); set of proper maximal ideals
%
% Universelle Algebra; universal algebra
%
\DeclareMathOperator{\oper}{Op} % Menge aller Operationen
\DeclareMathOperator{\clone}{Clo} % Operationenklon
%
% ########################################################
% -----------------------------------------------------------------------------------------------------------------
% NUMBER THEORY (Zahlentheorie)
% -----------------------------------------------------------------------------------------------------------------
% ########################################################
%
% Auf- und absteigende Fakultaeten (23.4.2004)
\newcommand{\untfak}[2]{(#1)_{\downarrow#2}}
\newcommand{\obfak}[2]{(#1)_{\uparrow#2}}
\DeclareMathOperator{\fak}{fac}
%
\newcommand{\ueber}[2]{\genfrac{}{}{0pt}{}{#1}{#2}}% ersetzt atop
%
\newcommand{\floor}[1]{\lfloor #1 \rfloor}
\newcommand{\ceil}[1]{\lceil #1 \rceil}
%
% Spezielle Zahlenfolgen (9.10.2005)
\newcommand{\bernoulliz}[1]{b_{#1}}
\newcommand{\bernoullip}[2]{b_{#1}(#2)}
\newcommand{\stirlinge}[2]{s_{#1,#2}}
\newcommand{\stirlingz}[2]{S_{#1,#2}}
\newcommand{\partition}[2]{p_{#1}(#2)}
\DeclareMathOperator{\partitiont}{p}
%
\DeclareMathOperator{\teilt}{\mid} % Teilbarkeitsbeziehung; divisor relation
\DeclareMathOperator{\nteilt}{\nmid} % negierte Teilbarkeitsbeziehung; negated divisor relation
\nc{\Prim}{\mc{PR}} % Primzahlen
\DeclareMathOperator{\ord}{ord}
\DeclareMathOperator{\ggt}{ggt}
\DeclareMathOperator{\kgv}{kgv}
% Ganzzahliger Rest und ganzzahlige Division
\DeclareMathOperator{\opmod}{mod}
\DeclareMathOperator{\opdiv}{div}
% Arithmetische Funktionen
\DeclareMathOperator{\eulphi}{\vp}
%
% Integrallogarithums (logarithmic integral)
\DeclareMathOperator{\Li}{Li}
% Integralexponentialfunction (exponential integral)
\DeclareMathOperator{\Ei}{Ei}
%
% ########################################################
% -----------------------------------------------------------------------------------------------------------------
% GEOMETRY (Geometrie)
% -----------------------------------------------------------------------------------------------------------------
% ########################################################
%
\newcommand{\vvek}[2]{\overrightarrow{{#1}{#2}}} % Verbindungsvektor
%
\DeclareMathOperator{\projr}{\mc{P}}
\newcommand{\projri}[1]{\mc{P}_{#1}}
\DeclareMathOperator{\affr}{\mc{A}}
\newcommand{\affri}[1]{\mc{A}_{#1}}
%
\DeclareMathOperator{\convh}{conv} % konvexe Huelle; convex hull
%
% ########################################################
% -----------------------------------------------------------------------------------------------------------------
% TOPOLOGY (TOPOLOGIE)
% -----------------------------------------------------------------------------------------------------------------
% ########################################################
%
\DeclareMathOperator{\offm}{\tau} % Menge der offenen Mengen
\DeclareMathOperator{\abgm}{\tau_c} % Menge der abgeschlossenen Mengen
\DeclareMathOperator{\offabgm}{clop} % Menge der offen-abgeschlossenen Mengen (``clopen'')
\DeclareMathOperator{\offabgmF}{Clop} % der entsprechende Funktor
\DeclareMathOperator{\offabgmFz}{Clop^*} % dieser Funktor eingeschr\"ankt auf die nulldimensionalen Raeume
\DeclareMathOperator{\offabgmFbl}{Clop_b} % jener Funktor noch weitergefuehrt in die booleschen Verbaende
\DeclareMathOperator{\offabgmFblz}{Clop_b^*} % % dieser Funktor eingeschr\"ankt auf die nulldimensionalen Raeume
\DeclareMathOperator{\nf}{NF} % Nachbarschaftsfilter
\DeclareMathOperator{\erzf}{\mf{GF}} % erzeugter Filter
\DeclareMathOperator{\erzt}{\Ktop} % erzeugte Topologie
\DeclareMathOperator{\abgh}{clo} % abgeschlossene Huelle
\DeclareMathOperator{\offk}{int} % offener Kern
\DeclareMathOperator{\rand}{fr} % alternativ koennte man den Rand auch mit \partial bezeichnen.
\DeclareMathOperator{\iso}{iso} % isolierte Punkte
\DeclareMathOperator{\haeu}{acc} % Haeufungspunkte (``Akkumulationspunkte'')
\DeclareMathOperator{\ohaeu}{acc_{\omega}} % omega-Haeufungspunkte
\DeclareMathOperator{\kond}{acc_{>\omega}} % Kondensationspunkte
\DeclareMathOperator{\haeuind}{\lambda} % Haeufungspunkt-Index
\DeclareMathOperator{\limitp}{lmp} % Menge der Grenzwerte; set of limit points
\DeclareMathOperator{\clup}{clp} % Menge der Haeufungswerte; set of cluster points
%
\DeclareMathOperator{\Off}{\mf{O}} % Verband der offenen Mengen; lattice of open sets
\DeclareMathOperator{\Abg}{\mf{A}} % Verband der abgeschlossenen Mengen; lattice of closed sets
%
\DeclareMathOperator{\erzalg}{gsa} % erzeugende Mengenalgebren
%
% Filter, Bornologien, Roste, Trichter
%
\DeclareMathOperator{\sfilt}{\mc{FLT}} % Menge der Filter; set of filters
\DeclareMathOperator{\sufilt}{\beta} % Menge der Ultrafilter; set of ultrafilters
\DeclareMathOperator{\sborn}{\mc{BRN}} % Menge der Bornologien; set of bornologies
\DeclareMathOperator{\suborn}{\sborn_u} % Menge der Ultrabornologien; set of ultra-bornologies
\DeclareMathOperator{\sfun}{\mc{FNL}} % Menge der Trichter (funnel)
\DeclareMathOperator{\sgri}{\mc{GRL}} % Menge der Roste (grill)
%
\DeclareMathOperator{\stet}{cont} % Menge der Stetigkeitspunkte; points of continuity
\DeclareMathOperator{\ustet}{dict} % Menge der Unstetigkeitspunkte; points of discontinuity
%
\newcommand{\T}[1]{T_{\mr{#1}}} % Trennungsaxiome
\DeclareMathOperator{\supp}{supp} % Traeger
\DeclareMathOperator{\hypo}{\mf{H}} % Hypergraph der nichtleeren offenen Mengen
%
\nc{\sselr}{\sse^{\mapsto}}
\nc{\sserl}{\sse^{\mapsfrom}}
\nc{\spelr}{\spe^{\mapsto}}
\nc{\sperl}{\spe^{\mapsfrom}}
%
\DeclareMathOperator{\okugel}{K} % offene Kugel
\DeclareMathOperator{\akugel}{\ol{K}} % abgeschlossene Kugel
\DeclareMathOperator{\rkugel}{K'} % Rand der Kugel
\DeclareMathOperator{\pokugel}{\dot{K}} % punktierte offene Kugel
\DeclareMathOperator{\pakugel}{\dot{\ol{K}}} % punktierte abgeschlossene Kugel
\nc{\ball}[1]{\mr{B}^{#1}} % Einheitsball
\nc{\oball}[1]{\breve{\mr{B}}^{#1}} % offener Einheitsball
\nc{\pball}[1]{\dot{\mr{B}}^{#1}} % punktierter Einheitsball
\nc{\prr}[1]{\dot{\RR}^{#1}} % punktierter n-dimensionaler euklidischer Raum
\nc{\sph}[1]{\mr{S}^{#1}} % Einheitssphaere
\nc{\ssim}[1]{s\sigma_{#1}} % Standardsimplex
%
\DeclareMathOperator{\offs}{int} % offenes Simplex
\nc{\koerper}[1]{\norm{#1}}
%
% Dimension theory
\nc{\Ccovdim}{\mc{CD}}
\nc{\Cinddim}{\mc{SID}}
\DeclareMathOperator{\inddim}{idim}
\nc{\CInddim}{\mc{LID}}
\DeclareMathOperator{\Inddim}{Idim}
%
\DeclareMathOperator{\sset}{sset} % Nachfolgermenge
%
% Manifolds
%
\DeclareMathOperator{\Karten}{\mc{CH}}
\DeclareMathOperator{\Atlanten}{\mc{AT}\!}
\DeclareMathOperator{\Kartifizierbar}{\mc{MP}} % mapable
\newcommand{\DiffKarten}[1]{\mc{CS}^{#1}}
%
% Mengen von Abbildungen
%
\DeclareMathOperator{\Abbp}{\Abb_p} % Menge der Abbildungen mit der punktweisen Konvergenz
\DeclareMathOperator{\sconm}{C} % Menge von stetigen Abbildungen; set of continous maps
\DeclareMathOperator{\sconmp}{C_p} % Menge von stetigen Abbildungen mit der punktweisen Konvergenz; set of continous maps with pointwise convergence
%
% ########################################################
% -----------------------------------------------------------------------------------------------------------------
% ANALYSIS (Analysis)
% -----------------------------------------------------------------------------------------------------------------
% ########################################################
%
\DeclareMathOperator{\diffop}{D} % totaler Ableitungsoperator
\DeclareMathOperator*{\diffoplimit}{D} % Interne Hilfskonstruktion
\nc{\diffopc}[1]{\sideset{_{#1}}{}\diffoplimit} % komponentenweiser Ableitungsoperator
\nc{\diffopp}[1]{\diffop_{#1}} % partieller Ableitungsoperator
\nc{\diffopcp}[2]{\sideset{_{#2}}{_{#1}}\diffoplimit} % kombinierter Ableitungsoperator
\DeclareMathOperator{\diff}{d} % Differential
%
\DeclareMathOperator{\lincop}{\mc{L}} % space of continuous linear operators
%
\DeclareMathOperator{\vpnorm}{N} % verallgemeinerte p-Norm
\nc{\meanH}[2]{\mf{M}_{#1,#2}} % Mittelwerte verallgemeinert nach Hardy et al
\DeclareMathOperator{\gmean}{\mf{M}} % Potenzmittelwerte
\nc{\emean}[2]{\mf{M}_{\exp_{#1},#2}} % Exponentialmittelwerte
%
% ########################################################
% -----------------------------------------------------------------------------------------------------------------
% CATEGORY THEORY (Kategorientheorie)
% -----------------------------------------------------------------------------------------------------------------
% ########################################################
%
\DeclareMathOperator{\obj}{Obj}
\DeclareMathOperator{\mor}{Mor}
\DeclareMathOperator{\Hom}{Hom} % Morphismenmengen in algebraischen Kontexten
\DeclareMathOperator{\mendo}{End} % Endomorphismenmonoid (ersetzt das aeltere \DeclareMathOperator{\edom}{End})
\DeclareMathOperator{\auto}{Aut} % Automorphismengruppe
\nc{\autoerw}[1]{\mr{Aut}^{#1}} % erweiterte Automorphismengruppe
\DeclareMathOperator{\autotrans}{Aut^{(t)}} % erweiterte Automorphismengruppe bzgl. Transposition
%
\newcommand{\dualk}[1]{\trans{#1}} % duale Kategorie
\newcommand{\dualf}[1]{\trans{#1}} % dualer Funktor
%
\newcommand{\ramono}{\hookrightarrow} % Monomorphismen; monomorphisms
\newcommand{\raepi}{\twoheadrightarrow} % Epimorphismen; epimorphisms
\newcommand{\raiso}{\ramono\hspace{-0.5em}\rightarrow} % Isomorphismen; isomorphisms
%
\newcommand{\prodmorf}{\bigotimes} % Produkt einer Morphismenfamilie f_i: X \ra Y_i als Morphismus von X nach dem Produkt der Y_i.
\DeclareMathOperator{\prodmor}{\otimes} % Produkt zweier Morphismen
\newcommand{\coprodmorf}{\bigsqcup} % Koprodukt einer Morphismenfamilie f_i: X_i \ra Y als Morphismus von dem Koprodukt der X_i nach Y.
\DeclareMathOperator{\coprodmor}{\sqcup} %  Koprodukt zweier Morphismen
\newcommand{\mprodmorf}{\prod} % "Mehrfaches" Produkt einer Familie
\DeclareMathOperator{\mprodmor}{\times}
\newcommand{\mcoprodmorf}{\coprod}
\DeclareMathOperator{\mcoprodmor}{\amalg}
%
\nc{\komma}[2]{(#1 \downarrow #2)} % Komma-Kategorie
%
\DeclareMathOperator{\prodsem}{\mc{PS}} % Halbgruppe bzgl. Produktbildung modulo Isomorphie
\DeclareMathOperator{\coprodsem}{\mc{CS}} % Halbgruppe bzgl. Koproduktbildung modulo Isomorphie
\DeclareMathOperator{\prodmon}{\mc{PM}} % Monoid bzgl. Produktbildung modulo Isomorphie
\DeclareMathOperator{\coprodmon}{\mc{CM}} % Monoid bzgl. Koproduktbildung modulo Isomorphie
%
\DeclareMathOperator{\zsh}{Z} % zusammenhaengende Objekte nach Preuss, Definition 5.1.5
%
\newcommand{\Kset}{\mf{SET}} % Mengen
\newcommand{\Kfset}{\Kset_{\mr{f}}} % endliche Mengen
\newcommand{\Kard}{\mc{CARD}} % Kardinalzahlen
\newcommand{\Ord}{\mc{ORD}} % Ordinalzahlen
\newcommand{\Kardm}{\Kard_{\!-1}}
\DeclareMathOperator{\cardsup}{cardsup}
\DeclareMathOperator{\cardsupl}{cardsup_<}
\DeclareMathOperator{\cardmin}{cardmin}
%
\newcommand{\Kkor}{\mf{KOR}} % Korrespondenzen
\newcommand{\liei}{\mr{lu}} % linkseindeutig
\newcommand{\reei}{\mr{ru}} % rechtseindeutig
\newcommand{\lito}{\mr{lt}} % linkstotal
\newcommand{\reto}{\mr{rt}} % rechtsstotal
\newcommand{\kortyp}{\mr{KT}} % Korrespondenztypen
%
\newcommand{\Krel}{\mf{REL}} % Relationale
\newcommand{\Krelk}[2]{\Krel_{#1}\mf{K}_{#2}} % Relationen mit Korrespondenzen
\newcommand{\Krelkr}[2]{\Krelk{#1}{#2}\mf{R}} % Rueckwaertsmorphismen
%
% Quasiorders (Quasiordnungen), lattices (Verbaende):
%
\newcommand{\Kqord}{\mf{QORD}} % Quasi-Ordnungen
\newcommand{\Kpord}{\mf{PORD}} % partielle Ordnungen
\newcommand{\Klord}{\mf{LORD}} % lineare Ordnungen
\newcommand{\seb}{\mr{s}} % supremumserhaltend binaer
\newcommand{\seu}{\seb^{\infty}} % supremumserhaltend unendlich
\newcommand{\sef}{\seb^{<\infty}} % supremumserhaltend endlich
\newcommand{\seun}{\seb^{\infty}_{\not= \es}} % supremumserhaltend unendlich nichtleer
\newcommand{\seen}{\seb^{<\infty}_{\not= \es}} % supremumserhaltend endlich nichtleer
\newcommand{\ieb}{\mr{i}} % infimumserhaltend binaer
\newcommand{\ieu}{\ieb^{\infty}} % infimumserhaltend unendlich
\newcommand{\iee}{\ieb^{<\infty}} % infimumserhaltend endlich
\newcommand{\ieun}{\ieb^{\infty}_{\not= \es}} % infimumserhaltend unendlich nichtleer
\newcommand{\ieen}{\ieb^{<\infty}_{\not= \es}} % infimumserhaltend endlich nichtleer
\newcommand{\Kvb}{\mf{LT}} % Verb\"ande
\newcommand{\Kuvb}{\ul{\mf{S}}\Kvb} % untere Halbverb\"ande
\newcommand{\Kovb}{\ol{\mf{S}}\Kvb} % obere Halbverb\"ande
\newcommand{\Kvvb}{\mf{C}\Kvb} % vollstaendige Verb\"ande
\newcommand{\Kvuvb}{\mf{C}\Kuvb} % untere vollstaendige Halbverb\"ande
\newcommand{\Kvovb}{\mf{C}\Kovb} % obere vollstaendige Halbverb\"ande
\newcommand{\Kbvb}{\mf{B}\Kvb} % Verb\"ande mit Null und Eins
\newcommand{\Kbuvb}{\mf{B}\Kuvb} % untere Halbverb\"ande mit Null
\newcommand{\Kbovb}{\mf{B}\Kovb} % obere Halbverb\"ande mit Eins
\newcommand{\Kdvb}{\mf{D}\Kvb} % distributive Verb\"ande
\newcommand{\Kkvb}{\mf{CO}\Kvb} % komplementierte Verb\"ande
\newcommand{\Kekvb}{\mf{UCO}\Kvb} % eindeutig komplementierte Verb\"ande
\newcommand{\Kboolvb}{\mf{BO}\Kvb} % boolesche Verb\"ande
\newcommand{\Kvboolvb}{\mf{C}\Kboolvb} % vollstaendige boolesche Verb\"ande
\newcommand{\Kboolalg}{\mf{BA}} % boolesche Algebren
\newcommand{\Kheyvb}{\mf{HO}} % Heyting-Verband
\newcommand{\Kheyalg}{\mf{HA}} % Heyting-Algebren
\newcommand{\dist}{\mr{d}} % distributiv
\newcommand{\vdist}{\mr{d}^{\infty}} % vollst\"andig-distributiv
\newcommand{\komp}{\mr{c}} % komplementiert
\newcommand{\ekomp}{\mr{c}_1} % eindeutig komplementiert
\DeclareMathOperator{\qo}{QO} % formation of quasi-order
%
% Matrices (Matrizen):
%
\nc{\Kmat}{\mf{MAT}} % category of matrices (with principal homotopisms); Kategorie der Matrizen mit Haupthomotopismen als Morphismen
\nc{\Khmat}{\mf{HMAT}} % category of matrices with homotopisms; Kategorie der Matrizen mit Homotopismen als Morphismen
%
% Gruppoids (Gruppoide):
%
\newcommand{\Kgod}{\mf{GOD}} % Gruppoide
\newcommand{\Khgod}{\mf{HGOD}} % Gruppoide mit Homotopismen als Morphismen; groupoids with homotopisms as morphisms
\newcommand{\Ksgr}{\mf{SGR}} % Halbgruppen
\newcommand{\Kugod}{\mf{UGOD}} % unitales Gruppoid
\newcommand{\Kmon}{\mf{MON}} % Monoide
\newcommand{\Kcgod}{\mf{CGOD}} % kuerzbare Gruppoide
\newcommand{\Krgod}{\mf{RGOD}} % loesbare Gruppoide
\newcommand{\Kqgp}{\mf{QGP}} % Quasigruppen
\newcommand{\Kcsgr}{\mf{CSGR}} % kuerzbare Halbgruppen
\newcommand{\Kcugod}{\mf{CUGOD}} % kuerzbare unitale Gruppoide
\newcommand{\Krugod}{\mf{RUGOD}} % loesbare unitale Gruppoide
\newcommand{\Kcmon}{\mf{CMON}} % kuerzbare Monoide
\newcommand{\Klop}{\mf{LOP}} % Loops
\newcommand{\Kgr}{\mf{GRP}} % Gruppen
\newcommand{\Kagod}{\mf{AGOD}} % abelsche Gruppoide
\newcommand{\Kasgr}{\mf{ASGR}} % abelsche Halbgruppen
\newcommand{\Kaugod}{\mf{AUGOD}} % abelsche unitales Gruppoid
\newcommand{\Kamon}{\mf{AMON}} % abelsche Monoide
\newcommand{\Kacgod}{\mf{ACGOD}} % abelsche kuerzbare Gruppoide
\newcommand{\Kargod}{\mf{ARGOD}} % abelsche loesbare Gruppoide
\newcommand{\Kaqgp}{\mf{AQGP}} % abelsche Quasigruppen
\newcommand{\Kacsgr}{\mf{ACSGR}} % abelsche kuerzbare Halbgruppen
\newcommand{\Kacugod}{\mf{ACUGOD}} % abelsche kuerzbare unitale Gruppoide
\newcommand{\Karugod}{\mf{ARUGOD}} % abelsche loesbare unitale Gruppoide
\newcommand{\Kacmon}{\mf{ACMON}} % abelsche kuerzbare Monoide
\newcommand{\Kalop}{\mf{ALOP}} % abelsche Loops
\newcommand{\Kagr}{\mf{AGRP}} % abelsche Gruppen
%
\newcommand{\Klact}[2]{{}_{#1} #2} % Links-Aktionen (oder -Operation)
\newcommand{\Kract}[2]{#2_{#1}} % Rechts-Aktionen (oder -Operation)
\newcommand{\Kact}{\mf{ACT}} % Aktionen
\newcommand{\Kopt}{\mf{OPR}} % Operationen
%
% Automaten (automata):
%
\newcommand{\Kdfa}{\mf{DFA}}
\newcommand{\Kdfma}{\mf{DFMA}}
%
% Rings (Ringe):
%
\newcommand{\Kring}{\mf{RNG}} % Ringe
\newcommand{\Kkring}{\mf{CRNG}} % kommutative Ringe
\newcommand{\Khring}{\mf{SRNG}} % Halbringe
\newcommand{\Kpring}{\mf{PRNG}} % Pseudoringe
\newcommand{\Kphring}{\mf{PSRNG}} % Pseudohalbringe
\newcommand{\Kkhring}{\mf{CSRNG}} % kommutative Halbringe
\newcommand{\Kkpring}{\mf{CPRNG}} % kommutative Pseudoringe
\newcommand{\Kkphring}{\mf{CPSRNG}} % kommutative Pseudohalbringe
\newcommand{\Kringfr}{\mf{RR}} % Ringrahmen
\newcommand{\Kprering}{\mf{PRG}} % Pr\"aeringe
\newcommand{\Kaprering}{\mf{APRG}} % assoziative Pr\"aeringe
\newcommand{\Knprering}{\mf{NPRG}} % normale Pr\"aeringe
%
% Modules (Module):
%
\newcommand{\Kpremodul}[1]{\mf{PMD}_{#1}} % Praemodule
\newcommand{\Kppremodul}[1]{\mf{PPMD}_{#1}} % Pseudopraemodule
\newcommand{\Kapremodul}[1]{\mf{APMD}_{#1}} % abelscher Praemodule
\newcommand{\Ksmodul}[1]{\mf{SMOD}_{#1}} % Semimodule
\newcommand{\Kmodul}[1]{\mf{MOD}_{#1}} % Module
\newcommand{\Krmodul}[1]{\mf{MOD}^{#1}} % Rechtsmodule
\newcommand{\Krsmodul}[1]{\mf{SMOD}^{#1}} % Rechtssemimodule
\newcommand{\Kbmodul}[2]{\mf{MOD}_{#1}^{#2}} % Bimodule
\newcommand{\Kbsmodul}[2]{\mf{SMOD}_{#1}^{#2}} % Bisemimodule
\newcommand{\Kvec}[1]{\mf{VEC}_{#1}} % Vektorraeume
%
% Kategorien (categories):
%
\newcommand{\Kkat}{\mf{CAT}} % Kategorie der Kategorien
\newcommand{\Kpaddkat}{\mf{PACAT}} % praeadditive Kategorien
\newcommand{\Knpaddkat}{\mf{NPACAT}} % normale praeadditive Kategorien
\newcommand{\Ksaddkat}{\mf{SACAT}} % semiadditive Kategorien
\newcommand{\Kaddkat}{\mf{ACAT}} % additive Kategorien
\newcommand{\Kkatn}{\mf{CAT}_2} % Kategorie der Kategorien mit natuerlichen Transformationen als Morphismen
%
\DeclareMathOperator{\Kfun}{\mf{FUN}} % Funktorkategorie
\DeclareMathOperator{\nattr}{NAT} % natuerliche Transformationen
%
\DeclareMathOperator{\symm}{Symm} % Symmetrisierung einer Kategorie
%
\nc{\homfun}[1]{\mor_{#1}(-_1,-_2)} % Hom-Funktor
\nc{\homfunae}[1]{\mor_{#1}(-_1)} % Hom-Funktor als (Objekt -> kovarianter Hom-Funktor)
\nc{\homfunbe}[1]{\mor_{#1}(-_2)} % Hom-Funktor (Objekt -> kontravarianter Hom-Funktor)
\nc{\homfunxy}[3]{\mor_{#1}(#2(-_1), #3(-_2))}
\nc{\homfunx}[2]{\mor_{#1}(#2(-_1), -_2)}
\nc{\homfuny}[2]{\mor_{#1}(-_1, #2(-_2))}
\nc{\homfuna}[2]{\mor_{#1}(#2, -)} % kovarianter Homfunktor
\nc{\homfunb}[2]{\mor_{#1}(-, #2)} % kontravarianter Homfunktor
\nc{\hhomfuna}[2]{\Hom_{#1}(#2, -)} % kovarianter Homfunktor (geschrieben Hom)
\nc{\hhomfunb}[2]{\Hom_{#1}(-, #2)} % kontravarianter Homfunktor (geschrieben Hom)
%
% Topologie (Topology):
%
\newcommand{\Ktop}{\mf{TOP}} % Kategorie der topologischen R\"aume
\newcommand{\modhom}{\mf{H}} % modulo Homotopie
\newcommand{\Ktophom}{\Ktop\modhom} % Kategorie der topologischen R\"aume modulo Homotopie
\newcommand{\Kptophom}{\mf{TOP}_*\modhom} % Kategorie der punktierten topologischen R\"aume modulo Homotopie
\newcommand{\Ktoptmhom}{\Ktop_{[1]}\modhom_1} % Topologische Raeume mit Teilmengen modulo Homotopie
\newcommand{\Katop}{\mf{STOP}} % Kategorie der A-topologischen Raeume
\newcommand{\Kktop}{\mf{COMP}} % Kategorie der kompakten topologischen R\"aume
\newcommand{\Khtop}{\mf{HTOP}} % Kategorie der hausdorffschen topologischen R\"aume
\newcommand{\Kztop}{\mf{Z}\Ktop} % Kategorie der nulldimensionalen Raeume
%
\DeclareMathOperator{\Kdelta}{\Delta} % Mengen {0, ..., n-1} mit monoton steigenden Abbildungen
\DeclareMathOperator{\Kdeltas}{\Delta_{\mr{s}}} % Mengen {0, ..., n-1} mit streng monoton steigenden Abbildungen
\DeclareMathOperator{\Kdeltat}{\Delta^{\mr{t}}}
\DeclareMathOperator{\Kdeltast}{\Delta_{\mr{s}}^{\mr{t}}}
\DeclareMathOperator{\Kdeltap}{\Delta^{\!+}}
\DeclareMathOperator{\Kdeltasp}{\Delta_{\mr{s}}^{\!+}}
\DeclareMathOperator{\Kdeltapt}{\Delta^{\!+ \mr{t}}}
\DeclareMathOperator{\Kdeltaspt}{\Delta_{\mr{s}}^{\!+ \mr{t}}}
%
\newcommand{\Kloc}[1]{\mf{LOC}_{#1}} % Kategorie der lokal #1-charakterisierten Raeume
%
% Topological groups (topologische Gruppen):
%
\DeclareMathOperator{\Ktopgr}{\mf{TG}} % topologische Gruppen
\DeclareMathOperator{\Ktopagr}{\mf{TA}} % topologische abelsche Gruppen
\DeclareMathOperator{\Klqkgr}{\mf{LQCG}} % lokal quasikompakte topologische Gruppen
\DeclareMathOperator{\Klqkagr}{\mf{LQCA}} % lokal quasikompakte topologische abelsche Gruppen
\DeclareMathOperator{\Klkgr}{\mf{LCG}} % lokalkompakte topologische Gruppen
\DeclareMathOperator{\Klkagr}{\mf{LCA}} % lokalkompakte topologische abelsche Gruppen
%
% Topological vector spaces (topologische Vektorraeume):
%
\newcommand{\Ktpremodul}[1]{\mf{TPMD}_{#1}} % topologische Praemodule
\newcommand{\Ktppremodul}[1]{\mf{TPPMD}_{#1}} % topologische Pseudopraemodule
\newcommand{\Ktapremodul}[1]{\mf{TAPMD}_{#1}} % topologische abelscher Praemodule
\newcommand{\Ktsmodul}[1]{\mf{TSMOD}_{#1}} % topologische Semimodule
\newcommand{\Ktmodul}[1]{\mf{TMOD}_{#1}} % topologische Module
\newcommand{\Ktrmodul}[1]{\mf{TMOD}^{#1}} % topologische Linksmodule
\newcommand{\Ktbmodul}[2]{\mf{TMOD}_{#1}^{#2}} % topologische  Bimodule
\newcommand{\Ktvec}[1]{\mf{TVEC}_{#1}} % topologische  Vektorraeume
%
% Graphs (Graphen):
%
\newcommand{\Kdgg}{\mf{GDG}} % directed general graphs (general digraphs); allgemeine gerichtete Graphen
\newcommand{\Kdg}{\mf{DG}} % directed graphs
\newcommand{\Kdgl}{\mf{DGL}} % directed graphs with loops
\newcommand{\Kdrg}{\mf{DRG}} % directed reflexive graphs
\newcommand{\Kgg}{\mf{GG}} % general graphs
\newcommand{\Kg}{\mf{GR}} % graphs
\newcommand{\Kgl}{\mf{GRL}} % graphs with loops
\newcommand{\Kfg}{\Kg_{\mr{f}}} % finite graphs
\newcommand{\Kfgl}{\Kgl_{\mr{f}}} % finite graphs with loops
\newcommand{\Kfgg}{\Kgg_{\mr{f}}} % finite general graphs
%
\newcommand{\Khg}{\mf{HGR}} % hypergraphs
\newcommand{\Khgi}{\Khg^{\infty}} % hypergraphs with infinite edges
\newcommand{\Khgiinz}{\Khg^{\infty}_{\mr{inz}}} % hypergraphs with infinite edges and incidence morphisms
\newcommand{\Kfhg}{\Khg_{\mr{f}}} % finite hypergraphs 
\newcommand{\Kghg}{\mf{G}\Khg} % general hypergraphs
\newcommand{\Kghgi}{\Kghg^{\infty}} % general hypergraphs with infinite edges
\newcommand{\Kfghg}{\Kghg_{\mr{f}}} % finite general hypergraphs
\newcommand{\Kbip}{\mf{BIP}} % bipartite Systeme
%
\DeclareMathOperator{\dgg}{dgg} % formation of directed general graphs
\DeclareMathOperator{\dgl}{dgl} % formation of directed graphs with loops
\DeclareMathOperator{\dg}{dg} % formation of directed graphs
\DeclareMathOperator{\ugg}{ugg} % zugrundeliegender allgemeiner Graph
\DeclareMathOperator{\ugl}{ugl} % zugrundeliegender Graph mit Schlingen
\DeclareMathOperator{\ug}{ug} % zugrundeliegender Graph
\DeclareMathOperator{\sg}{sg} % Symmetrisierung eines Graphen
\DeclareMathOperator{\sgl}{sgl} % Symmetrisierung eines Graphen mit Schlingen
%
\DeclareMathOperator{\bip}{bip} % Das bipartite System assoziiert mit einem allgemeinen Hypergraphen
\DeclareMathOperator{\hyp}{hyp} % der allgemeine Hypergraph assoziiert mit einem bipartiten System.
%
\DeclareMathOperator{\cat}{cat} % formation of categories (e.g., the free category created by a graph)
%
% Base sets with set systems
%
\newcommand{\Ksetsys}{\mf{SSYS}}
\newcommand{\Ksetrng}{\mf{SRNG}} % Kategorie der Mengenringe
\newcommand{\Ksetalg}{\mf{SALG}} % Kategorie der Mengenalgebren
\DeclareMathOperator{\gtopsalg}{Gtop} % Functor: generated topology from set-algebra
\DeclareMathOperator{\salgbolt}{V_{\!\Ksetalg}} % Funktor: von Mengenalgebren in boolesche Verbaende
%
% Klauselmengen (clause-sets)
%
\newcommand{\Kcls}{\mf{CLS}} % Kategorie der Klauselmengen (Vorwaertsmorphismen); category of clause-sets (forward morphisms)
\newcommand{\Kfcls}{\mf{FCLS}} % Kategorie der formalen Klauselmengen; category of formal clause-sets
\newcommand{\Klcls}{\mf{LCLS}} % Kategorie der markierten Klauselmengen; category of labelled clause-sets
% Indizierte Versionen (indexed versions):
\newcommand{\Kfclsi}[1]{\Kfcls_{\mr{#1}}}
\newcommand{\Klclsi}[1]{\Klcls_\mr{#1}}
% Zusaetzlich auch mit oberem Index (allowing also super-indices):
\newcommand{\Kfclsis}[2]{\Kfcls_{\mr{#1}}^{#2}}
\newcommand{\Klclsis}[2]{\Klcls_\mr{#1}^{#2}}
% Old (no longer used):
\newcommand{\Kccls}{\mf{CLS}} % Kategorie der potentiell kollidierenden Klauselmengen; category of c-clause-sets
\newcommand{\Klccls}{\mf{L}\mr{c}\mf{CLS}} % Kategorie der markierten potentiell kollidierenden Klauselmengen; category of c-clause-sets
\newcommand{\Kfccls}{\mf{F}\mr{c}\mf{CLS}}
\newcommand{\Klcclsi}[1]{\Klccls_\mr{#1}}
\newcommand{\Kfcclsi}[1]{\Kfccls_\mr{#1}}
\newcommand{\Klcclsis}[2]{\Klccls_\mr{#1}^{#2}}
\newcommand{\Kfcclsis}[2]{\Kfccls_\mr{#1}^{#2}}
%
% ########################################################
% ----------------------------------------------------------------------------------------------------------------
% LOGIK (Logik)
% -----------------------------------------------------------------------------------------------------------------
% ########################################################
%
\newcommand{\einfTyp}{\mc{T}_{\ra}} % Menge der einfachen Typen
\newcommand{\einfType}{\mc{T}_{\ra,\times}} % Menge der erweiterten einfachen Typen
%
\DeclareMathOperator{\theory}{Th} % Theorie zu einer Menge von Strukturen; theory for set of structures
%
%
% ########################################################
% ----------------------------------------------------------------------------------------------------------------
% COMBINATORICS (Kombinatorik)
% -----------------------------------------------------------------------------------------------------------------
%
\newcommand{\gbinom}[3]{{\binom{#1}{#2}}_{\! #3}} % verallgemeinerte Binomialkoeffizienten (mittels Multimengen); generalised binomial coefficients (via multisets).
\newcommand{\mjections}[2]{\mr{MJ}_{#1}(#2)} % \mjections km ist die Anzahl der k-stelligen Multijektionen f\"ur eine m-elemente Menge; the number of k-ary multijections for an m-set.
\DeclareMathOperator{\quasigp}{QG} % lateinische Quadrate (Quasigruppen); latin squares (quasigroups)
\DeclareMathOperator{\uquasigp}{uQG} % unitale Quasigruppen (loops); unital quasigroups (loops)
%
\DeclareMathOperator{\suquasigp}{uQG_s} % standardisierte unitale Quasigruppen (loops); standardised unital quasigroups (loops)
\DeclareMathOperator{\luquasigp}{luQG} % links-unitale Quasigruppen; left unital quasigroups
%
\DeclareMathOperator{\sluquasigp}{luQG_s} % standardisierte links-unitale Quasigruppen; standardised left unital quasigroups
%
% ########################################################
% ----------------------------------------------------------------------------------------------------------------
% SAT NOTATIONS (SAT-Notationen)
% -----------------------------------------------------------------------------------------------------------------
% ########################################################
%
% Fundamentales:
%
\newcommand{\Va}{\mc{V\hspace{-0.1em}A}}
\newcommand{\Dom}{\mc{DO\hspace{-0.08em}M}}
\newcommand{\Lit}{\mc{LIT}}
\newcommand{\Cl}{\mc{CL}}
\newcommand{\Cls}{\mc{CLS}}
\newcommand{\Tcls}{3\mbox{--}\Cls}
\newcommand{\Pcls}[1]{#1\mbox{--}\Cls}
\newcommand{\Clsab}[2]{\Cls\mbox{-}(#1, #2)}
\newcommand{\Clsr}[1]{\Cls\mbox{-}#1}
\newcommand{\Pass}{\mc{P\hspace{-0.32em}ASS}}
\newcommand{\epa}{\pab{}} % leere partielle Belegung; empty partial assignment
\newcommand{\Tass}{\mc{T\hspace{-0.35em}ASS}}
\newcommand{\Sat}{\mc{SAT}}
\newcommand{\Tsat}{3\mbox{-}\Sat}
\newcommand{\Usat}{\mc{USAT}}
\newcommand{\Usati}[1]{\Usat_{\!\!#1}}
\newcommand{\Musat}{\mc{M\hspace{0.8pt}U}} % minimally unsatisfiable clause-sets; minimal unerfuellbare Klauselmengen
\newcommand{\Musati}[1]{\Musat_{\!#1}} % used to select subsets, e.g., $\Musati{\delta=k}$; zur Auswahl von Teilklassen
\newcommand{\Smusat}{\mc{S}\Musat} % saturated (``maximal'') minimally unsatisfiable clause-sets; saturierte (``maximale'') minimal unerfuellbare Klauselmengen
\newcommand{\Smusati}[1]{\Smusat_{\!#1}}
\newcommand{\Mmusat}{\mc{M}\Musat}% marginale minimal unerf\"ullbare Klms (25.4.2001)
\newcommand{\Mmusati}[1]{\Mmusat_{\!#1}}
\newcommand{\Vmusat}{\mc{V}\Musat} % variable minimally unsatisfiable clause-sets; Variablen-minimal unerfuellbare Klauselmengen
\newcommand{\Vmusati}[1]{\Vmusat_{\!#1}}

%
\newcommand{\Fpass}{{\mc{F}}\Pass}
\nc{\Clsoo}{\Cls^{1,1}} % jede Variable hat 1-1-Vorkommen; each variable occurs once positively and once negatively
%
\DeclareMathOperator{\lit}{lit}
\DeclareMathOperator{\var}{var}
\DeclareMathOperator{\val}{val}
\DMO{\dos}{ds} % domain size
\DMO{\mdos}{mds} % maximal domain size
%
\newcommand{\sateq}{\overset{\mr{sat}}{\equiv}}
%
% Treffklauselmengen
%
\newcommand{\AF}{\mc{A}}% ausgezeichnete KNF
\newcommand{\Clash}{\mc{HIT}} % schwach-resolvierbare Klauselmengen (global umdefiniert 24.7.2003)
\newcommand{\Clashi}[1]{\Clash_{\!\!#1}}
\newcommand{\Uclash}{\mc{U}\Clash} % unerfuellbare Treffklauselmengen
\newcommand{\Uclashi}[1]{\Uclash_{\!\!#1}}
\newcommand{\Sclash}{\mc{R}\Clash} % stark-resolvierbare Klauselmengen (global umdefiniert 24.7.2003; 7.4.2007: nun ``R'' fuer regulaer)
\newcommand{\Sclashk}[1]{\Sclash_{\! #1}}
\newcommand{\Mclash}{\mc{M}\Clash} % Multi-hitting 25.9.2004
\newcommand{\Pclash}[1]{{#1}_{\nless}\mbox{\!--}\Clash} % Klausell\"ange exakt k 13.11.2005; einen allgemeinen Standard einfuehren fuer solche exakten Klausellaengenangaben
\newcommand{\Puclash}[1]{{#1}_{\nless}\mbox{--}\Uclash}
\DeclareMathOperator{\munpuclash}{\mu{}NH}
\newcommand{\Pclashr}[2]{\Pclash{#1}\mbox{-}{#2}} % Zus\"atzlich haben alle Variablen den Grad $r$
\newcommand{\Puclashr}[1]{\Puclash{#1}\mbox{-}*} % Variablen-regulaere uniforme unerfuellbare Treffklauselmengen.
\newcommand{\Tuclash}{\mc{TUH}} % Baum-Treffklm; tree-hitting clause-sets
\newcommand{\Nclash}{\mc{NHIT}} % Fast-Treffklm; nearly hitting clause-sets
%
% Klauselmengen entsprechend Kongruenz-Ueberdeckungen; clause-sets related to
% covers by congruences:
\newcommand{\Cos}{\mc{CC}} % congruence-classes clause-sets
\newcommand{\Ucos}{\mc{CS}} % covering-systems clause-sets
\newcommand{\Clashcos}{\mc{DCC}} % disjoint congruence-classes clause-sets
\newcommand{\Uclashcos}{\mc{DCS}} % disjoint convering systems cls
%
\DeclareMathOperator{\hdef}{\delta_{\mr{h}}} % hermitean defect 6.1.2004
\DeclareMathOperator{\rdef}{\delta_{\mr{r}}} % rank defect 12.9.2004
\newcommand{\Mclshd}{\Mcls_{\hdef}} % 24.7.2003
\newcommand{\Clshd}{\Cls_{\hdef}} % 24.1.2004
\newcommand{\Mmhd}{\Mclshd(1)} % 24.7.2003
\newcommand{\Mclsp}{\Mcls_{i_+}} % 24.7.2003
\newcommand{\Mclsn}{\Mcls_{i_-}} % 24.7.2003
\newcommand{\Mclsh}{\Mcls_h} % 24.7.2003
\newcommand{\Clsp}{\Cls_{i_+}} % 24.1.2004
\newcommand{\Clsn}{\Cls_{i_-}} % 24.1.2004
\newcommand{\Clsh}{\Cls_h} % 24.1.2004
\newcommand{\Musath}{\Musat_{\!h}} % 24.7.2003
\newcommand{\Musatd}{\Musat_{\!\delta}} % 24.7.2003
\newcommand{\Smusatd}{\Smusat_{\!\delta}} % 24.7.2003
\newcommand{\Mclsds}{\Mcls_{\delta^*}} % 24.7.2003
\newcommand{\Trcls}{\mc{TCLS}}
\newcommand{\Gtrcls}{\mc{GTCLS}}
\newcommand{\Satein}{\Sat\mbox{-}1}
\newcommand{\Lsat}{\mc{L}\Sat}
\newcommand{\Ho}{\mc{HO}} % Hornformeln; Horn clause-sets
\newcommand{\Rho}{\mc{R}\Ho} % Umbenennbare Hornformeln; renamable Horn clause-sets
\newcommand{\Qho}{\mc{Q}\Ho}
\newcommand{\Lean}{\mc{LEAN}}
\newcommand{\Leani}[1]{\Lean_{\!#1}}
\newcommand{\Llean}{\mc{L}\Lean}
\newcommand{\Umusat}{\mc{U}\Musat}
\newcommand{\Mlean}{\mc{M}\Lean}
\newcommand{\Mleani}[1]{\Mlean_{\!#1}}
\newcommand{\Msat}{\mc{M}\Sat}
\newcommand{\Lsatz}{{\Lsat\!}_0}
\newcommand{\Satv}[1]{\Sat\!_{#1}} % 15.8.2001
\newcommand{\Psat}{\mc{P}\Sat}
\newcommand{\Plean}{\mc{P}\Lean}
\newcommand{\Mcls}{\mc{M}\Cls} % 27.12.2002
\newcommand{\Pmcls}[1]{#1\mbox{--}\Mcls}
\newcommand{\Scls}{\mc{SCL}} % 27.5.2004
\newcommand{\Ssat}{\Sat} % 9.6.2004
\newcommand{\Dcls}{\mc{C}\Cls} % 28.11.2004 (diagonale Klauselmengen), umbenannt am 12.9.2006 zu ``colouring clause-sets''
%
\DeclareMathOperator{\nulli}{null} % 17.5.2004
%
\newcommand{\Dnf}{\mathrm{DNF}}
\newcommand{\pdnf}{p\mbox{-}\mathrm{DNF}}
\newcommand{\tdnf}{3\mbox{-}\mathrm{DNF}}
\newcommand{\twdnf}{2\mbox{-}\mathrm{DNF}}
\newcommand{\Dnfa}{\mathrm{DNF}\mbox{-}(1, s)}
\newcommand{\Dnfoi}{\mathrm{DNF}\mbox{-}(1, \infty)}
\newcommand{\Dnfot}{\mathrm{DNF}\mbox{-}(1, 2)}
\newcommand{\tdnfot}{3\mbox{-}\mathrm{DNF}\mbox{-}(1, 2)}
\newcommand{\pdnfa}{p\mbox{-}\mathrm{DNF}\mbox{-}(1, s)}
%
\newcommand{\Cnf}{\mathrm{CNF}}
%
%
\newcommand{\pcl}[1]{${\SST\le} #1$-clause}
\newcommand{\pkl}[1]{${\SST\le} #1$-Klausel}
%
% Resolution:
%
\DeclareMathOperator{\res}{\diamond} % die (partielle) Resolutionsoperation
\DeclareMathOperator{\resop}{Res} % der Resolutionsoperator (Hinzunahme aller Resolventen)
\DeclareMathOperator{\mresop}{mRes} % der Resolutionsoperator (Hinzunahme aller Resolventen)
\DeclareMathOperator{\dpl}{DP} % der DP-Operator
\newcommand{\dpi}[1]{\dpl_{\!#1}}
%
% Beweissysteme:
%
\DeclareMathOperator{\comp}{Comp} % Komplexitaet bzgl. eines Beweisystems
\DeclareMathOperator{\compex}{\comp_{ER}} % Erweiterte Resolution; extended resolution
\DeclareMathOperator{\comptex}{\comp_{ER}^*} % baumartige erweiterte Resolution; extended resolution, tree-like
\DeclareMathOperator{\compr}{\comp_R} % volle Resolution; full resolution
\DeclareMathOperator{\comptr}{\comp_{R}^*} % baumartige Resolution; tree-like resolution
% Resolution:
\DMO{\premr}{ax} % premisses (axioms) in resolution proof
\DMO{\concr}{C} % conclusion in resolution proof
\DMO{\allcr}{cl} % all clauses in a resolution proof
% Orakel:
\newcommand{\Us}{\mc{U}} % August 2000 : Orakel fuer UNSAT
\DeclareMathOperator{\comptru}{\comp_{tR(\Us)}} % Komplexitaet bzgl. baumartiger Resolution unter Verwendung des Orakel Us
\DeclareMathOperator{\compru}{\comp_{R(\Us)}}
\providecommand{\comptruv}[1]{\comp_{\mr{tR}(#1)}}
\providecommand{\compruv}[1]{\comp_{\mr{R}(#1)}}
\newcommand{\Sa}{\mc{S}} % September 2001
% Hardness measurements:
\DeclareMathOperator{\hardness}{hd}
\DMO{\thardness}{thd} % tree-hardness (= hardness)
\DMO{\phardness}{phd} % propagation-hardness
\DeclareMathOperator{\wid}{wid} % symmetric width
\DMO{\whardness}{awid} % width-based hardness (asymmetric width)
\DMO{\dep}{dep} % depth
\DMO{\hts}{hs} % Horton-Strahler number
\DMO{\semspace}{css} % semantic clause-space complexity
\DMO{\resspace}{crs} % resolution clause-space complexity
\DMO{\treespace}{cts} % tree clause-space complexity of resolution
%
% Schubfachformeln (pigeonhole formulas):
\newcommand{\php}{\mathrm{PHP}}
\newcommand{\fphp}{\mathrm{FPHP}} % mit AMO-Klauseln
\newcommand{\ophp}{\mathrm{OPHP}} % jedes Fach verwendet; every hole is used
\newcommand{\ofphp}{\mathrm{BPHP}} % bijektive Form
\newcommand{\ephp}{\mathrm{EPHP}} % PHP with extended-resolution clauses from Cook1976short
%
% Pebbling:
\DeclareMathOperator{\pebf}{PF} % pebbling formulas (veraendert am 18.5.2001; vormals \pf)
%
% Baeume; trees:
%
\DeclareMathOperator{\rt}{rt} % Wurzel; root
\DeclareMathOperator{\nds}{nds} % Menge der Knoten; set of nodes
\DeclareMathOperator{\lvs}{lvs} % Menge der Blaetter; set of leaves
\DeclareMathOperator{\nlvs}{\#lvs} % Anzahl der Blaetter; number of leaves
\DeclareMathOperator{\nnds}{\#nds} % Anzahl der Knoten; number of nodes
\DeclareMathOperator{\height}{ht}
\DeclareMathOperator{\depth}{d}
\DeclareMathOperator{\cls}{cls}
\DeclareMathOperator{\newcommandls}{\#cls}
\DeclareMathOperator{\ds}{ds}
\DeclareMathOperator{\dst}{ds_T}
\DeclareMathOperator{\dsg}{ds_G}
\DeclareMathOperator{\dpr}{dp}
\DeclareMathOperator{\dprt}{dp_T}
\DeclareMathOperator{\dprg}{dp_G}
\DeclareMathOperator{\ind}{in}
\DeclareMathOperator{\indg}{in_G}
\DeclareMathOperator{\outd}{out}
\DeclareMathOperator{\outdg}{out_G}
%
\DeclareMathOperator{\peb}{peb} % 20.2.2000: pebbling complexity
%
\newcommand{\pab}[1]{\langle #1 \rangle}
\newcommand{\pao}[2]{\langle #1 \ra #2 \rangle}
\newcommand{\pat}[4]{\langle #1 \ra #2, #3 \ra #4 \rangle}
%
% Verzweigungstupel:
%
\DeclareMathOperator{\taum}{\max \tau}
\DeclareMathOperator{\tauprob}{\tau^p} % probability distribution via tau
\DeclareMathOperator{\mtau}{\mf{T}} % associated mean
%
\newcommand{\Btnz}{\mc{BT}} % non-zero branching tuples
\newcommand{\Bt}{{\Btnz_{\! 0}}} % branching tuples
\newcommand{\Btk}[1]{{\Btnz^{(#1)}}} % non-zero branching tuples of length k
%
\DeclareMathOperator{\concatbt}{;} % concatenation of branching tuples
\DeclareMathOperator{\compobt}{\merge} % composition of branching tuples
\nc{\bth}[1]{\langle{#1}\rangle} % branching tuple hull
%
%
% Autarkien (autarkies):
%
% \DeclareMathOperator{\pc}{pc} % nicht mehr benutzt (13.5.2012); zu entfernen
%
\DeclareMathOperator{\aut}{Auk} % Autarkie-Monoid; autarky monoid
\DeclareMathOperator{\autf}{Auk^r} % finitised (or ``restricted'') autarky monoid
\DeclareMathOperator{\auts}{Auks} % autarky semigroup
\DeclareMathOperator{\autsf}{Auks^r} % finitised (or ``restricted'') autarky semigroup
\DeclareMathOperator{\autmax}{Auk\!\!\uparrow} % maximal autarkies
\DeclareMathOperator{\autmin}{Auk\!\!\downarrow} % minimal autarkies
\DeclareMathOperator{\autu}{Auk^s} % autark subsets
\DeclareMathOperator{\autsu}{Auks^s} % autark non-empty subsets
%
\DeclareMathOperator{\laut}{LAuk} % lineare Autarkien
\DeclareMathOperator{\lautz}{LAuk_0} % einfache lineare Autarkien
\DeclareMathOperator{\maut}{MAuk}
\newcommand{\A}{\mc{A}} % Autarkie-System; autarky system
\DeclareMathOperator{\nv}{N} % ``Normalform'' (schlanker Kern) i.a.; ``normal form'' (lean kernel) in general
\DeclareMathOperator{\na}{\nv_a} % schlanker Kern; lean kernel
\DeclareMathOperator{\nA}{\nv_{\A}} % Normalform fuer Autarkiesystem A; leank kernel for autarky system A
\DeclareMathOperator{\nla}{\nv_{la}}
\DeclareMathOperator{\nbla}{\nv_{bla}}
\DeclareMathOperator{\nma}{\nv_{ma}}
\DeclareMathOperator{\npa}{\nv_{pa}}
\DeclareMathOperator{\baut}{BAuk} % balancierte Autarkien
\DeclareMathOperator{\blaut}{BLAuk} % balancierte lineare Autarkien
\DeclareMathOperator{\blautz}{BLAuk_0} % balancierte lineare Autarkien
\DeclareMathOperator{\paut}{PAut} % pure Autarkien
\DeclareMathOperator{\pautz}{PAut_0} % einfache pure Autarkien
\newcommand{\SatA}{\Sat\!_{\A}}
\newcommand{\LeanA}{\Lean\!_{\A}}
%
% Relativierte Resolution:
%
\DeclareMathOperator{\resouz}{\overset{\Us, 0}{\vdash}}
\DeclareMathOperator{\resouo}{\overset{\Us, 1}{\vdash}}
\DeclareMathOperator{\resouk}{\overset{\Us,\, k}{\vdash}}
\DeclareMathOperator{\resou}{\,\overset{\Us}{\vdash}\,}
\DeclareMathOperator{\resour}{\,\overset{\Us_0}{\vdash}\,}
\DeclareMathOperator{\resourz}{\,\overset{\Us_0, 0}{\vdash}\,}
\DeclareMathOperator{\uresouk}{\resouk\hspace{-0.6em}\mbox{\raisebox{0.8ex}{\tiny u}}}
\DeclareMathOperator{\bresouk}{\resouk\hspace{-0.6em}\mbox{\raisebox{0.8ex}{\tiny b}}}
\DeclareMathOperator{\iresouk}{\resouk\hspace{-0.6em}\mbox{\raisebox{0.8ex}{\tiny i}}}
\DeclareMathOperator{\resok}{\overset{k}{\vdash}} % 27.8.2000
\newcommand{\resokv}[1]{\overset{#1}{\vdash}}
\newcommand{\resoukv}[1]{\overset{\Us, \,#1}{\vdash}}
%
% Reductions:
%
\DMO{\rsub}{r_S} % subsumption-elimination
\DMO{\rk}{r} % propagation-reduction
\DMO{\ro}{\rk_1} % unit-clause propagation
\DMO{\rki}{\rk_{\infty}} % the limit of r_k
\DMO{\rpl}{r^{pl}} % elimination of pure literals
\DMO{\ropl}{\rk_1^{pl}} % r_1 with elimination of pure literals
%
% Slur:
%
\nc{\rslur}{\xrightarrow{\text{SLUR}}} % SLUR-reduction
\nc{\rslurs}{\rslur_{\!*}} % reflexive-transitive closure
\DMO{\slur}{slur} % set of fully reduced results of SLUR-reduction
\nc{\Slur}{\mc{SLUR}} % the SLUR-class
\nc{\rkslur}[1]{\xrightarrow{\text{SLUR}_{#1}}} % k-SLUR-reduction
\nc{\rkslurs}[1]{\rkslur{#1}_{\!*}} % reflexive-transitive closure
% Alternative SLUR hierarchies:
\nc{\Altsluri}[1]{\Slur(#1)}
\nc{\Altslurstari}[1]{\Slur\text{\textasteriskcentered}(#1)}
\nc{\Canoni}[1]{\mr{CANON}(#1)}
\nc{\rkslurstar}[1]{\xrightarrow{\text{SLUR\textasteriskcentered}#1}} % k-SLUR*-reduction
\nc{\rkslursstar}[1]{\rkslurstar{#1}_{\!*}} % reflexive-transitive closure
\DMO{\slurstar}{\slur\!\text{\textasteriskcentered}}
% Unit-refutation-completeness:
\nc{\Urefc}{\mc{UC}}
% Propagation-completeness:
\nc{\Propc}{\mc{PC}}
% Width-refutation-completeness:
\nc{\Wrefc}{\mc{WC}} % class of clause-sets of w-hardness k
%
\DeclareMathOperator{\widl}{\hspace*{-1.5pt}wid}
\DeclareMathOperator{\widb}{\sideset{^{\mr{b}}}{}\widl}
\DeclareMathOperator{\widi}{\sideset{^{\mr{i}}}{}\widl}
\DeclareMathOperator{\cwid}{\mc{W}} % classes
\DeclareMathOperator{\cwidl}{\hspace*{-1pt}\mc{W}} % classes
\DeclareMathOperator{\cwidb}{\sideset{^{\mr{b}}}{}\cwidl}
\DeclareMathOperator{\cwidi}{\sideset{^{\mr{i}}}{}\cwidl}
%
\DeclareMathOperator{\modp}{mod_p} % partielle Modelle
\DeclareMathOperator{\modt}{mod_t} % totale Modelle
\DeclareMathOperator{\moda}{\mf{S}} % 15.1.2003; ``allgemeine Modelle''
\DeclareMathOperator{\modf}{fal} % 27.11.2004; ``falsifizierende Modelle''; von \mf{F} zu \ol{mod} geaendert am 21.10.2006; zu ``fal'' geaendert am 27.9.2007
\DeclareMathOperator{\mods}{mod} % 21.10.2006; ``erfuellende Modelle'' (satisfying models)
%
% Systems of Problem Instances:
%
\newcommand{\PI}{\mc{PI}}
\newcommand{\Spi}{\mc{S}\PI}
\newcommand{\Upi}{\,\mc{U}\PI}
%
% Hypergraphs of minimally unsatisfiable sub-clause-sets etc.:
%
\DeclareMathOperator{\mus}{MU}
\DeclareMathOperator{\mss}{MS}
\DeclareMathOperator{\cmus}{CMU}
\DeclareMathOperator{\cmss}{CMS}
% 13.5.2007: Verallgemeinerung fuer beliebige Klm
\DeclareMathOperator{\eqs}{EQ} % aequivalente Teilmengen
\DeclareMathOperator{\neqs}{NEQ} % nicht-aequivalente Teilmengen
%
% Konfliktmatrizen und Konfliktgraphen; conflict matrices and conflict graphs:
%
\DeclareMathOperator{\scf}{CM} % symmetric conflict matrix
\DeclareMathOperator{\acf}{DCM} % asymmetric conflict matrix
\DeclareMathOperator{\cmg}{cmg} % conflict multigraph
\DeclareMathOperator{\cmdg}{cmdg} % conflict multidigraph
\DeclareMathOperator{\cg}{cg} % conflict graph
\DeclareMathOperator{\gcg}{cgg} % general conflict graph (allgemeiner Konfliktgraph)
\DeclareMathOperator{\gcdg}{cgdg} % general conflict digraph (allgemeiner gerichteter Konfliktgraph)
%
\DeclareMathOperator{\rsg}{rg} % resolution graph
\DeclareMathOperator{\srsg}{srg} % resolution graph without subsumptions
%
\DeclareMathOperator{\vhg}{vhg} % variable hypergraph
\DeclareMathOperator{\cvg}{cvg} % common-variable graph (clauses joined by an edge of some variable in common)
\DeclareMathOperator{\cvmg}{cvmg} % common-variable multi graph
\DeclareMathOperator{\vig}{vig} % variable interaction graph
\DeclareMathOperator{\vcg}{vcg} % variable-clause graph
%
\DeclareMathOperator{\nscf}{bcp} % symmetric conflict number
\DeclareMathOperator{\nacf}{bcp_d} % asymmetric conflict number
\DeclareMathOperator{\bcp}{bcp} % biclique partition number
%
\DeclareMathOperator{\tbcp}{tbcp} % ``triviale'' Biclique-Zerlegung; trivial biclique partition
%
% Grade und Raenge:
%
\DeclareMathOperator{\nsat}{\#sat}
\DeclareMathOperator{\nusat}{\#usat}
\DeclareMathOperator{\maxsat}{maxsat}
%
\DeclareMathOperator{\pmin}{\rankmin}
\DeclareMathOperator{\pmax}{\rankmax}
\DeclareMathOperator{\pav}{\rankdur}
%
\DeclareMathOperator{\ldeg}{ld} % Literalgrad; literal degree
\DeclareMathOperator{\minldeg}{\mu\!\ldeg} % minimaler Literalgrad; minimal literal degree
\DeclareMathOperator{\maxldeg}{\nu\ldeg} % maximaler Literalgrad; maximal literal degree
%
\DeclareMathOperator{\vdeg}{vd} % Variablengrad; variable degree
\DeclareMathOperator{\minvdeg}{\mu\!\vdeg} % minimaler Variablengrad; minimal variable degree
\DeclareMathOperator{\maxvdeg}{\nu\!\vdeg} % maximaler Variablengrad; maximal variable degree
\DeclareMathOperator{\avvdeg}{\widetilde{\vdeg}} % durchschnittlicher Variablengrad; average variable degree
%
\DeclareMathOperator{\cldeg}{cldg} % complementary literal degree
%
% Minimaler Variablengrad von k-uniformen minimal unerfuellbaren booleschen Klm
\DeclareMathOperator{\mvardu}{\mu\!\vdeg}
\DMO{\varmvd}{\var_{\minvdeg}} % variables with minimal degree
\DMO{\nfc}{fc} % number of full clauses
\DMO{\maxnfc}{\nu\!\nfc} % maximal number of full clauses
%
%
%
% Spezielle Klauselmengen:
%
\nc{\Dt}[1]{\mc{F}_{#1}} % the saturated clause-sets of deficiency 2 are the $\Dt{n}$
\DeclareMathOperator{\Inj}{Inj}
%
% OKlibrary:
%
\newcommand{\OKsolver}{\texttt{OKsolver}}
\newcommand{\OKlibrary}{\texttt{OKlibrary}}
\newcommand{\OKgenerator}{\texttt{OKgenerator}}
\newcommand{\OKplatform}{\texttt{OKplatform}}
\newcommand{\OKdatabase}{\texttt{OKdatabase}}
\newcommand{\OKsystem}{\texttt{OKsystem}}
\newcommand{\OKinternet}{\url{http://www.ok-sat-library.org}}
\newcommand{\OKramsey}{\url{http://cs.swan.ac.uk/~csoliver/ok-sat-library/internet_html/doc/local_html/RamseyTheory.html}}
%
% Umfeld der minimalen Unerfuellbarkeit; context of MU:
%
\DeclareMathOperator{\Ex}{Ex} % Expansion; expansion
\DeclareMathOperator{\surp}{\sigma} % Surplus
\DeclareMathOperator{\nonmer}{nM} % Nicht-Mersenne-Zahl; non-Mersenne number
\newcommand{\inonmer}{\operatorname{i}_{\mathrm{nM}}} % Index der nM-Zahl; index of nM-number
\nc{\svbf}{\mc{VB}} % set of valid bounds-functions
\nc{\svbfs}{\mc{VB}^*} % those <= nM
\DMO{\potp}{pp} % potential pairs
\DMO{\potprec}{NM} % recursion via potential pairs
\DMO{\minnonmer}{VDM} % minimal variable degree for MU(k)
\DMO{\minnonmerh}{VDH} % minimal variable degree for UHIT(k)
\DMO{\maxsmar}{FCM} % maximal number of full clauses for MU(k)
\DMO{\maxsmarh}{FCH} % maximal number of full clauses for UHIT(k)
\newcommand{\Esnm}{\mc{SN\!M}} % k with nM(k)=S(k)
%
% Singularitaet; singularity:
\DMO{\varsing}{\var_s} % singular variables
\DMO{\varosing}{\var_{1s}} % 1-singular variables
\DMO{\varnosing}{\var_{\neg1s}} % non-1-singular variables
\nc{\Musatns}{\Musat'} % non-singular minimally unsatisfiable clause-sets
\nc{\Musatnsi}[1]{\Musati{#1}'}
\nc{\Smusatns}{\Smusat'} % non-singular saturated minimally unsatisfiable clause-sets
\nc{\Smusatnsi}[1]{\Smusati{#1}'}
\nc{\Uclashns}{\Uclash'} % non-singular hitting clause-sets
\nc{\Uclashnsi}[1]{\Uclashi{#1}'}
% Singular DP-reduction:
\nc{\tsdp}{\xrightarrow{\text{sDP}}}
\nc{\tsdps}{\tsdp_{\!*}}
\nc{\tosdp}{\xrightarrow{\text{1sDP}}}
\nc{\tosdps}{\tosdp_{\!*}}
\DMO{\sdp}{sDP} % the set of reachable non-singular clause-sets
\DMO{\osdp}{sDP_1} % the normalform obtained by 1sDP-reduction
\nc{\cflmusat}{\mc{CF}\Musat} % F in MU with confluent singular DP-reduction
\nc{\cflmusati}[1]{\mc{CF}\Musati{#1}}
\nc{\cflimusat}{\mc{CFI}\Musat} % F in MU with confluent singular DP-reduction
\DMO{\sNF}{sNF} % singular normal form
\DMO{\eqp}{eqp} % equality-preserving permutations
\DMO{\sgp}{sp} % singularity-preserving permutations
\DMO{\singind}{si} % singularity index
\DMO{\osingind}{si_1} % 1-singularity index
\DMO{\shyp}{svh} % singularity-hypergraph
\DMO{\sdph}{ssh} % sDP-hypergraph (hypergraph of singular sets)
\DMO{\msdph}{mss} % maximal singular sets
\DMO{\osdph}{ssh_1} % hypergraph of 1-singular sets
\DMO{\mosdph}{mss_1} % maximal 1-singular set
%
% Minimale Praemissenmengen (minimal premise sets):
\newcommand{\Mps}{\mathcal{MPS}} % set of minimal premise clause-sets
\DMO{\mps}{mps} % set of minimal-premise-sub-clause-sets
\DMO{\purec}{puc} % pure clause
\DMO{\doping}{D}
% Repraesentierung boolescher Funktionen; representations of boolean functions:
%
\DeclareMathOperator{\primec}{prc} % Prim-Klauseln; prime clauses (\primec_0 fuer CNF, \primec_1 fuer DNF)
\DeclareMathOperator{\vcan}{vct} % variables of canonical translation
\DeclareMathOperator{\cant}{ct} % canonical translation
\DeclareMathOperator{\fcanoe}{fce} % full canonical 1-extension
\DeclareMathOperator{\fcant}{fct} % full canonical translation
\DeclareMathOperator{\fcanpt}{fctp} % full canonical p-translation
%
% Kombinatorische Operationen mit Klauselmengen
%
\nc{\glue}[4]{\operatorname{glue}((#1,#2), (#3,#4))} % glueing
\newcommand{\fglue}{\mathbin{\boxplus}} % full gluing
\newcommand{\fvdglue}{\mathbin{\widetilde{\boxplus}}} % full variable-disjoint gluing
\nc{\gluea}[3]{#1 \mathbin{\boxplus}_{#3} #2} % gluing with identification
\newcommand{\cor}{\mathbin{\ovee}} % combinatorial disjunction
%
% ########################################################
% ----------------------------------------------------------------------------------------------------------------
% BOOLEAN FUNCTIONS (Boolesche Funktionen)
% -----------------------------------------------------------------------------------------------------------------
% ########################################################
%
\newcommand{\Bf}[1]{\mc{BF}^{#1}} % Menge der booleschen Funktionen; set of boolean functions
\DeclareMathOperator{\boolf}{bf} % zugehoerige boolesche Funktion
\DeclareMathOperator{\essent}{es} % essentieller Teil; essential part
\DeclareMathOperator{\unforced}{uf} % nicht-erzwungener Teil; unforced part
\DeclareMathOperator{\essunf}{eu} % essentieller nicht-erzwungener Teil; essential unforced part
\DeclareMathOperator{\booldiff}{\sm} % boolesche Differenz; boolean difference
\DMO{\frl}{fl} % forced literals
%
% ########################################################
% ----------------------------------------------------------------------------------------------------------------
% COMPLEXITY THEORY (Komplexitaetstheorie)
% -----------------------------------------------------------------------------------------------------------------
% ########################################################
%
% Platz- und Zeitklassen i.a.
%
\DeclareMathOperator{\timem}{time}
\DeclareMathOperator{\spacem}{space}
\DeclareMathOperator{\dtime}{DTime}
\DeclareMathOperator{\dspace}{DSpace}
\DeclareMathOperator{\ndtime}{NTime}
\DeclareMathOperator{\ndspace}{NSpace}
\DeclareMathOperator{\condtime}{coNTime}
\DeclareMathOperator{\condspace}{coNSpace}
%
\nc{\Con}{\mr{Con}}
\nc{\Log}{\mr{Log}}
\nc{\Lin}{\mr{Lin}}
\nc{\Pol}{\mr{Pol}}
\nc{\ExL}{\mr{ExL}}
\nc{\ExP}{\mr{ExP}}
%
\nc{\CTime}{\mr{CTime}}
\nc{\CSpace}{\mr{CSpace}}
%
\nc{\LTime}{\mr{LTime}}
\nc{\LSpace}{\mr{L}}
\nc{\NLSpace}{\mr{NL}}
%
\nc{\LinTime}{\mr{LinTime}}
\nc{\LinSpace}{\mr{LinSpace}}
%
\nc{\PTime}{\mr{P}}
\nc{\PSpace}{\mr{PSpace}}
%
\nc{\Np}{\mr{NP}}
\nc{\Conp}{\text{coNP}}
\nc{\NPSpace}{\mr{NPSpace}}
\nc{\CoNPSpace}{\mr{coNPSpace}}
%
\nc{\ELTime}{\mr{ELTime}}
\nc{\ELSpace}{\mr{ELSpace}}
\nc{\EPTime}{\mr{EPTime}}
\nc{\EPSpace}{\mr{EPSpace}}
\nc{\NEPTime}{\mr{NEPTime}}
%
% Polynomiale Hierarchie
%
\nc{\polydelta}[1]{\Delta_{#1}^{\mr P}}
\nc{\polypi}[1]{\Pi_{#1}^{\mr P}}
\nc{\polysigma}[1]{\Sigma_{#1}^{\mr P}}
\nc{\Ph}{\mr{PH}}
\DeclareMathOperator{\exP}{\ex^{\mr P}}
\DeclareMathOperator{\faP}{\fa^{\mr P}}
%
% Boolesche Hierarchie
%
\nc{\Dp}{D^P}
%
% Parallelitaet
%
\nc{\PllC}[2]{{\text{$\mr{PT}$/$\mr{WK}$}(#1, #2)}} % uniforme Schaltkreise der Tiefe O(#1) und der Groesse O(#2)
\nc{\Nc}{\mr{NC}}
\nc{\Nci}[1]{\Nc^{#1}}
\nc{\Ac}{\mr{AC}}
\nc{\Aci}[1]{\Ac^{#1}}
%
% Nichtuniforme Komplexitaet
\nc{\pmodpoly}{P / \mathrm{poly}}
%
% Parametrisierung
%
\nc{\Wh}[1]{\mr{W}[#1]} % W-Hierarchie
%
% Randomisierung
%
\nc{\Rl}{\mr{RL}}
\nc{\coRl}{\mr{coRL}}
\nc{\Rp}{\mr{RP}}
\nc{\coRp}{\mr{coRP}}
\nc{\Zpp}{\mr{ZPP}}
\nc{\Bpp}{\mr{BPP}}
\nc{\Pp}{\mr{PP}}
%
% Probleme
%
\nc{\Reach}{\mr{STCON}} % gerichtete Erreichbarkeit
\nc{\Undreach}{\mr{USTCON}} % ungerichtete Erreichbarkeit
\nc{\Pcol}[2]{\mr{COL}(#1,#2)} % Hypergraphenfaerbung von #1-uniformen Hypergraph mit #2 Farben
\nc{\Pscol}[2]{\mr{SCOL}(#1,#2)} % starke Hypergraphenfaerbung von #1-uniformen Hypergraph mit #2 Farben
\nc{\Psorcol}[2]{\mr{SORCOL}(#1,#2)} % sortierte Hypergraphenfaerbung von #1-uniformen Hypergraph mit #2 Farben
%
% Algebraische Komplexitaet (algebraic complexity):
%
\DMO{\slp}{slp}
%
% ########################################################
% ----------------------------------------------------------------------------------------------------------------
% CRYPTOLOGY (Kryptologie)
% -----------------------------------------------------------------------------------------------------------------
%
% Blockchiffren
%
\nc{\Mss}{\mr{MSS}}
\nc{\Key}{\mr{KEY}}
\nc{\Keyi}[1]{\Key_{\!#1}}
\nc{\Nbmss}{N_{\mr{bm}}} % Anzahl Bits in Botschaften
\nc{\Nbkey}{N_{\mr{bk}}} % Anzahl Bits in Schluesseln
%
% AES
%
\nc{\Rnb}{N_{\mr{b}}}
\nc{\Rnk}{N_{\mr{k}}}
\nc{\Rnr}{N_{\mr{r}}}
\DeclareMathOperator{\Rijn}{RIJN}
\DeclareMathOperator{\byte}{byte}
\DeclareMathOperator{\qbyte}{qbyte}
\nc{\Byte}{\mr{B}[8]}
\nc{\QByte}{\mr{B}[4,8]}
\nc{\KByte}{\mc{B}} % der K\"orper der Bytes
\nc{\RQByte}{\mc{QB}} % der Ring der 4-Byte-Bloecke
\DeclareMathOperator{\Rrk}{rk} % round key
\DeclareMathOperator{\Rsm}{sm} % state mix
\DeclareMathOperator{\Rsubbytes}{\mathtt{SubBytes}}
\DeclareMathOperator{\Rshiftrows}{\mathtt{ShiftRows}}
\DeclareMathOperator{\Rmixcolumns}{\mathtt{MixColumns}}
\DeclareMathOperator{\Rsb}{S_{RD}} % S-box
%
% ##################################
% # RAMSEY THEORY (Ramsey-Theorie) #
% ##################################
%
% Ramsey-Zahlen und aehnliches; Ramsey numbers and other similar numbers
\nc{\ramz}[3]{\mr{ram}_{#1}^{#2}(#3)} % Ramsey-Zahl; Ramsey number
\DeclareMathOperator{\ramzg}{ram} % generische Funktion fuer Ramsey-Zahlen; generic function for Ramsey numbers
\nc{\waez}[2]{\mr{vdw}_{#1}(#2)} % van der Waerden-Zahl; van der Waerden number
\DeclareMathOperator{\waezg}{vdw} % generische Funktion fuer van der Waerden-Zahlen; generic function for van der Waerden numbers
\nc{\gtz}[2]{\mr{grt}_{#1}(#2)} % Green-Tao-Zahl; Green-Tao number
\DeclareMathOperator{\gtzg}{grt} % generische Funktion fuer Green-Tao-Zahlen; generic function for Treen-Tao numbers
\nc{\pdwaez}[2]{\mr{vdw}_{#1}^{\mr{pd}}(#2)} % palindromische vdW-Zahlen; palindromic vdW-numbers
\DeclareMathOperator{\schurz}{shr} % Schur-Zahl
\DeclareMathOperator{\wschurz}{wshr} % schwache Schur-Zahl
%
\DeclareMathOperator{\FvdW}{F_{W}} % Klauselmenge, die die van-der-Waerden-Zahl darstellt; clause-set corresponding to van der Waerden numbers
\DeclareMathOperator{\FRam}{F_{R}} % Klauselmenge, die Ramsey-Zahl darstellt; clause-sets corresponding to Ramsey numbers
\DeclareMathOperator{\arithp}{ap} % Menge aller arithmetischen Progressionen; set of all arithmetic progressions.
\DeclareMathOperator{\arithpp}{ap_{pr}} % Menge aller arithmetischen Progressionen in der Menge der Primzahlen; set of all arithmetic progressions in the primes.
\DeclareMathOperator{\crarithp}{cr_{ap}} % Konvergenzrate der relativen Unabhaengigkeitszahl von arithp(k,n); convergence rate of relative independence number of arithp(k,n)
\DeclareMathOperator{\crarithpp}{cr_{ap}^{pr}} % Konvergenzrate der relativen Unabhaengigkeitszahl von arithpp(k,n); convergence rate of relative independence number of arithpp(k,n)
%
%
% ####################################################
% # PROBABILITY THEORY (Wahrscheinlichkeits-Theorie) #
% ####################################################
%
\DeclareMathOperator{\prob}{Pr}
%
%
%
% ######################
% # NUMERICS (Numerik) #
% ######################
%
\nc{\absfeh}[1]{\delta_{#1}} % absoluter Fehler; absolute error
\nc{\relfeh}[1]{\ve_{#1}} % relativer Fehler; relative error
%
\newcommand{\Flpp}{\mr{Flpp}} % Gleitkommazahldarstellungen; floating point number representations
\DeclareMathOperator{\rflp}{flp} % die reelle Zahl zu einer Gleitkommazahldarstellungen; the real number for a floating point number representations
\DeclareMathOperator{\meps}{\epsilon} % die Maschinengenauigkeit; machine epsilon
\DeclareMathOperator{\Nmin}{N_{min}} % die kleinste positive normale Gleitkommazahl; smallest normalised floating point number
\DeclareMathOperator{\Nmax}{N_{max}} % die groesste positive normale Gleitkommazahl; largest normalised floating point number
\DeclareMathOperator{\Nrg}{Nrg} % normale Bereich; normal range
%
%
% ########################################################
%%% Local Variables:
%%% mode: latex
%%% TeX-parse-self: t
%%% TeX-auto-save: t
%%% TeX-master: "Definitionen"
%%% End:
