% 23.12.1999
% Definitionsdatei fuer normale Texte
%
\input Latex_macros/Basis.tex
%
\usepackage{theorem} % am 18.5.2001 von ``Basis'' hierhin verschoben
\usepackage[driverfallback=hypertex]{hyperref} % fuer ARXIV ohne Option!
%
%\parskip1ex plus0.5ex minus0.2ex
\renewcommand{\thefootnote}{\arabic{footnote})}%
%
\newtheorem{defi}{Definition}[section]
\newtheorem{lem}[defi]{Lemma}
\newtheorem{thm}[defi]{Theorem}
\newtheorem{corol}[defi]{Corollary}
\newtheorem{corold}[defi]{Korollar}
\newtheorem{propo}[defi]{Proposition}
\newtheorem{exerc}[defi]{Exercise}
\newtheorem{exercd}[defi]{{\"U}bung}
\newtheorem{conj}[defi]{Conjecture}
\newtheorem{conjd}[defi]{Vermutung}
\newtheorem{examp}[defi]{Example}
\newtheorem{exampd}[defi]{Beispiel}
\newtheorem{quest}[defi]{Question}
\newtheorem{questd}[defi]{Frage}
\newtheorem{spec}[defi]{Speculation}
\newtheorem{specd}[defi]{Spekulation}
\newtheorem{oprbl}[defi]{Open Problem}
\newtheorem{oprbld}[defi]{Offenes Problem}
%\newtheorem{claim}[defi]{Claim}
% 
\theorembodyfont{\rmfamily}
\newtheorem{auf}{Aufgabe}[section]
\theorembodyfont{}
%
\newenvironment{prf}{\noindent\textbf{Proof:}\;}{\par\noindent\ignorespacesafterend}
\newenvironment{prfd}{\noindent\textbf{Beweis:}\;}{\par\noindent\ignorespacesafterend}
\newcommand{\Qed}{\hfill $\square$}
% In mathematischer Latex-Umgebung (in Latex-math-environment):
\newcommand{\Qqed}{\quad \square}
%
% die folgenden Umgebungen wurden am 19.5.2001 hinzugefuegt fuer die
% (deutschsprachigen) Literaturbearbeitungen
% Nun obsolet.
%
\newcounter{dDef} % 16.11.2002; verschieden vom Argument(!)
\newenvironment{dDef}[1]{\refstepcounter{dDef}\begin{sloppypar}\noindent\textbf{Definition #1}\,\itshape}{\end{sloppypar}}
\newcounter{dLem} % 16.11.2002; verschieden vom Argument(!)
\newenvironment{dLem}[1]{\refstepcounter{dLem}\begin{sloppypar}\noindent\textbf{Lemma #1}\,\itshape}{\end{sloppypar}}
\newcounter{dThm} % 16.11.2002; verschieden vom Argument(!)
\newenvironment{dThm}[1]{\refstepcounter{dThm}\begin{sloppypar}\noindent\textbf{Theorem #1}\,\itshape}{\end{sloppypar}}
\newcounter{dPro} % 16.11.2002; verschieden vom Argument(!)
\newenvironment{dPro}[1]{\refstepcounter{dPro}\begin{sloppypar}\noindent\textbf{Proposition #1}\,\itshape}{\end{sloppypar}}
\newenvironment{dKor}[1]{\begin{sloppypar}\noindent\textbf{Korollar #1}\,\itshape}{\end{sloppypar}}
\newenvironment{dKoro}{\begin{sloppypar}\noindent\textbf{Korollar }\itshape}{\end{sloppypar}}
\newenvironment{dSat}[1]{\begin{sloppypar}\noindent\textbf{Satz #1}\,\itshape}{\end{sloppypar}}
\newcounter{Beispielzaehler}
\newenvironment{dBsp}{\begin{par}\noindent\textbf{Beispiele}\begin{list}{\arabic{Beispielzaehler}.}{\usecounter{Beispielzaehler}\setlength{\topsep}{0.0ex}\setlength{\itemsep}{0.0ex}}}{\end{list}\end{par}\smallskip}
\newenvironment{dEin}{\begin{par}\noindent\textbf{Einschub}}{\textbf{Ende}\end{par}}
% 15.8.2001
\newenvironment{dVer}{\begin{par}\noindent\textbf{Vermutung}}{\textbf{Ende}\end{par}}
%
% 26.9.2001 Von ``Basis'' hierher verschoben, um Kompatibilitaet mit
% foiltex zu gewaehrleisten:
%
\nc{\bm}{\boldmath}
\nc{\bmm}[1]{\mbox{\bm$\DST #1$}}% Math. Fettdruck im Text- wie im math. Modus
\nc{\mi}[1]{\bmm{\mathrm{(#1):}} \quad}% math item (verschoben am 28.1.2003)
%
%%% Local Variables:
%%% mode: latex
%%% TeX-parse-self: t
%%% TeX-auto-save: t
%%% TeX-master: t
%%% End:
