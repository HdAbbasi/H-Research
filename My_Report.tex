%--------------------------- Hoda Abbasi ----------------------------
%----------------------- 10 November 2015 ------------------------

\documentclass[12pt]{book}

\usepackage{a4}
\usepackage[graph]{xy}
\usepackage{graphicx}
\input Latex_macros/Definitionen.tex
\setcounter{tocdepth}{4}
\setcounter{secnumdepth}{4}
\usepackage{enumerate}
\usepackage[active]{srcltx}
\newcommand{\Schrift}{report}
%\newcommand{\Schrift}{book}
%--------------------------------------------------------------------------------------------------------------------------------------
\begin{document}
\title{\bf \Huge Report}

\author{ \bf Hoda Abbasi\\
             PhD Candidate\\
             Computer Science Department\\
             Swansea University\\}
\maketitle
%--------------------------------------------------------------------------------------------------------------------------------------
\tableofcontents
%***************************************************************************************************************************************
%***************************************************************************************************************************************
\chapter{Set Theory}
\label{cha:settheory}
%--------------------------------------------------------------------------------------------------------------------------------------
\section{Ontological Preparations}
\label{sec:Ontological preparations}

A \textbf{thing} is in general fuzzy, perhaps escapes precise definition at all. An \textbf{object} is a clearly defined, clearly outlined 
thing. A \textbf{theory} is about a realm of objects. \textbf{Mathematical objects} are for example various types of numbers and spaces. 
In set theory, every mathematical object is given to us as a \textbf{set}.
%--------------------------------------------------------------------------------------------------------------------------------------
\section{Naive Set Theory}
\label{sec:Naivesettheory}

Pure set theory has exactly one type of object, called ``set'', and thus everything is a set here  \cite{h1}. We find it useful to use an 
extension, where there are ``sets'' and urelements, that is, objects are either urelements (``atomic'') or sets.
\begin{examp}\label{exp:urelemente}
      If we want to assume ``numbers'', like $\NNZ \subset \NN \subset \ZZ \subset \QQ \subset \RR \subset \CC$, then we can consider them 
	  as urelements, that is, they are already given (from the outside), and they have no internal structure, that is, none of these numbers 
	  have elements; compare Section \ref{def:natnumberssimple} for the alternative approach of of defining $\NNZ = \omega$ via sets.
\end{examp}
Intuitively, a set is a ``collection'' of objects, that is, a ``collection'' of sets and urelements. ``Collection'' here means a set $x$ 
(not an urelement), such that for every object $y$ it can be said, whether $y$ is an element of $x$, that is, ``\bmm{y \in x}'', or $y$ is 
not an element of $x$, that is, ``\bmm{y \notin x}''. Thus there is no order on the elements of a set, and an object can not be multiple 
times in a set (it is just ``in or out''). Two objects $x, y$ can be compared for equality, that is, we either have \bmm{x = y} or \bmm{x \ne y}. 
Since a set is given by its elements, we have $x = y$ for sets $x, y$ iff for all objects $z$ holds $z \in x \Lra z \in y$, while we have $x \ne y$ 
iff there is an object $z$ with $z \in x$ but $z \notin y$, or $z \in y$ but $z \notin x$. For an urelement $x$ and every object $y$ holds $y \notin x$. 
Urelements can be compared by equality, where this relation is atomic (given).

\begin{defi}\label{def:sse}
      For sets $x, y$ holds \bmm{x \sse y} if for all objects $z \in x$ holds $z \in y$, while \bmm{x \subset y} holds if $x \sse y$ and $x \ne y$.
\end{defi}

Remarks:
\begin{enumerate}
      \item For any set $x$, we define the \textbf{complement} of $x$, denoted by $x'$, to be the set $x' = \{ z : z \not \in x \}$.
      \item So $x = y$ iff $x \sse y$ and $y \sse x$.
      \item And $x \subset y$ iff for all elements $z \in x$ holds $z \in y$, while there is a $z' \in y$ with $z' \notin x$.
      \item A set $x$ is called a \textbf{proper subset} of a set $y$ if $x \subset y$ but $x \not = y$ \cite{h3}. 
	  \item \cite{h2} uses ``$\subset$'' instead of ``$\sse$''.
\end{enumerate}
%--------------------------------------------------------------------------------------------------------------------------------------
\subsection{Set Formation}
\label{sec:setformation}

Given objects $x, y$ we can form the \textbf{singleton} \bmm{\set{x}} and the \textbf{2-set} \bmm{\set{x,y}}, that is, sets, where for every object $z$ holds:
\begin{enumerate}
      \item $z \in \set{x} \Lra z = x$;
      \item $z \in \set{x,y} \Lra z = x \vee z = y$.
\end{enumerate}

More fundamentally, there is the \textbf{empty set} \bmm{\es}, characterised by the property, that for all objects $x$ holds $x \notin \es$.

Given sets $x, y$, we can form three further sets:
\begin{itemize}
      \item the \textbf{union} \bmm{x \cup y}, characterised by $\fa\, z : z \in x \cup y \Lra z \in x \vee z \in y$;
      \item the \textbf{intersection} \bmm{x \cap y}, characterised by $\fa\, z : z \in x \cap y \Lra z \in x \wedge z \in y$;
      \item the \textbf{(set-)difference} \bmm{x \sm y}, characterised by $\fa\, z : z \in x \sm y \Lra z \in x \wedge z \notin y$.
\end{itemize}
Remarks:
\begin{enumerate}
      \item Intersection and union operation are {\it commutative} and {\it associative}.
      \item For sets $x, y, z$ we have the following properties \cite{h1}.
      \begin{enumerate}
            \item $x - \emptyset = x\ ;\ x - x = \emptyset $. 
            \item $x \cup x = x\ ;\ x \cap x = x $.
            \item $x \subseteq x \cup y\ ;\ x \cap y \subseteq x$.
            \item $x \cup (y \cap z) = (x \cup y) \cap (x \cup z)$.
            \item $x \cap (y \cup z) = (x \cap y) \cup (x \cap z)$.
            \item $(x \subseteq y) \leftrightarrow (x \cap y=x) \leftrightarrow (x \cup y = y)$.
            \item $x - y = x - (x \cap y)$.
      \end{enumerate}
      \item Two sets are said to be \bmm{disjoint} if they have no member in common; in symbols \cite{h1},
      $$ x \cap y = \emptyset .$$
      \item  For sets $x, y$ the \textbf{De Morgan's Laws} are:
	  \begin{enumerate}
	        \item $ ( x \cup y)' = x' \cap y' $;
			\item $ ( x \cap y)' = x' \cup y' $.
	  \end{enumerate}
	  \item If $x, y $ are subset of $z$ , we have the following properties:
      \begin{enumerate}
	        \item $z - ( z - x ) = x$;
	        \item $(x \subseteq y) \leftrightarrow [(z - y) \subseteq (z - x)]}$;
	        \item $x \cup (z - x) = z$;
	        \item $z - (x \cup y) = (z - x) \cap (z - y)$;
	        \item $z - (x \cap y) = (z - x) \cup (z - y)$.
	  \end{enumerate}
      \item From the {\it axiom of extensionality}, it is proved that there is only one empty set \cite{h1}.
\end{enumerate}
%--------------------------------------------------------------------------------------------------------------------------------------
\subsection{(Ordered) Pairs}
\label{sec:ordpairs}

\begin{defi}\label{def:pairs}
      For objects $x, y$ we define the \textbf{pair} \bmm{(x,y)} via
      \begin{displaymath}
            (x,y) := \set{\set{x},\set{x,y}}.
      \end{displaymath}
      A set $x$ is called \textbf{a pair}, if there are objects $y,z$ with $x = (y,z)$.
\end{defi}
Remarks:
\begin{enumerate}
      \item If $x=y$, then $(x,y) = \set{\set{x}}$.
      \item For example, $\es$ is not a pair.
\end{enumerate}
\begin{lem}\label{lem:pairs}
      For objects $x, y, x', y'$ holds $(x,y) = (x',y') \Lra x = x' \wedge y = y'$.
\end{lem}
\begin{defi}\label{def:projpairs}
      For a pair $x$ we define the object \bmm{\proj_1(x)}, \bmm{\proj_2(x)} as those unique objects with $x = (\proj_1(x), \proj_2(x))$ (``projections'').
\end{defi}

%--------------------------------------------------------------------------------------------------------------------------------------
\subsection{Binary Operation}
\label{sec:Binary Operation}

\begin{defi}\label{def:bop}
      The \textbf{binary operation $P$} or \textbf{law of composition} on a set $x$ is the set 
	  $$ \{ (a,b) \mid a \in x \wedge b \in x \wedge P(x,y) \}.$$
\end{defi}
In other words, the binary operation on set $x$ is a rule which for every two elements of $x$ gives another element of $x$.
For example, addition is a binary operation on $\RR$, because given any two real numbers, their sum is a real number.
If $R$ is a binary relation on a set $x$, we write $ \bmm{R(a,b)}$ or $\bmm{(aRb)}$ \cite{h1}.
%--------------------------------------------------------------------------------------------------------------------------------------
\subsection{Product of Sets}
\label{sec:productsets}

\begin{defi}\label{def:productsets}
  For sets $X, Y$ the set of all pairs $x, y$ with $x \in X$ and $y \in Y$ exists and is denoted by
  \begin{displaymath}
    \bmm{X \times Y} := \set{(x,y): x \in X \wedge y \in Y}.
  \end{displaymath}
\end{defi}
Remarks:
\begin{enumerate}
      \item For sets $A, B, C, D$ we have the following properties.
      \begin{enumerate}
            \item\label{thm:propprod5} $A \times \emptyset = \emptyset \times A = \emptyset$.
            \item\label{thm:propprod1} $A \times (B \cap C) = (A\times B) \cap (A \times C)$.
            \item\label{thm:propprod2} $A \times (B \cup C) = (A \times B) \cup (A \times C)$.
            \item\label{thm:propprod3} $(A \times B) \cap (C \times D) = (A \cap C) \times (B \cap D)$.
            \item\label{thm:propprod4} $(A \times B) \cup (C \times D) \subseteq (A \cup C) \times (B \cup D)$.
      \end{enumerate}
\end{enumerate}
%--------------------------------------------------------------------------------------------------------------------------------------
\subsection{Infinitary Set Constructions}
\label{sec:Infinitary set constructions}

\begin{itemize}
      \item A \textbf{set system} is a set $X$, such that for all $y \in X$ holds that also $y$ is a set (all elements are sets). For a set system 
	  $X$ the \textbf{union} \bmm{\bc X} is characterised by $x \in \bc X \Lra \ex\, y \in X : x \in y$.
      \item For a set $x$ there is the \textbf{power set} \bmm{\pot(x)}, characterised by $y \in \pot(x) \Lra y \sse x$ for sets $y$ (the power set 
	  is the set of all subsets). So a powerset is a set system.
      \item If $x$ is a set, and $\Phi(y)$ a property of a set variable $y$ (a logical formula), then $\set{y \mb y \in x \wedge \Phi(y)}$ is a set, 
	  characterised by the property, that the elements of it are precisely those $y$ with $y \in x$ and $\Phi(y)$. In this way we can form specific 
	  subsets (via \textbf{comprehension}).
      \item If $X$ is a set, and $F(x)$ a specification of a unique object for all $x \in X$, then we have the set $\set{F(x) \mb x \in X}$, which 
	  is the set of all objects $y$, such that there is some $x \in X$ with $y = F(x)$. This is a form of image (via \textbf{replacement}).
\end{itemize}

\begin{examp}\label{exp:infset}
      For set $A= \{1,2\}$ the power set is : $\pot(A)=\{ \emptyset,\{1\},\{2\},\{1,2\}\}$.
\end{examp}
%--------------------------------------------------------------------------------------------------------------------------------------
\subsection{Relations and Maps}
\label{sec:maps}

\begin{defi}\label{def:reli}
      A \textbf{relation} is a set of pairs. The \textbf{first projection} of a relation $R$ is $\bmm{\proj_1(R)} := \set{\proj_1(p) : p \in R}$, 
	  the \textbf{second projection} is $\bmm{\proj_2(R)} := \set{\proj_2(p) : p \in R}$. For a relation $R$ and a set $X$ one defines 
	  $\bmm{R(X)}:= \set{y \in \proj_2(R) \mb \ex\, x \in X : (x,y) \in R}$ (the ``image'' of $X$ under $R$).
\end{defi}

\begin{defi}\label{def:reli2}
      A relation is \textbf{well-defined} if each element in the domain is assigned to a unique element in the range.
\end{defi}	  
Remarks:
\begin{enumerate}
      \item Instead of ``second projection'' one might also use ``range''. However for the first projection to use ``domain'' is misleading, since 
	  a relation does not need to have every element of the domain in its first projection (different from a map).

      The notations $\proj_i(R)$ for $i \in \tb{1,2}$ can be understood as a special case of the notation ``$R(X)$'', but where $R$ is not a set.
      \item $\es$ is a relation, and for all sets $X$ holds $\es(X) = \es$.
\end{enumerate}

\begin{defi}\label{def:binop}
      For any binary relation on a set $x$ we say:
	  \begin{itemize}
	        \item $R$ is \textbf{reflective} if $ (\forall a \in x) \ ( aRb)$;
			\item $R$ is \textbf{symmetric} if $ (\forall a,b \in x) \ ( aRb \rightarrow bRa)$;
			\item $R$ is \textbf{antisymmetric} if $ (\forall a,b \in x) [ ( aRb \wedge a \not = b)\rightarrow \neg (bRa)]$;
			\item $R$ is \textbf{connected} if $ (\forall a,b \in x) [(a \not = b) \rightarrow (aRb \vee bRa)]$;
			\item $R$ is \textbf{transisive} if $ (\forall a,b,c \in x) [ ( aRb \wedge bRc) \rightarrow (aRc)]$ \cite{h1}.
	  \end{itemize}
\end{defi}

A binary relation on a set is said to be an \textbf{equivalence relation} just in the case it is reflexive, symmetric and transisive \cite{h1}.

\begin{defi}\label{def:map}
      A \textbf{map} is a relation $f$ such that for all $x \in \proj_1(f)$ there is exactly one $y \in \proj_2(F)$ with $(x,y) \in f$, where then 
	  $y =: \bmm{f(x)}$ is used. Furthermore $\bmm{\dom(f)} := \proj_1(f)$ (``domain'') and $\bmm{\rg(f)} := \proj_2(f)$ (``range'').
\end{defi}
Remarks:
\begin{enumerate}
      \item Normally there is no confusion between $f(X) = \set{f(x) : x \in X \cap \dom(f)}$ for a set $X$ according to Definition \ref{def:reli}, 
	  and $f(x) = y$ for a \emph{single} argument $x \in \dom(f)$.
\end{enumerate}
%--------------------------------------------------------------------------------------------------------------------------------------

\subsection{Special Maps}
\label{sec:Specialmaps}

\begin{defi}\label{def:inj}
      A map $f$ is called an \textbf{injection} (is ``injective'' or ``one-to-one'') if for all $x, x' \in \dom(f)$ holds $f(x) = f(x') \Ra x = x'$.
\end{defi}
\begin{defi}\label{def:surj}
      For a map $f$ holds the statement \bmm{f:X \ra Y} if $X, Y$ are sets with $X = \dom(f)$ and $\rg(f) \sse Y$. Such a statement attaches to 
	  \textbf{codomian} (or ``target set'') $Y$ to the map $f$. Given such a codomain $Y$, $f$ is called \textbf{surjective} or ``onto'' if $\rg(F) = Y$.
\end{defi}
\begin{defi}\label{def:bij}
      A map $f$ with codomain $Y$ is called \textbf{bijective} (``is a bijection'') if $f$ is injective and surjective.
\end{defi}
Remarks:
\begin{enumerate}
      \item If sets $A , B$ are injective and infinite, then we have $\abs{A} \leq \abs{B}$
\end{enumerate}
\begin{examp}\label{exp:Specialmaps}
      For map $f: \NN \to \NN$ and $f=2x$, the codomain is $\NN$ but the range is even numbers.
\end{examp}
\begin{examp}\label{exp:Specialmaps}
      The map $F(x) = 3 x + 7$ is injective but the map $G(x) = x^4 - x$ is not. Fig \ref{fig:injct} showes another examples.
\end{examp}

\begin{figure}
      \begin{center}
	  %\centering
      \includegraphics[scale =0.8]{p11.png}
      \caption{Examples of injective and non-injective maps.}
	  \label{fig:injct}
\end{center}
	
\end{figure}
\begin{examp}\label{exp:Specialmaps}
      The map $F(x) = 2 x $ from natural numbers to the set of non-negative even numbers is surjective but from the set of natural numbers 
	  to natural numbers is non-surjective.
\end{examp}
\begin{examp}\label{exp:Specialmaps}
      The map $F(x) = x^2$ from the set of positive numbers to positive real numbers is injective and surjective. Therefore, it is bijective.
\end{examp}
%\begin{figure}
       %\begin{center}
	   %\centering
       %\includegraphics{p3.png}
       %\caption{Example of different typs of map.}
       %\end{center}
%\end{figure}
%--------------------------------------------------------------------------------------------------------------------------------------
\subsection{Decompositions and Equivalence Relations}
\label{sec:Decompositions}

\begin{defi}\label{def:Decompositions}
      A \textbf{partition} of a set $X$ is some $P \sse \pot(X) \sm \set{\es}$ wich is disjoint, i.e., $\fa\, A, B \in P : A \ne B \Ra A \cap B = \es$, 
	  and where $\bc P = X$.
\end{defi}
\begin{examp}\label{exp:Decompositions}
      The set $X = \{ 1, 2, 3 \} $ has these five partitions:
      $$ \{ \{1\}, \{2\}, \{3\}\}$$
      $$ \{ \{1,2\}, \{3\}\}$$
      $$ \{ \{1, 3\}, \{2\}\}$$
      $$ \{ \{1\}, \{2, 3\}\}$$
      $$ \{ \{ 1, 2, 3 \}\}$$
      The following are not partitions of X.
      $$ \{ \{\}, \{2, 3\}\}$$
      $$ \{ \{1, 2\}, \{2, 3\}\}$$
      $$ \{ \{1\}, \{2\}\}$$
\end{examp}
%--------------------------------------------------------------------------------------------------------------------------------------
\subsection{Finite Sets}
\label{sec:Finite sets}

\begin{defi}\label{def:finite}
      A set $X$ is called \textbf{infinite} if there is a bijection $f: X \ra X'$ for some $X' \subset X$, otherwise $X$ is called \textbf{finite}.
\end{defi}
\begin{defi}\label{def:finitesubs}
      For a set $X$ let $\bmm{\pote(X)} := \set{S \in \pot(X) : S \text{ finite}}$ be the set of finite subsets of $X$.
\end{defi}
\begin{lem}\label{lem:finitesubs}
      For a set $X$ holds $\pot(X) = \pote(X)$ iff $X$ is finite.
\end{lem}
%--------------------------------------------------------------------------------------------------------------------------------------
\subsection{Natural Numbers}
\label{sec:natnumbers}

\begin{defi}\label{def:natnumberssimple}
      We can define the first \textbf{natural numbers} as $\bmm{0} := \es$, $\bmm{1} := \set{0}$, $\bmm{2} := \set{0,1}$, $\bmm{3} := \set{0,1,2}$.
\end{defi}
	  \begin{defi}\label{def:successor}
      In general, we can define the \textbf{successor} of a set $x$ as $\bmm{x'} := \set{x} \cup x$. So $1 = 0'$, $2 = 1'$, $3 = 2'$, and so on.
\end{defi}
Remarks:
\begin{enumerate}
      \item An important axiom of set theory is the \textbf{axiom of infinity}, which can be stated as the statement, that there is a set $X$ with
	  $\es \in X$ and $\fa\, x \in X : x' \in X$.
	  \item A nonempty subset $S$ of $\ZZ$ is  \textbf{well-ordered} if $S$ contains a least element. The set $\ZZ$ is not well-ordered since it does not 
	  contain a smallest element. However, the natural numbers are well-ordered.
	  \item \textbf{Principle of Well-Ordering}: Every nonempty subset of the natural numbers is well-ordered.
\end{enumerate}

\begin{lem}\label{lem:omega}
      There exists a smallest set $\omega$ with $\es \in \omega$ and $\fa\, x \in \omega : x \in \omega \Ra x' \in \omega$, that is, $\omega$ is 
	  the unique set with these two properties and the condition, that for every set $X$ with these two conditions we have $\omega \sse X$.
\end{lem}
Remarks:
\begin{enumerate}
      \item We use $\omega$ for the set of natural numbers including zero, if we wish to use the concrete representation of natural numbers, otherwise 
	  we use \bmm{\NNZ}, and $\bmm{\NN} := \NNZ \sm \set{0}$.

      So we can define $\NNZ := \omega$, but when using this notation, then we do not make use of the internal structure of natural numbers 
	  so we might consider them as urelements).
\end{enumerate}
%--------------------------------------------------------------------------------------------------------------------------------------
\subsection{The Size of Sets}
\label{sec:sizeofsets}

\begin{defi}\label{def:equalsize}
      Two sets $X, Y$ \textbf{have the same size}, if there exists a bijection from $X$ to $y$.
\end{defi}
\begin{lem}\label{lem:sizefiniteset}
      A set $X$ is finite if and only if there is $n \in \omega$ such that $X$ has the same size as $n$, and where $n$ is uniquely determined.
\end{lem}
\begin{defi}\label{def:sizefiniteset}
      For a finite set $X$ by $\bmm{\abs{X}} \in \omega$ we denote the unique $n \in \omega$ with the same size as $X$ (according to Lemma 
	  \ref{lem:sizefiniteset}).
\end{defi}
Remarks:
\begin{enumerate}
      \item For the sets $A , B$, we have the following properties:
      \begin{enumerate}
            \item $\abs{A \times B }= \abs{A} \times \abs{B}$.
	        \item If $\abs{A} = n$ then $\abs{\pot(A)} = 2^n$.
	        \item $\abs{A \cup B} = \abs{A} + \abs{B} - \abs{A \cap B}$.
      \end{enumerate} 
\end{enumerate}
%--------------------------------------------------------------------------------------------------------------------------------------
\section{Brief Introduction to Abstract Algebra}
\label{sec:Abstract Algebra}
%--------------------------------------------------------------------------------------------------------------------------------------
%\subsection{Matrix}
%\label{sec:Matrix}

%--------------------------------------------------------------------------------------------------------------------------------------
\subsection{Group}
\label{sec:Group}

\begin{defi}\label{def:group1}
      A \textbf{group} is a pair $(G,\circ)$ where $G$ is a set and $\circ$ is a binary operation on $G$, such that the following four properties hold
	  \begin{enumerate}
	        \item (\textbf{closure}) for all $a, b \in G, a \circ b \in G$;
			\item (\textbf{associativity}) for all $a, b, c \in G, a \circ (b \circ c) = (a \circ b) \circ c$;
			\item (\textbf{existence of the identity element}) there is an element $e \in G$ such that for all $a \in G, a \circ e = e \circ a = a$;
			\item (\textbf{existence of inverses}) for every $a \in G$, there is an element $b \in G$ (called the inverse of $a$) such that
            $a \circ b = b \circ a = e$.
	  \end{enumerate}
\end{defi}

\begin{examp}\label{exp:group1}
      $(\RR,+)$ is a group. We know already that addition is a binary operation on $\RR$, so ‘closure’ holds. We know addition of real numbers
       is associative. We want an element $e \in \RR$ so that $a + e = e + a = a$ for all $a \in \RR$. It is clear that $e = 0$ works and 
	   is the only possible choice. Moreover, the (additive) inverse of $a$ is $-a : a +(-a) = (-a) + a = 0$.
\end{examp}

\begin{examp}\label{exp:group2}
      $(\RR^2,+)$ is a group. Since the  elements of $\RR^2$ are pairs $(a_1,a_2)$ where $a_1, a_2$ are real numbers. Addition is defined
      by 
      $$(a_1,a_2)+(b_1,b_2) = (a_1+b_1,a_2+b_2).$$
      Note that the entries $a_1 + b_1$ and $a_2 + b_2$ are real numbers, and so $(a_1 + b_1,a_2 +b_2)$ is a pair of real numbers. Hence 
	  $(a_1 + b_2,a_2 + b_2)$ is in $\RR^2$ (closure). It can be proved that other laws (associativity, existence of the identity element 
	  and exictence of inverse) are satisfied.
\end{examp}

\begin{defi}\label{def:group2}
      A group $(G,\circ)$ is called \textbf{abelian} if it satisfies (\textbf{commutativity}) for all $a, b \in G, a \circ b = b \circ a$.
\end{defi}

\begin{examp}\label{exp:group3}
      These are some examples of abelian groups: 
	  $$(\RR,+), (\CC,+), (\QQ,+), (\RR^n,+).$$
\end{examp}
\begin{defi}\label{def:group3}
      Let $(G,\circ)$ be a group. Let $H$ be a subset of $G$ and suppose that $(H,\circ)$ is also a group. Then we say that $H$ is a 
	  \textbf{subgroup} of $G$ (or more formally $(H,\circ)$ is a subgroup of $(G,\circ)$). 
\end{defi}	  
For $H$ to be a subgroup of $G$, we want $H$ to a group with respect to the same binary operation that makes $G$ a group.

\begin{examp}\label{exp:group4}
      $\ZZ$ is a subgroup of $\RR$ (or more formally, $(\ZZ,+)$ is a subgroup of $(\RR,+)$); because Z is a subset of $R$ and both are
	  groups with respect to the same binary operation which is addition.
\end{examp}

\begin{examp}\label{exp:group5}
      The set $V = \{(a,a) : a \in \RR \}$ is the line $y = x$. It contains the identity element $(0, 0)$, is closed under addition
      and negation. Therefore it is a subgroup of $\RR^2$.
\end{examp}

\begin{examp}\label{exp:group6}
      The ray $W = \{(a,a) : a \in \RR, a \geq 0 \}$ is not a subgroup of $\RR^2$. It contains the identity element $(0,0)$ and
      is closed under addition. The problem is with the existence of additive inverses; e.g. $(1,1)$ is in $W$ but its inverse
      $(-1,-1)$ is not in $W$.
\end{examp}

\begin{lem}\label{lem:sizefiniteset}
      Let $G$ be a group. A subset $H$ of $G$ is a subgroup if and only if it satisfies the following three conditions:
      \begin{enumerate}
	        \item $1 \in H$,
		 	\item if $a, b \in H$ then $ab \in G$,
		 	\item if $a \in H$ then $a^-^1 \in H$.
	   \end{enumerate}
\end{lem}
	 
\begin{defi}\label{def:group4}
      The \textbf{order} of an element $a$ in a group $G$ is the smallest positive integer $n$ such that $a^n =1$. If there is no 
	  such positive integer $n$, we say a has \textbf{infinite order}.
\end{defi}
%--------------------------------------------------------------------------------------------------------------------------------------
\subsection{Ring}
\label{sec:Ring}

\begin{defi}\label{def:ring1}
      A \textbf{ring} is a triple $(R,+, \cdot)$, where $R$ is a set and $+, \cdot$ are binary operations on $R$ such that the following properties hold:
      \begin{enumerate}
	         \item (closure) for all $a, b \in R, a + b \in R$ and $a \cdot b \in R$;
			 \item (associativity of addition) for all $a, b, c \in R, (a+b) +c = a +(b+c)$;
			 \item (existence of an additive identity element) there is an element $0 \in R$ such that for all $a \in R, a + 0 = 0 + a = a$.
			 \item (existence of additive inverses) for all $a \in R$, there an element, denoted by $-a$, such that $a +(-a) = (-a)+ a = 0$;
			 \item (commutativity of addition) for all $a, b \in R, a + b = b +a$;
			 \item (associativity of multiplication) for all $a, b, c \in R, a \cdot (b \cdot c) = (a \cdot b) \cdot c$;
			 \item (distributivity) for all $a, b, c \in R, a \cdot (b + c) = a \cdotb + a \cdot c; \ (b + c) \cdot a = b \cdot a + c \cdot a$;
			 \item (existence of a multiplicative identity) there is an element $1 \in R$ such that $1 \not = 0$ and for all $a \in R, 1 \cdot a = a \cdot 1 = a$.
	  \end{enumerate}
\end{defi}

Moreover, a ring $(R,+, \cdot)$ is said to be \textbf{commutative}, if it satisfies the following additional property (commutativity of multiplication):
$$ \forall a, b \in R, a \cdot b = b \cdot a.$$

\begin{examp}\label{exp:ring1}
      We know lots of examples of rings: $\ZZ, \QQ, \RR, \CC$,  etc.
      All these examples are also commutative rings.
\end{examp}

\begin{defi}\label{def:ring2}
      Let $(R,+, \cdot)$ be a ring. Let $S$ be a subset of $R$ and suppose that $(S,+, \cdot)$ is also a ring. Then, we say that $S$ is a \textbf{subring} of $R$ 
	  (or more formally $(S,+, \cdot)$ is a subring of $(R,+, \cdot)$).
\end{defi}
For $S$ to be a subring of $R$, we want $S$ to a ring with respect to the same two binary operations that makes R a ring.

\begin{examp}\label{exp:ring2}
            $\ZZ$ is a subring of $\RR$; $\QQ$ is a subring of $\RR$; $\ZZ$ is a subring of $\QQ$;
\end{examp}

\begin{defi}\label{def:ring3}
      Let $R$ be a ring. A subset $S$ of $R$ is a subring if and only if it satisfies the following conditions:
      \begin{enumerate}
	         \item $0, 1 \in S$ (that is $S$ contains the additive and multiplicative identity elements of $R$);
			 \item if $a, b \in S$ then $a + b \in S$;
			 \item if $a \in S$ then $-a \in S$;
			 \item if $a, b \in S$ then $ab \in S$. 
	  \end{enumerate}
\end{defi}

Remarks:
\begin{enumerate}
      \item The easiest way to show that a set is a ring is to show that it is subring of a known ring. If we do this, 
      we only have four properties to check (1),(2),(3),(4).
\end{enumerate}
\begin{defi}\label{def:rrr}
      Let $R$ be a ring. An element $u$ is called a \textbf{unit} if there is some element $v \in R$ such that $uv = vu = 1$. In other words, 
	  an element $u$ of $R$ is a unit if it has a multiplicative inverse that belongs to $R$.
\end{defi}

\begin{examp}\label{exp:ring3}
      In any ring, $0$ is a non-unit.     
\end{examp}
\begin{examp}\label{exp:ring4}
      In $\RR$, $\QQ$, $\CC$, every non-zero element has a multiplicative inverse. So the units are the non-zero elements.     
\end{examp}
\begin{examp}\label{exp:ring5}
      The only integers $u$ such that $1/u$ is also an integer are $\pm 1$. So the units in $\ZZ$ are $\pm 1$.    
\end{examp}

\begin{defi}\label{def:ring4}
      Let $R$ be a ring. We define the \textbf{unit group} of $R$ to be the set
	  $R^* = \{a \in R : a$ is  a unit in $R \}.$
\end{defi}
\begin{lem}\label{lem::ring5}
      Let $(R,+, \circ)$ be a ring and let $R^*$ be the subset defined in \ref{def:ring4}. Then $(R^*, \circ)$ is a group.
\end{lem}

\begin{examp}\label{exp:ring6}
      The units of $\ZZ$ are $\pm 1$. Therefore the unit group of $\ZZ$ is $\ZZ ^* = \{1,-1 \}$.  
\end{examp}
%--------------------------------------------------------------------------------------------------------------------------------------
\subsection{Field}
\label{sec:Field}

\begin{defi}\label{def:field}
      A \textbf{field} $(F,+,\cdot)$ is a commutative ring such that every non-zero element is a unit. Thus a commutative ring $F$ is a field 
	  if and only if its unit group is
	  $$F^*= \{a \in F : a \not = 0 \}.$$
\end{defi}
\begin{examp}\label{exp:field1}
      $\RR, \CC, \QQ$ are fields.
\end{examp} 
\begin{examp}\label{exp:field2}
      $\ZZ$ is not a field, since for example $2 \in \ZZ$ is non-zero but not a unit.
\end{examp}

%--------------------------------------------------------------------------------------------------------------------------------------
\subsection{Vector Space}
\label{sec:Vector Space}

A vector space is a set $V$ with two operations: addition of vectors and scalar multiplication. These operations satisfy certain properties. 
The scalars are taken from a field $F$, where for the remainder of these notes $F$ stands either for the real numbers $\RR$ or the complex 
numbers $\CC$. The real and complex numbers are examples of fields. Vector addition can be thought of as a map $+ : V \times V \rightarrow V $, 
mapping two vectors $u, v \in V$ to their sum $u+ v \in V$ . Scalar multiplication can be described as a map $F \times V \rightarrow V$ , which 
assigns to a scalar $a \in F$ and a vector $v \in V $a new vector $av$.

\begin{defi}\label{def:vcs}
      A \textbf{vector space} over $F$ is a set $V$ together with the operations of addition $V \times V \rightarrow V $ and scalar multiplication 
	  $F \times V \rightarrow V$ satisfying the following properties:
	  \begin{enumerate}
	        \item Commutativity: $u + v = v + u$ for all $u, v \in V$ ;
			\item Associativity: $(u + v) + w = u + (v + w)$ and $(ab)v = a(bv)$ for all $u, v,w \in V$ and $a, b \in F$;
			\item Additive identity: There exists an element $0 \in V$ such that $0 + v = v$ for all $v \in V$ ;
			\item Additive inverse: For every $v \in V$, there exists an element $w \in V$ such that $v+w = 0$;
			\item Multiplicative identity: $1v = v$ for all $v \in V$ ;
			\item Distributivity: $a(u + v) = au + av$ and $(a + b)u = au + bu$ for all $u, v \in V$ and $a, b \in F$.
      \end{enumerate}
\end{defi}
Usually, a vector space over $\RR$ is called a \textbf{real vector space} and a vector space over $\CC$ is called a \textbf{complex vector space}. 
The elemens $v \in V$ of a vector space are called \textbf{vectors}.

Remarks:
\begin{enumerate}
	  \item Simple properties of vector spaces:
	        \begin{enumerate}
			      \item Every vector space has a unique additive identity.
				  \item Every $v \in V$ has a unique additive inverse
				  \item $0v = 0$ for all $v \in V$.
				  \item $a0 = 0$ for every $a \in F$.
				  \item $(−1)v = −v$ for every $v \in V$
		    \end{enumerate}
\end{enumerate}

\begin{examp}\label{exp:vs12}
      Consider the set $F^n$ of all $n$-tuples with elements in $F$. This is a vector space. Addition and scalar multiplication are
	  defined componentwise. That is, for $u = (u_1, u_2, ... , u_n)$, $v = (v_1, v_2, . . . , v_n) \in F^n$ and $a \in F$, we define
	  $$u + v = (u_1 + v_1, u_2 + v_2, . . . , u_n + v_n),$$
	  $$au = (au_1, au_2, . . . , au_n).$$      
\end{examp}

\begin{examp}\label{exp:vs2}
      Let $\beta {(F)}$ be the set of all polynomials $p : F \rightarrow F$ with coefficients in $F$. More precisely, $p(z)$ is a polynomial if 
	  there exist $a_0, a_1, . . . , a_n \in F$ such that
      $$p(z) = a_n z^n + a_{n-1} z^{n-1} + ... + a_1z + a_0.$$
      Addition and scalar multiplication are defined as:
      $$(p + q)(z) = p(z) + q(z),$$
      $$(ap)(z) = ap(z),$$
      where $p, q \in \beta (F)$ and $a \in F$. For example, if $p(z) = 5z + 1$ and $q(z) = 2z^2 + z + 1$, then $(p + q)(z) = 2z^2 + 6z + 2$ 
	  and $(2p)(z) = 10z + 2$. 
	  
      Again, it can be easily verified that $\beta(F)$ forms a vector space over $F$. The additive identity in this 
	  case is the zero polynomial, for which all coefficients are equal to zero. The additive inverse of $p(z)= a_n z^n + a_{n-1} z^{n-1} + ... + a_1z + a_0$  
	  is $-p(z) = -a_n z^n - a_{n-1}z^{n-1} - ... - a_1 z - a_0.
\end{examp}

\begin{defi}\label{def:vcs_2}	
      A subset $U \subset V$ of a vector space $V$ over $F$ is a \textbf{subspace} of $V$ if $U$ itself is a vector space over $F$.
\end{defi}	

To check that a subset $U \subset V$ is a subspace, it suffices to check only a couple of the conditions of a vector space.		  
\begin{lem}\label{lem::vcs}
      Let $U \subset V$ be a subset of a vector space $V$ over $F$. Then $U$ is a subspace of $V$ if and only if:
	  \begin{enumerate}
	       \item additive identity: $0 \in U$;
		   \item closure under addition: $u, v \in U$ implies $u + v \in U$;
		   \item closure under scalar multiplication: $a \in F$, $u \in U$ implies that $au \in U$.
	  \end{enumerate}
\end{lem}

\begin{examp}\label{exp:subv1}
      In every vector space $V$ , the subsets $\{ 0 \}$ and $V$ are trivial subspaces.      
\end{examp}
\begin{examp}\label{exp:subv2}
      $\{ (x_1, 0) \mid x_1 \in \RR \}$ is a subspace of $\RR^ 2$.      
\end{examp}
\begin{examp}\label{exp:subv3}
      $U = \{ (x_1, x_2, x_3) \in F^3 \mid x_1 + 2x_2 = 0 \}$ is a subspace of $F^3$.      
\end{examp}

\begin{defi}\label{def:vcs_3}
      Let $U_1, U_2 \subset V$ be subspaces of $V$. We define the \textbf{sum} of $U_1$ and $U_2$ as:
      $$ U_1 + U_2 = \{u_1 + u_2 \mid u_1 \in U_1, u_2 \in U_2 \}.$$
	  In fact, $U_1 + U_2$ is the smallest subspace of $V$ that contains both $U_1$ and $U_2$.
\end{defi}

\begin{examp}\label{exp:subv4}
      Let 
      $$U_1 = \{(x, 0, 0) \in F^3 \mid x \in F \},$$
      $$U_2 = \{(0, y, 0) \in F^3 \mid y \in F \}.$$
      Then
      $$U_1 + U_2 = {(x, y, 0) \in F^3 \mid x, y \in F}. (2)$$
      If alternatively $U2 = \{(y, y, 0) \in F^3 \mid y \in F \}$ then the above sum still holds.    
\end{examp}

\begin{defi}\label{def:vcs_4}
      Suppose every $u \in U$ can be uniquely written as $u = u_1 +u_2$ for $u_1 \in U_1$ and $u_2 \in U_2$. 
	  Then
      $$U = U_1 \oplus U_2$$
      is the \textbf{direct sum} of $U_1$ and $U_2$.
\end{defi}
\begin{examp}\label{exp:subv5}
      Let
      $$U_1 = \{(x, y, 0) \in R^3 \mid x, y \in \RR \},$$
      $$U_2 = \{(0, 0, z) \in R^3 \mid z \in \RR \}.$$
      Then $R^3 = U_1 \oplus U_2$. However, if instead
      $$U_2 = \{(0,w, z) \mid w, z \in \RR \},$$
      then $R^3 = U_1 + U_2$, but it is not the direct sum of $U_1$ and $U_2$.
\end{examp}

%***************************************************************************************************************************************************************
%***************************************************************************************************************************************************************
\chapter{From Variables to Clause-sets}
\label{cha:vartocls}

\section{Variables}
\label{sec:Variables}

\begin{defi}\label{def:var}
      The set of ``variables'' is denoted by \bmm{\Va}. For every variable $v \in \Va$ its \textbf{domain} is a finite and non-empty set, denoted by 
	  $\bmm{D_v} \ne \es$. Together $(\Va, (D_v)_{v \in \Va})$ is the \textbf{variable-frame}.
      \begin{itemize}
            \item A \textbf{standard domain} is of the form $D_v = \tb 0m \subset \NNZ$ for some $m \in \NNZ$.
            \item A \textbf{boolean variable} $v$ has domain $D_v = \set{0,1}$.
            \item The domain-size of variables is denoted by $\bmm{\dos}: \Va \ra \NN$, with $\dos(v) := \abs{D_v}$.
      \end{itemize}
      It is assumed that for every occurring domain-size $m \in \dos(\Va)$ there are infinitely many variables with this domain-size, that is, $\dos^{-1}(m)$ is infinite.
      \begin{itemize}
            \item The \textbf{standard boolean var-set} is $\Va := \NN$, where all variables are boolean.
            \item The \textbf{standard non-boolean var-set} is $\Va := \NN \times \NN_{\ge2}$, where all variables have a standardised domain, and where 
			$\dos((n,m)) = m$ for $(n,m) \in \Va$.
      \end{itemize}
\end{defi}
Remarks:
\begin{enumerate}
      \item For variables $v$ with standard domains holds $D_v = \tb{0}{\dos(v)-1}$.
      \item It seems not useful to allow domain-size zero; such variables couldn't be assigned at all.
      \item Allowing infinite domains would yield a very different situation; for example \href{https://en.wikipedia.org/wiki/Compact_space}{compactness} would fail.
      \item Typically we just mention $\Va$ in the basic set-up, not the full variable-frame $(\Va,
      D_v)_{v \in \Va})$, which is understood implicitly. Then one says whether we have only boolean variables or also non-boolean variables.
\end{enumerate}

\begin{examp}\label{exp:var}
      The standard boolean variables are $1, 2, 3, \dots \in \NN$, the standardd non-boolean boolean variables(!) are $(1,2), (2,2), (3,2), \dots \in \NN \times \set{2}$. 
	  The standard ternary variables (three-valued) are $(1,3), (2,3), (3,3), \dots \in \NN \times \set{3}$.
\end{examp}
%--------------------------------------------------------------------------------------------------------------------------------------
\section{Partial Assignments}
\label{sec:Partialassignments}

\begin{defi}\label{def:Pass}
      A \textbf{partial assignment} is a map $\vp$ with domain $\bmm{\var(\vp)} := \dom(\vp) \in \pote(\Va)$, such that for all $v \in \var(\vp)$ holds $\vp(v) \in D_v$. 
	  The set of all partial assignments is denoted by \bmm{\Pass}.

      For $V \in \pote(\Va)$ let $\bmm{\Tass(V)} := \set{\vp \in \Pass : \var(\vp) = V}$ be the set of \textbf{total assignments} over $V$.

      A special partial assignment is $\bmm{\epa} := \es \in \Pass$ (the \textbf{empty partial assignment}).
\end{defi}
Remarks:
\begin{enumerate}
      \item For variables $v_1, \ldots, v_n \in \mva, \; n\in \mathbb{N}_0$ with $v_i \neq v_j$ for $i\neq j$, and truth values $\varepsilon_1, \ldots, \varepsilon_n \in \{0,1\}$ we write
      \begin{displaymath}
            \pmb{\langle v_1 \to \varepsilon_1, \ldots, v_n \to \varepsilon_n\rangle}
      \end{displaymath}
      for the partial assignment with
      \begin{displaymath}
            \mbox{domain} \; \{v_1, \ldots, v_n\}
      \end{displaymath}
      which maps $v_i \mapsto \varepsilon_i$.
      \item So $\Tass(V) = \prod_{v \in V} D_v$.
      \item $n(\varphi) := \mid \var(\varphi) \mid$ (the number of variables in a partial assignment).
\end{enumerate}

%--------------------------------------------------------------------------------------------------------------------------------------
\section{Literals}
\label{sec:Litsvar}

\begin{defi}\label{def:litdervar}
      For a variable-frame $(\Va, (D_v)_{v \in \Va})$, a \textbf{literal} is a pair $(v,\ve)$ with $v \in \Va$ and $\ve \in D_v$. The set of all literals is denoted by 
	  $\bmm{\Lit}$. In other words, the set $\Lit$ of \textbf{literals} (over $\Va$) is defined as:
      $$\Lit( \Va): = \Va \times \{0,1\}.$$
      For $(v,\ve) \in \Lit$:
      \begin{enumerate}
            \item $\bmm{\var((v,\ve))} := \proj_1((v,\ve)) = v \in \Va$.
            \item $\bmm{\val((v,\ve))} := \proj_2((v,\ve)) = \ve$.
            \item \textbf{positive} in case of $\varepsilon = 0$.
            \item \textbf{negative} in case of $\varepsilon = 1$.
      \end{enumerate}
      Literals $x, y \in \Lit$ \textbf{clash} (or ``have a conflict'') if $\var(x) = \var(y)$ and $\val(x) \ne \val(y)$. For $L \sse \Lit$ we say that 
	  \textbf{$L$ is clash-free} if there are no $x, y \in L$ which clash.
\end{defi}
Remarks:
\begin{enumerate}
      \item As customary, in case we are not especially interested in the underlying set   $\Va$ of variables, we will just use  $\Lit$ instead of $\Lit$($\Va$)
      \item The set of partial assignments is the set of clash-free $\vp \in \pote(\Lit)$ (recall that by Subsection \ref{sec:maps} a map $f$ is the set of 
	  pairs $(x,f(x))$ for $x \in \dom(f)$).
      \item Simple properties:
      \begin{enumerate}
            \item $({}^{\overline{~~}}): \Lit \to  \Lit.$
            \item $\overline{(v, \varepsilon)}: = (v, 1-\varepsilon).$
            \item For a literal $x$ we call $\overline{{\bm x}}$ the \textbf{complement} of $x$ and we have
            $$\forall x \in \mlit: \overline{\overline{x}} = x.$$
      \end{enumerate}
\end{enumerate}
\begin{examp}\label{}
      Consider a variables $a\in  \Va$. Using the identification of variables and positive literals, we have
      \begin{eqnarray*}
            &a = (a,0)& \\
            &\overline{a} = \overline{(a,0)} = (a,1)&\\
            &\overline{\overline{a}} = \overline{(a,1)} = (a,0) = a&\\
            &\var(a) = \var((a,0)) = a& \\
            &\var(\overline{a}) = \var((a,1)) = a. &
      \end{eqnarray*}
\end{examp}
%--------------------------------------------------------------------------------------------------------------------------------------
\section{Implementing Variables and Literals}
\label{sec:varlit}
For concrete implementations it is often efficient to identify
$$\Va = \mathbb{N},$$
$$\Lit = \mathbb{Z}\setminus \{0\},$$ 
and to use the identification
$$\overline{x} = -x$$
for $x \in \Lit$.

In C thus it is a possible choice to define the types Var as well as Lit as the basic type int (do not use unsigned int, since in this 
way you get into unnecessary trouble with negation). It is an easy implementation, however not type-safe.
To associate information with variables and literals, the most natural way for this representation is to use variables and literals as 
\textit{indices}.
%--------------------------------------------------------------------------------------------------------------------------------------
\section{Clauses}
\label{sec:Clauses}

\begin{defi}\label{def:clauses}
      For a set $L \sse \Lit$ we define:
      \begin{itemize}
            \item $\bmm{\var(L)} := \set{\var(x) : x \in L}$
            \item $\bmm{\lit(L)} := \set{x \in \Lit : \var(x) \in \var(L)}$
            \item $\bmm{\ol{L}} := \lit(L) \sm L$.
      \end{itemize}
\end{defi}
Remarks:
\begin{enumerate}
      \item So $\var(L)$ is the set of ``variables of $L$'', $\lit(L)$ is the set of ``literals having a variable in $L$'', while $\ol{L}$ 
	  is the set-complement of $L$ in the set of literals with variables in $L$.
      \item For $\vp \in \Pass$ holds $\ol{\vp} \in \Pass$ iff $\dos(\vp) \sse \set{1,2}$.
      \item $\set{\lit(\set{x})}_{x \in \Lit}$ is a partition of $\Lit$, and the partial assignments are the finite sets of literals which 
	  intersect with each element of this set-system in at most one element.
      \item For a finite $L \subset \Lit$ the following properties are equivalent:
      \begin{enumerate}
            \item $L$ is clashfree.
            \item $\fa\, x \in \Lit : \abs{\lit(\set{x}) \cap L} \le 1$.
            \item $\abs{L} = \abs{\var(L)}$.
      \end{enumerate}
      \item If $L$ is a set of boolean variables, then $\lit(L)} = L \cup \ol{L}$.
      \item If $L$ is a set of non-boolean variables, then $\ol{L} = \emptyset$.
\end{enumerate}
\begin{examp}\label{exp:lit1}
      For sets $L_1, L_2 \sse \Lit$ with boolean variables, if $L_1 = \{1, 2, 3 \}$ and $L_2 = \{1, -1, -2, 3 \}$ we have:
      $$lit(L_1) = \{1, -1, 2, -2, 3, -3\}$$
      $$lit(L_2) = \{1, -1, 2, -2, 3, -3\}$$
\end{examp}
\begin{examp}\label{exp:lit1}
      For sets $L_1, L_2 \sse \Lit$ with non-boolean variables, if $L_1 = \{1, 2, 3 \}$, and $L_2 = \{1, -1, -2, 3 \}$ we have:
      $$lit(L_1) = \{1, 2, 3\}$$
      $$lit(L_2) = \{1, -1, -2, 3\}$$
\end{examp}
\begin{defi}\label{def:cl}
      A \textbf{clause} is a finite and clash-free set of literals, the set of all clauses is denoted by
      $$\bmm{\Cl} := \set{C \in \pote(\Lit) : \abs{C} = \abs{\var(C)}}.$$
      A special clause is $\bmm{\bot} := \es \in \Cl$, the \textbf{empty clause}.
\end{defi}
\begin{lem}\label{lem::CLPASS}
      $\Cl = \Pass$.
\end{lem}
Remarks:
\begin{enumerate}
      \item Since a clause is a \textit{set}, every element occurs only once in a clause;
      \item There is no order on the elements of a clause.
      \item So to partial assignments $\vp \in \Pass$ we can apply the operations which can be applied to sets of literals. Especially
      $$\ol{\vp} = \set{(v,\ve) \in \Lit : v \in \var(\vp) \wedge \vp(v) \ne \ve}.$$
\end{enumerate}
%--------------------------------------------------------------------------------------------------------------------------------------
\section{Clause-sets}
\label{sec:cls}

\begin{defi}\label{def:cls}
      A \textbf{clause-set} is a finite set of clauses, the set of all clause-sets is denoted by $\bmm{\Cls} := \pote(\Cl)$.

      A special clause-set is $\bmm{\top} := \es \in \Cls$, the \textbf{empty clause-set}.
\end{defi}
\begin{defi}\label{def:cls2}
      The clause-set $F \in \Cls$ where each clause contains at most $k$ literals (has width at most $k$) is called \textbf{k-}$\Cls$.
      This means that $\mid C \mid \leq k$ \cite{h5}.
\end{defi}
\begin{defi}\label{def:cls3}
      A literal $x$ is called \textbf{pure} for $F$ if \ $\overline{x} \not \in \bigcup F$ 
      (the literals occuring in $F$ are give by $\bigcup F \subset \Lit$) \cite{h5}.
\end{defi}
\begin{defi}\label{def:cls4}
      For $F \in \Cls$:
      \begin{enumerate}
            \item $\bmm{\var(F)} := \bc_{C \in F} \var(C) \in \pote(\Va)$.
            \item $\bmm{\lit(F)} := \lit(\var(F)) \in \pote(\Lit)$ or $\lit(F) := \var(F) \cup \overline{\var(F)}$.
            \item $\bmm{n(F)} := \abs{\var(F)} \in \NNZ$ (the number of variables).
            \item $\bmm{c(F)} := \abs{F} \in \NNZ$ (the number of clauses).
            \item $\bmm{\ell(F)} := \sum_{C \in F} \abs{C} \in \NNZ$ (the number of literal occurrences).
      \end{enumerate}
\end{defi}
\begin{defi}\label{def:cls6}
      Clause-sets $F,G$ are called \textbf{isomorphic}, if the variables of $F$ can be renamed and potentially flipped so that $F$ is 
	  turned into $G$. More precisely, an isomorphism $\alpha$ from $F$ to $G$ is a bijection $\alpha : \lit(F) \rightarrow \lit(G)$ 
	  which preservers complementation $(\alpha(\ol x) = \overline {\alpha(x)}$, and which maps the clauses of $F$ precisely to the 
	  clauses of $G$; when considering multi-clause-sets, then the isomorphism must preserve the multiplicity of clauses (that is, 
	  $G(\alpha(C)) = F(C)$ for all $C \in \Cl$) \cite{h9}.
\end{defi}
Remarks:
\begin{enumerate}
      \item Simple properties:
      \begin{enumerate}
            \item The logical laws of compositions $\wedge$ and $\vee$ are commutative, that is
            $$a\wedge b \leftrightarrow b\wedge a, \quad  a\vee b \leftrightarrow b\vee a.$$
            \item Repeating a proposition doesn't make any change, i.e.,
            $$a\wedge a \leftrightarrow a, \quad  a\vee a \leftrightarrow a.$$
      \end{enumerate}
      \item No clause may occur more than once in a clause-set.
      \item There is no order on the elements of a clause-set.
\end{enumerate}
\begin{examp}\label{exp:cls}
      Some examples of clause-sets are as folow:
      $$\left\{\{a\}, \{\overline{a}\}\right\}$$
      $$\left\{\{a,b\}, \{\overline{a},b\}, \{a, \overline{b}\}, \{\overline{a},\overline{b}\}\right\}$$
      $$\left\{\{a\}, \{\overline{a},b\}, \{\overline{a}, \overline{b}, c\}, \{\overline{a}, \overline{b}, \overline{c}\}\right\}$$
\end{examp}
\begin{examp}\label{exp:cls}
      For $F := \{\bot, \{1\}, \{−1, 2\}\}$ we have:
	  \begin{enumerate}
	        \item $\var(F) = \{1, 2\},\ \lit(F) = \{- 1, 1,- 2, 2\},\ \bigcup F = \{-1, 1, 2\}$.
			\item Literal 2 is pure for $F$ (the other literals in $\lit(F)$ are not pure).
			\item $n(F) = 2,\ c(F) = 3,\ \ell(F) = 3$.
			\item $\{-1, 2\}$ is a full clause of $F$, while the two other clauses are not full.
			\item $F$ has no full variable, while $F \setminus \{ \bot\}$ has the (single) full variable 1  \cite{h9}.
	  \end{enumerate}
\end{examp}
%--------------------------------------------------------------------------------------------------------------------------------------
%????????????
%--------------------------------------------------------------------------------------------------------------------------------------
\section{The Operation of Partial Assignments on Clause-sets}
\label{sec:oppasscls}

\begin{defi}\label{def:oppassCls}
      For $\vp \in \Pass$ and $F \in \Cls$:
      $$\bmm{\vp * F} := \set{C \sm \vp : C \in F \wedge C \cap \ol{\vp} = \es} \in \Cls.$$
\end{defi}
Remarks:
\begin{enumerate}
      \item Simple properties:
      \begin{enumerate}
            \item $\vp * (F \cup G) = \vp * F \cup \vp * G$.
            \item $\vp * \set{C} = \set{\bot}$ iff $C \sse \vp$.
            \item $\vp * \set{C} = \top$ iff $C \cap \ol{\vp} \ne \es$.
            \item $\vp = \set{x \in \Lit : \vp * \set{x} = \set{\bot}}$.
            \item $\vp * F = \top \Lra \fa\, C \in F : C \cap \ol{\vp} \ne \es$.
            \item $\bot \in \vp * F \Lra \ex\, C \in F : C \sse \vp$.
            \item $\bot \in F \Ra \bot \in \vp * F$.
            \item $\vp * \top = \top$.
            \item $ var(\vp * F) = F \Leftrightarrow var(\vp) \cap var(F) = \emptyset$.
            \item $ var(\vp * F) \subseteq var(F) \setminus var(\vp)$.
      \end{enumerate}
      \item If $\vp * F \in \set{\top, \set{\bot}}$ and $\vp' \supseteq \vp$, then $\vp' * F = \vp * F$.
      \item More generally, for $\vp' \supseteq \vp$ with $(\var(\vp') \sm \var(\vp)) \cap \var(\vp * F) = \es$ holds $\vp' * F = \vp * F$. 
	  Especially for $\var(\vp) \cap \var(F) = \es$ holds $\vp * F = F$.
\end{enumerate}

\begin{examp}\label{exp:op1}
      Some examples of operation of partial assignments are as folow:
      $$\varphi_{\{a,\overline{b},c\}} = \langle a, \overline{b}, c\to 0\rangle = \langle a\to 0, b\to 1, c\to 0 \rangle $$
      $$C_{\langle x\to 0, y\to 1, z\to 0 \rangle} = \{x, \overline{y}, z\}$$
      $$\langle a\to 1 \rangle * \left\{\{a,b,c\}, \{\overline{a}, \overline{c}\}, \{\overline{b}, \overline{c}\} \right\} = \left\{\{\overline{c}\}, \{\overline{b}, c\} \right\}$$
      $$\langle a\to & 0, b\to 1, c\to 0 \rangle * \left\{\{a,b,x\}, \{a,\overline{b},c\}, \{\overline{x}, \overline{y}\}, \{a,c,x\} \right\} = \left\{\bot, \{\overline{x}, \overline{y}\}, \{x\}\right\}$$
      (a falsifying assignment)
      $$\langle x\to & 1, y\to 0, z\to 1 \rangle * \left\{\{x,y,\overline{z}\}, \{x,\overline{y},z\}, \{\overline{x}, \overline{y}\}, \{x,y\} \right\} = \top$$
      (a satisfying assignment)
\end{examp}
%--------------------------------------------------------------------------------------------------------------------------------------
\section{Composition of Partial Assignments}
\label{sec:Compositionpass}

\begin{defi}\label{def:comppass}
      For $\vp, \psi \in \Pass$: $\bmm{\vp \circ \psi} := \psi \cup (\vp \sm \lit(\psi)) \in \Pass$.
      In other words: For the composition of two partial assignments take their ``union'' for non-conflicting variables, while in case of a 
	  conflict the right assignments ``wins''.
\end{defi}
\begin{lem}\label{lem:passmon}
      $(\Pass,\circ,\epa)$ is a monoid (an associative groupoid with identity element). It is generated by 
      the \textbf{elementary partial assignments} $\langle v\to \varepsilon\rangle$ for $v\in \Va$ and $\varepsilon \in \{0,1\}$, and the ``defining relations'' are
      $$\langle v\to \varepsilon \rangle \circ \langle v\to \varepsilon' \rangle = \langle v\to \varepsilon' \rangle $$
      $$\langle v\to \varepsilon \rangle \circ \langle w\to \varepsilon' \rangle = \langle w\to \varepsilon' \rangle \circ \langle v\to \varepsilon \rangle$$
      for $v, w\in \Va, v\neq w$, and $\varepsilon, \varepsilon' \in \{0, 1\}$.
\end{lem}

Remarks:
\begin{enumerate}
      \item Simple properties:
      \begin{enumerate}
            \item $\var(\varphi \circ \psi)  = \var(\varphi)\cup \var(\psi)$.
            \item $\varphi \circ \emptyset  = \emptyset \circ \varphi = \varphi$.
            \item $\varphi \circ (\psi \circ \vartheta) = (\varphi \circ \psi) \circ \vartheta$.
      \end{enumerate}
\end{enumerate}
\begin{lem}\label{lem:comcomp}
  For $\vp, \psi \in \Pass$ the following properties are equivalent:
  \begin{enumerate}
  \item $\vp \circ \psi = \psi \circ \vp$.
  \item $\vp \cup \psi \in \Pass$.
  \item $\vp, \psi$ do not clash, that is, $\vp \cap \ol{\psi} = \es$.
  \end{enumerate}
\end{lem}
\begin{examp}\label{exp:cmp}
      Consider variables $a,b,c,d \in \Va$. By the term $\langle a\to 0, b\to 1 \rangle$ the partial assignment $\varphi$ with domain 
	  $\var(\varphi) = \{a, b\}$ and $\varphi(a) = 0$ and $\varphi(b) = 1$ is denoted.
      Examples for the defining relations are:
      $$\langle a\to 0 \rangle \circ \langle a\to 1 \rangle & = \langle a\to 1 \rangle $$
      $$\langle b\to 1 \rangle \circ \langle b\to 1 \rangle & = \langle b\to 1 \rangle $$
      $$\langle c\to 1 \rangle \circ \langle c\to 0 \rangle & = \langle c\to 0 \rangle$$
      and
      $$ \langle a\to 0 \rangle \circ \langle d\to 1 \rangle = \langle d\to 1 \rangle \circ \langle a\to 0 \rangle$$
      $$\langle b\to 1 \rangle \circ \langle c\to 1 \rangle = \langle c\to 1 \rangle \circ \langle b\to 1 \rangle$$
\end{examp}
\begin{examp}\label{exp:cmp2}
      Every partial assignments is a composition of elementary partial assignments, for example:
      $$\langle a\to 0, b\to 1, c\to 0 \rangle = \langle a\to 0 \rangle \circ \langle b\to 1 \rangle \circ \langle c\to 0 \rangle. $$
      And an example for a composition of two non-elementary partial assignments is:
      $$\langle a\to 0, b\to 1, c\to 0 \rangle & \circ \langle a\to 1, b\to 1, d\to 0 \rangle = $$
      $$ \langle a\to 1, b\to 1, c\to 0, d\to 0 \rangle.$$
\end{examp}
%--------------------------------------------------------------------------------------------------------------------------------------
\section{Conjunctive Normal Forms}
\label{sec:Conjunctive Normal Forms}

\begin{defi}\label{def:CNF}
      An important special case of propositional formulas are \textbf{propositional formulas in conjunctive normal form (CNF),} which are
      conjunctions of disjunctions of \textbf{literals,}  where a ``literal'' is a variable or a negated variable.  
\end{defi}
\begin{defi}\label{def:CNF-2}
      A CNF formula where each clause contains exactly $k$ literals is called \textbf{k-CNF}.
\end{defi}
\begin{examp}\label{exp:cnf}
      For Example, \begin{eqnarray*}
      &(a\vee b) \wedge (\neg a \vee c) \wedge (b\vee \neg c)& \\
      &a\wedge b \wedge c& \\
      &(\neg a \vee b) \wedge (\neg b \vee c) \wedge \neg c& \\
      &a \vee b \vee \neg c&
\end{eqnarray*} 
are conjunctive normal forms, while
\begin{eqnarray*}
      &a\vee (b \wedge c)&\\
      &\neg(a\vee b)& \\
      &a \wedge (b\vee (c\wedge d))&
\end{eqnarray*}
are not. 
\end{examp}
%--------------------------------------------------------------------------------------------------------------------------------------
\section{Transformation to CNF}
\label{sec:Transformation to CNF}

A propositional logic formula can be transformed to a logically equivalent CNF formula by using the following rules:
\begin{enumerate}
      \item $ a = b \Longleftrightarrow a \leftrightarrow b$.
      \item $a \leftrightarrow b \Longleftrightarrow (a \rightarrow b) \wedge (b \rightarrow a)}$.
      \item $ a \rightarrow b \Longleftrightarrow \neg a \vee b$.
	  \item $ a + b \Longleftrightarrow a \vee b$.
	  \item $ a . b \Longleftrightarrow a \wedge b$.
\end{enumerate}
\begin{examp}\label{exp:tocnf}
      Here is the example of transforming AND gate to CNF.
      $$ z = x . y$$
      $$ z  \leftrightarrow x . y$$
      $$(z \rightarrow x . y) . (x . y \rightarrow z)$$
      $$ ( \neg z \vee (x \wedge y)) \wedge (\neg (x \wedge y) \vee z)$$
      $$ (\neg z \vee x) \wedge ( \neg z \vee y) \wedge ( \neg x \vee \neg \vee z)$$
\end{examp}  
%--------------------------------------------------------------------------------------------------------------------------------------
\section{Satisfiability and Unsatisfiability}
\label{sec:Satisfiability and Unsatisfiability}

\begin{defi}\label{def:sat}
      $\bmm{\Sat} := \set{F \in \Cls \mb \ex\, \vp \in \Pass : \vp * F = \top}$ and $\bmm{\Usat} := \Cls \sm \Sat$; a partial assignment 
	  $\vp \in \Pass$ with $\vp * F = \top$ is called a \textbf{satisfying assignment} for $F \in \Cls$.
\end{defi}
Remarks:
\begin{enumerate}
      \item $\top \in \Sat$ and $\set{\bot} \in \Usat$.
      \item If $\bot \in F$, then $F \in \Usat$.
      \item If $F \in \Usat$, the $\vp * F \in \Usat$.
      \item If $F \in \Sat$ and $F' \sse F$, then also $F' \in \Sat$.
\end{enumerate}
The elements of $\Sat$ are called \mbox{\textbf{satisfiable clause-sets,}}
while the elements of  $\Usat$ are called \mbox{\textbf{unsatisfiable clause-sets.}}
A partial assignment $\varphi \in \Pass$ with $\varphi(F) = 1$ is called a \textbf{satisfying assignment} for $F$.
\begin{examp}\label{exp:sat1} The following clause-sets are $\Usat$:
      $$\left\{\{a\}, \{\overline{a}\}\right\}$$
      $$\left\{\{a,b\}, \{\overline{a},b\}, \{a, \overline{b}\}, \{\overline{a},\overline{b}\}\right\}$$
      $$\left\{\{a\}, \{\overline{a},b\}, \{\overline{a}, \overline{b}, c\}, \{\overline{a}, \overline{b}, \overline{c}\}\right\}$$
\end{examp}
\begin{examp}\label{exp:sat2}
      Some examples of satisfying assignment and  falsifying assignment are as folow.
      Falsifying assignment:
      $$\langle a\to & 0, b\to 1, c\to 0 \rangle * \left\{\{a,b,x\}, \{a,\overline{b},c\}, \{\overline{x}, \overline{y}\}, \{a,c,x\} \right\} = \left\{\bot, \{\overline{x}, \overline{y}\}, \{x\}\right\}$$
      and satisfying assignment:
      $$\langle x\to & 1, y\to 0, z\to 1 \rangle * \left\{\{x,y,\overline{z}\}, \{x,\overline{y},z\}, \{\overline{x}, \overline{y}\}, \{x,y\} \right\} = \top.$$
\end{examp}
??????????????????????????????????
\begin{defi}\label{def:hypergraphs}
  A \textbf{hypergraph} is a pair $(V,E)$, where $V$ is a set (the set of ``vertices''), while $E \sse \pote(V)$ is a set of finite subsets of $V$ (the ``hyperedges''). We use for a hypergraph $G = (V,E)$ the notations $\bmm{V(G)} := V$ and $\bmm{E(G)} := E$.
\end{defi}

%***************************************************************************************************************************************************************
%***************************************************************************************************************************************************************
\chapter{The SAT Problem and SAT Solvers}
\label{cha:SAT Problem and SAT Solvers}
%--------------------------------------------------------------------------------------------------------------------------------------
\section{The P versus NP Problem}
\label{sec:The P versus NP Problem}

\begin{defi}\label{def:np} 
      \textbf{The P versus NP problem} is to determine whether every language accepted by some nondeterministic algorithm in polynomial time is also accepted by some
      (deterministic) algorithm in polynomial time. To define the problem precisely it is necessary to give a formal model of a computer.
      The standard computer model in computability theory is the Turing machine, introduced by Alan Turing. Although the model was introduced before physical computers were built, it
      nevertheless continues to be accepted as the proper computer model for the purpose of defining the notion of computable function \cite{h4}.
\end{defi} 
An easy explanation according to \href{http://www.claymath.org/prizeproblems/pvsnp.htm}{Stephen Cook} is:

"Suppose that you are organizing housing accommodations for a group of four hundred university students. Space is limited and only one hundred of the students will receive places in the dormitory. 
To complicate matters, the Dean has provided you with a list of pairs of incompatible students, and requested that no pair from this list appears in your final choice (seems to be odd to demand 
that if student x gets a place, then student y doesn’t get a place, but this kind of condition is a very easy one). This is an example of what computer scientists call an \textbf{NP-problem}, 
since it is easy to check if a given choice of one hundred students proposed by a coworker is satisfactory, however the task of generating such a list from scratch seems to be so hard as to be 
completely impractical. Indeed, the total number of ways of choosing one hundred students from the four hundred applicants is greater than the number of atoms in the known universe!"
%--------------------------------------------------------------------------------------------------------------------------------------
\section{Binary Decision Diagram}
\label{sec:Binary Decision Diagram}

\begin{defi}\label{def:bdd} 
      \textbf{A Binary Decision Diagram (BDD)} is a data structure that is used to represent a boolean function. 
      On a more abstract level, BDDs can be considered as a compressed representation of sets or relations. 
      The basic idea from which the data structure was created is the \href{https://en.wikipedia.org/wiki/Boole%27s_expansion_theorem}{Shannon expansion}. 
	  A switching function is split into two sub-functions (cofactors) by assigning one variable. 
\end{defi} 
\begin{examp}\label{exp:bdd}
      As an BDD, consider the boolean function $f(v_1,v_2,v_3) := v_1 \implies (v_2 \iff v_3)$, with the truth table given in Fig \ref{fig:BooleanFunctionDTExampTT}, 
	  may be represented by either of the BDD shown in Fig \ref{fig:BooleanFunctionDTExampDT1} or Fig \ref{fig:BooleanFunctionDTExampDT2}. 
      $$f(v_1,v_2,v_3) = \{ \{ \overline{v_1},\overline{v_2}, v_3 \}, \{ \overline{v_1}, v_2, \overline{v_3}\} \}$$
      \begin{figure}[h]
            \centering
            \begin{tabular}{|c|c|c|c|} 
                  \hline
                  $v_1$ & $v_2$ & $v_3$ & $v_1 \implies (v_2 \iff v_3)$ \\ \hline
                  0 & 0 & 0 & 1 \\ \hline
                  0 & 0 & 1 & 1 \\ \hline
                  0 & 1 & 0 & 1 \\ \hline
                  0 & 1 & 1 & 1 \\ \hline
                  1 & 0 & 0 & 1 \\ \hline
                  1 & 0 & 1 & 0 \\ \hline
                  1 & 1 & 0 & 0 \\ \hline
                  1 & 1 & 1 & 1 \\ \hline
            \end{tabular}
            \caption{Truth table for boolean function $f$.}
            \label{fig:BooleanFunctionDTExampTT}
      \end{figure}
      \begin{figure}[h]
            \centering
            \begin{displaymath}
                  \xygraph{
                  []{v_1} ( 
                  - [dll]{1}_0 (),
                  - [drr]{v_2}^1 (
                  - [dll]{v_3}_0 (
                  - [dl]{1}_0 (),
                  - [dr]{0}^1 ()
                  ),
                  -[drr]{v_3}^1 (
                  - [dl]{0}_0 (),
                  - [dr]{1}^1 ()
                  )
                  )
                  )
                  }
            \end{displaymath}
            \caption{A BDD representation of $f$.}
            \label{fig:BooleanFunctionDTExampDT1}
      \end{figure}
      \begin{figure}[h]
            \centering
            \begin{displaymath}
                  \xygraph{
                 []{v_2} ( 
                 - [dlll]{v_3}_0 (
                 - [dl]{1}_0 (),
                 - [dr]{v_1}^1 (
                 - [dl]{1}_0 (),
                 - [dr]{0}_1 ()
                 )
                 ),
                 - [drrr]{v_3}^1 (
                 - [dl]{v_1}_0 (
                 - [dl]{1}_0 (),
                 - [dr]{0}^1 ()
                 ),
                 -[dr]{1}^1 ()
                 )
                 )
                 }
            \end{displaymath}
           \caption{Another BDD representation of $f$.}
           \label{fig:BooleanFunctionDTExampDT2}
      \end{figure}
\end{examp}
%--------------------------------------------------------------------------------------------------------------------------------------
%--------------------------------------------------------------------------------------------------------------------------------------
\section{The SAT Problem}
\label{sec:The SAT Problem}

\begin{defi}\label{def:sat} Basically, the meaning of \textbf{the SAT problem} is that for a conjunctive normal form $F$, decide whether $F$ is satisfiable or not.
      The SAT problem is one of the most versatile NP-complete problems. On the theoretical side, due to its expressiveness and flexibility it 
      serves as a tool for many results in complexity theory. On the practical side, NP-completeness of SAT is used “positively”, and many problems 
      are translated into  the “universal language” SAT and solved via SAT solvers. 
\end{defi}
%--------------------------------------------------------------------------------------------------------------------------------------
\section{Deficiency}
\label{sec:Deficiency}

A very interesting measure for the complexity of formulas is $CNF$ is the so-called "deficiency" which indicates the difference 
between the number of clauses and the number of variables.
\begin{defi}\label{def:df1}
      Let $F$ be a formula in $CNF$ with $n$ clauses and $k$ variables. Then, the \textbf{deficiency} of $F$ is defined as $\delta (F) = n - k$. 
	  Further, we define the maximal deficiency as $\delta ^*(F) = max \{ \delta (G) \mid G \subseteq F \}$ \cite{h6}.
\end{defi} 
Let $k-CNF$ be the set of $CNF$-formulas with deficiency (exactly) $k$. Then, the satisfiability problem for $k-CNF$ is NP-complete. 
Similarly, let $CNF ^* (k)$ be the set of formulas with maximal deficiency $k$, i.e. 
$\{F \in CNF : \delta ^*(F) = k \}$. It can been shown that the satisfiability problem for these classes is solvable in polynomial time \cite{h6}.

%--------------------------------------------------------------------------------------------------------------------------------------
\section{Backtracking for SAT}
\label{sec:Backtracking for SAT}

\begin{defi}\label{def:bsat} \textbf{Backtracking} is a general algorithm for finding all (or some) solutions to constraint satisfaction problems, 
      that incrementally builds candidates to the solutions, and abandons each partial candidate $c$ ("backtracks") as soon as it determines that c cannot 
      possibly be completed to a valid solution. 

      In general, a backtracking solver can be described as the following recursive procedure on input $F$:
      \begin{enumerate}
            \item First we try to simplify $F$ (using for example unit clause elimination as often as we can).
            \item Then, using simple criterions, we check whether we immediately see, whether $F$ is satisfiable or
            unsatisfiable, in which case we return "satisfiable" resp. "unsatisfiable".
            \item Otherwise some "branching variable" v in $F$ is chosen, and it is determined which truth value $\epsilon \in \{ 0, 1\}$ to consider first.
            \item Compute $\langle v \to \epsilon \rangle * F$, the result of setting variable v to value $\epsilon$, and apply the procedure recursively to
            $\langle v \to \epsilon \rangle * F$. If the result is "satisfiable" then we return "satisfiable".
            \item Otherwise, the second branch $\langle v \to \overline{\epsilon} \rangle * F$ has to be considered: If the result is “satisfiable” then
            “satisfiable” is returned, otherwise “unsatisfiable”.
      \end{enumerate}
\end{defi}
\begin{examp}\label{exp:bdd}
      If the number of variables in CNF $F$ is $n$, then we have $2^n$ assignments in order to find out whether $F$ is satisfiable or not.
      If $n$ is a large number, it would be very difficult to consider all the assignments and we need something more clever.
      The basic idea is to make case distinctions on variables, and to exploit simplifications enabled by these case distinctions.
      Here, we explain this idea using a simple example. Consider the $F = \{ \{c,b\}, \{\overline{c},b,a\}, \{\overline{b},c\}, \{\overline{b},c,a\}, \{{\overline{a},b}\} \}$. 
      %We have 3 variables and $2^3 = 8$ assignments in order to find out whether $F$ is satisfiable or not. If the number of variables is larger, 
      For the first case distinction, we consider $\langle a \to 0 \rangle$:
      $$ F_1 := \langle a \to 0 \rangle * F = \{ \{c,b\}, \{\overline{c},b\}, \{\overline{b},c\}, \{\overline{b},c\} \}$$
      Then, we split on $b$ and:
      $$ F_2 := \langle b \to 0 \rangle * F = \{ \{c\}, \{\overline{c}\} \}$$
      which is $\Usat$. So, we have to backtrack to the last case and consider $\langle b \to 1 \rangle$. We have:
      $$ F_3 := \langle b \to 1 \rangle * F = \{ \{c\}, \{\overline{c}\} \}$$
      which is again $\Usat$. So, we have to backtrack to the higher level, that is, we have to consider $\langle a \to 1 \rangle$:
      $$ F_4 := \langle a \to 1 * F = \{ \{b\} \}.$$
      And  now, if we consider $\langle b \to 1 \rangle$ the result is $\Sat$. It is obvious that this example was simple. But, if we have more variables, 
      sometimes we need to backtrack to the 2-higher level or more to investigate the satisfiablity of $F$.
\end{examp}
%--------------------------------------------------------------------------------------------------------------------------------------
\section{Methods to Address SAT}
\label{sec:Methods to Address SAT}

%***************************************************************************************************************************************************************
%***************************************************************************************************************************************************************
\chapter{Hardness Measures}
\label{cha:Hardness Measures}

\section{Implication-relation}
\label{sec:Implication-relation}

\begin{defi}\label{def:imp1}
      For $F, F' \in \Cls$ the \textbf{implication-relation} is defined as: 
      $F \models F' : \Leftrightarrow  \forall \varphi \in \Pass : \varphi * F = \top \Rightarrow \varphi * F' = \top$. 
      We write $F \models C$ for $F \models \{ C \}$ \cite{h5}.
\end{defi}
\begin{defi}\label{def:imp2}
      A clause $C$ with $F \models C$ is an \textbf{implicate} of $F$, while a \textbf{prime implicate}
      is an implicate $C$ such that no $C' \subset C$ is also an implicate; $\bmm{prc_0(F)}$ is the set of prime implicates of $F$ \cite{h5}.
\end{defi}
\begin{defi}\label{def:imp3}
      A boolean function $f$ is \textbf{monotone} iff flipping any input variable from 0 to 1 never flips the output from 1 to 0.
	  A boolean function $f$(or its corresponding CNF-clause-set $F$) is monotone iff f has only positive prime implicates \cite{h8}.
\end{defi}
\begin{examp}\label{exp:imp3}
      For $F = \{ \{a\}, \{ \overline a, b \}, \{ \overline a, \overline b, c \} \}$ we have:
      $$ prc_0(F) = \{ \{a\}, \{ b \}, \{ c \} \} $$
\end{examp}
\begin{examp}\label{exp:imp3}
      For the $F= \{ \{ \overline a, \overline b \}, \{ a, b \}, \{ a, c \}, \{ \overline b, c\}, \{ b, d \}, \{ c, d \}\}$ 
      ????????
	  %we have two $prc_0(F)$ as below:
      %$$ prc_0(F)=\{ \{ \overline a\}, \{ b\}, \{ c\} \} , $$
      %$$ prc_0(F)=\{ \{ a\}, \{ \overline b \}, \{ d\} \}$$
\end{examp}
\begin{examp}\label{exp:imp3}
      For the boolean function $ a \vee b$ we have $prc_0(a \vee b) = \{ \{a, b \} \}$, while for the boolean function 
	  $a \wedge b$ we have $prc_0(a \wedge b) = \{ \{ a \}, \{b\}\}$.
\end{examp}

Remarks:
\begin{enumerate}
      \item Some properties of prime implicates:
      \begin{enumerate}
	        \item $prc_0(0^V) = \{ \bot \}$ and $prc_0(1^V) = \top$.
			\item A clause-set $F \in \Cls$ is unsatisfiable iff $prc_0(F) = \{ \bot \}$, while for satisfiable $F$ a literal 
			$x$ is forced iff $\{ x \} \in prc_0(F)$.
			\item 
	  \end{enumerate}	 
      \item 
\end{enumerate}

%--------------------------------------------------------------------------------------------------------------------------------------
\section{Resolution}
\label{sec:Resolution}

\begin{defi}\label{def:Resolution}
 Two clauses $C,D$ are \textbf{resolvable} if $\mid C \cap \overline D \mid = 1$ , i.e., they clash in exactly one variable:
\begin{itemize}
 \item For two resolvable clauses $C$ and $D$ the \textbf{resolvent} $C \diamond D := (C \cup D) \setminus \{x, \overline x\} $ for $C \cap \overline D = \{ x \}$ 
 is the union of the two clauses minus the resolution literals.
 \item x is called the \textbf{resolution literal}, while var(x) is the \textbf{resolution variable}.
 \end{itemize}
 \end{defi}
 
 ??????
 It is guaranteed to derive the empty clause if the given CNF is unsatisfiable.
%***************************************************************************************************************************************************************
%***************************************************************************************************************************************************************

\chapter{The Conflict Matrix of Multi-clause-sets}
\label{cha:The Conflict Matrix of Multi-clause-sets}
%--------------------------------------------------------------------------------------------------------------------------------------
\section{Definitions}
\label{sec:Definitions}

%***************************************************************************************************************************************************************
%***************************************************************************************************************************************************************
\chapter{Autarkies}
\label{cha:Autarkies}
%--------------------------------------------------------------------------------------------------------------------------------------
\section{Autarkies}
\label{sec:Autarkies}

\begin{defi}\label{def:autarky1}
      A partial assignment $\vp \in \Pass$ is called an \textbf{autarky} for $F \in \Cls$ if $\fa\, C \in F : \vp * \set{C} \in \set{\set{C}, \top}$.
\end{defi}

Let $F$ be a clause-set. An autarky for $F$ is a partial assignment $\vp$ such that: $\fa C \in F : \var(\vp) \cap \var(C) \neq \es \Ra \vp * \set{C} = \top$ where $\vp * \set{C}$ is given by $\top$ when $\vp$ 
satisfies $\set{C}$, while otherwise $\vp * \set{C} = \set{C}$. 
Namely, whenever a clause is touched by a partial assignment $\vp$, then $\vp$ satisfies it.

\begin{defi}\label{def:week autarky}
      A \textbf{week autarky} ????????
\end{defi} 
\begin{lem}\label{lem:compaut}
      If $\vp, \psi$ are autarkies for $F$, then also the composition $\vp \circ \psi$ is an autarky for $F$.
\end{lem}
\pr If $\vp \circ \psi$ touches a clause $C \in F$ then either:
\begin{enumerate}
      \item $\psi$ touches $C$. Now, since $\psi$ is an autarky for $F, C$ is satisfied by $\vp \circ \psi$ since the assignments to variables in var($\psi$) 
	  do not  change and still exist in $\vp \circ \psi$. \item $\psi$ does not touch $C$ and $\vp$ therefore does touch C. Now, since $\vp$ is an autarky 
	  for $F$, C is satisfied by $\vp \circ \psi$ since the assignments to variables in $var(\vp) \cap var(C)$ for $\vp$ do not change in $\vp \circ \psi$. 
\end{enumerate}
Remarks:
\begin{enumerate}
      \item The empty partial assignment $\epa$ is an autarky for every $F \in \Cls$ (no clause is touched), and more generally all $\varphi \in \Pass$ with $\var(\varphi) \cap \var(F) = \emptyset$
      are autarkies for $F$, the textbf{trivial autarkies}. On the other end of the spectrum every satisfying assignment for $F$ (i.e., $\varphi * F = \top $) is an autarky for $F$ 
	  (every clause is satisfied). A literal $x \in \Lit$ is a pure literal for $F$ iff $\langle x \rightarrow 1 \rangle $ is an autarky for $F$ \cite{h9}.
\end{enumerate}
\begin{examp}\label{exp:bdd} ??????????????
\end{exam}
%----------------------------------------------------------------------------------------------------------------------
\section{Hitting clause-sets}
\label{sec:hitting}

\begin{defi}\label{def:hitting}
      A \textbf{hitting clause-set} is an $F \in \Cls$ such that all $C, D \in F$, $C \ne D$, have a clash, i.e, $C \cap \ol{D} \ne \es$ holds; 
	  the set of all hitting clause-sets is denoted by $\bmm{\Clash} \subset \Cls$, while $\bmm{\Uclash} := \Clash \cap \Usat$.
\end{defi}
Remarks:
\begin{enumerate}
      \item For $C, D \in \Cl$ the following properties are equivalent (recall Lemma \ref{lem:comcomp}):
      \begin{enumerate}
            \item $C, D$ clash, that is, there are clashing literals $x \in C$, $y \in D$.
            \item $C \cap \ol{D} \ne \es$.
            \item $\ol{C} \cap D \ne \es$.
            \item $C \cup D \ne \Cl$.
      \end{enumerate}
      \item $F \in \Cls$ is hitting iff every $C \in F$, as partial assignment, is a satisfying assignment for $F \sm \set{C}$.
\end{enumerate}
\begin{examp}\label{exp:hit1} 
      We have e.g. $\{1, 2,-3 \} \in \Cl$, while $\{-1, 1\} \not \in \Cl$. The only clause-set in $\Clash$ containing the empty clause is $\{ \bot \} \in \Clash$ . 
	  An example of a non-hitting clause-set is $\{ \{ 1, 2\}, \{-1, 2\}, \{3 \} \} \in \Cls \setminus \Clash$ , where we obtain an element of $\Clash$ if we add 
	  literal $-2$ to the third clause \cite{h9}.
\end{exam}

\begin{defi}\label{def:nearlyhitting}
      A \textbf{nearly-hitting clause-set} is an $F \in \Cls$, such that for all $C, D \in F$, $C \ne D$ either $\var(C) \cap \var(D) = \es$ holds or $C, D$ 
	  have a clash; the set of all nearly-hitting clause-sets is denoted by $\bmm{\Nclash} \subset \Cls$.
\end{defi}
Remarks:
\begin{enumerate}
      \item $F \in \Cls$ is nearly-hitting iff every $C \in F$, as partial assignment, is an autarky for $F \sm \set{C}$.
\end{enumerate}

\begin{quest}\label{que:decisionnearlyhitting}
      The complexity of SAT-decision for nearly-hitting clause-sets is open (also in the boolean case).
\end{quest}
%----------------------------------------------------------------------------------------------------------------------
\section{Minimally Unsatisfiable}
\label{sec:minsat}

\begin{defi}\label{def:minsat1}
      An unsatisfiable clause-set F is called \textbf{minimally unsatisfiable}, if for every clause $C \in F$ the clause-set $F \setminus \{C\}$ 
	  is satisfiable, and the set of minimally unsatisfiable clause-sets is denoted by $\bmm{\Musat} \subset \Usat$ . A clause-set $F \in \Musat$ is called
	  \textbf{saturated}, if replacing any $C \in F$ by any super-clause $C' \supset C$ yields a satisfiable clause-set, and the set of saturated 
	  minimally unsatisfiable clause-sets is denoted by $\Smusat \subset \Musat$.
\end{defi}
%\begin{examp}\label{exp:minsat1} 
%      The simplest element of $\Usat \setminus \Musat$ is $\{ \bot, \{ 1\} \}$, while the simplest element of $\Musat \setminus \Smusat$ is $\{ \{1, 2\}, \{-1\}, \{-2\} \}$ \cite{h9}.
%\end{exam}	  
%--------------------------------------------------------------------------------------------------------------------------------------
\newpage
\bibliography{my_references}
\bibliographystyle{plainurl}


\end{document}

